%!Cau!%
\begin{ex}%[Thi Thử Lần 2, THPT Lương Thế Vinh - Hà Nội, 2019]%[Dương BùiĐức, dự án 12EX6]%[1D2G1-3]
Có bao nhiêu cách phân tích số $15^9$ thành tích của ba số nguyên dương, biết rằng các cách phân tích mà các nhân tử chỉ khác nhau về thứ tự thì chỉ được tính một lần?
\choice
{$493$}
{$516$}
{$492$}
{\True $517$}
\loigiai{
Ta có $15^9=3^9\cdot 5^9$.\\
Ba nhân tử có dạng $3^{a_1}\cdot 5^{a_1}$, $3^{a_2}\cdot 5^{a_2}$, $3^{a_3}\cdot 5^{a_3}$ trong đó $a_1+a_2+a_3=b_1+b_2+b_3=9$ và $a_i, b_i\in\{0;\ 1; \ldots; 9\}$.\\
Chia ngẫu nhiên các bộ số mũ:
\begin{itemize}
\item chọn $a_1$ có $ 10 $ cách;
\item chọn $a_2$ có $10-a_1$ cách;
\item chọn $a_3$ có 1 cách.
\end{itemize}
Vậy có $\sum\limits_{a_1=0}^{10}(10-a_1)=55$ cách chọn các $a_i$ để $a_1+a_2+a_3=9$.\\
Tương tự, có 55 cách chọn các bộ $b_i$, suy ra có $55\cdot 55$ cách chia ngẫu nhiên các bộ số mũ.\\
Khi chia như trên thì xảy ra các trường hợp bộ số mũ trùng nhau hoặc các bộ số mũ chỉ là hoán vị của nhau.
\begin{enumerate}[1)]
\item Số cách chia mà cả ba bộ số mũ trùng nhau: chỉ có 1 cách, đó là $(3;3)$, $(3;3)$, $(3;3)$.
\item Số cách chia mà chỉ có hai bộ số mũ trùng nhau: có 5 cách để chọn $a_i=a_j$ và 5 cách để chọn $b_i=b_i$ (cùng bằng từ 0 đến 4), do đó có $5\cdot 5=25$ cách, tuy nhiên ta phải trừ đi 1 cách khi cả ba bộ trùng nhau, do đó còn $24$ cách. Mỗi cách này ta có 3 hoán vị.
\end{enumerate}
Do mỗi cách chọn mà ba bộ $(a_i, b_i)$ khác nhau ta có $3!=6$ hoán vị, theo đề bài, 6 hoán vị này chỉ được tính 1 lần, do vậy số các cách thỏa mãn đề bài là $k=\dfrac{55\cdot 55-3\cdot 24-1}{6}+24+1=517$.
}
\end{ex}