%!Cau!%
\begin{ex}%[HK2, THPT Nguyễn Huệ, Vĩnh Phúc, 2019]%[Thịnh Trần, dự án(12EX-5-2019)]%[1D3Y2-3]
	Cho dãy số $(u_n)$ với $u_n=\dfrac{k}{3^n}$ ($k$: hằng số). Khẳng định nào sau đây là \textbf{sai}?
	\choice
	{Số hạng thứ $5$ của dãy số là $\dfrac{k}{3^5}$}
	{\True Số hạng thứ $n$ của dãy số là $\dfrac{k}{3^{n+1}}$}
	{Là dãy số giảm khi $k > 0$}
	{Là dãy số tăng khi $k > 0$}
	\loigiai{
		\begin{itemize}
			\item Số hạng thứ $5$ của dãy là $u_5=\dfrac{k}{3^5}$.
			\item Số hạng thứ $n$ của dãy là $u_n=\dfrac{k}{3^n}$.
			\item Với mọi $n\in\mathbb{N}^*$ ta có $u_{n+1}-u_n=\dfrac{k}{3^{n+1}}-\dfrac{k}{3^n}=-\dfrac{2k}{3^{n+1}}$. Do đó dãy số $(u_n)$ giảm khi $k>0$ và tăng khi $k<0$.
		\end{itemize}
	Vậy khẳng định sai là \lq\lq Số hạng thứ $n$ của dãy số là $\dfrac{k}{3^{n+1}}$\rq\rq.
	}
\end{ex}%!Cau!%
\begin{ex}%[Đề Tập Huấn -4, Sở GD và ĐT - Hải Phòng, 2019]%[Trần Xuân Thiện, 12EX10]%[1D3Y2-3]
	Cho dãy số $ \left( u_n \right) $ với $ u_n = a \cdot 3^n$ ( $a$ : hằng số). Khẳng định nào sau đây là \textbf{sai}?
	\choice
	{Dãy số có $ u_{n + 1} = a \cdot 3^{n + 1} $}
	{\True Hiệu số $ u_{n + 1} - u_n = 3 \cdot a $}
	{Với $ a > 0 $ thì dãy số tăng}
	{Với $ a < 0 $ thì dãy số giảm}
	\loigiai{
		Ta có $ u_{n + 1} - u_n = a \cdot 3^{n + 1} - a \cdot 3^n = 2a \cdot 3^n $.
	}
\end{ex}