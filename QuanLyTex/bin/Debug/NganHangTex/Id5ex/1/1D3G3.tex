%!Cau!%
\begin{ex}%[Thi thử L1, Star Education HCM, 2019]%[Nguyễn Ngọc Dũng, dự án 12EX6]%[1D3G3-6]
Chọn ngẫu nhiên một số nhị phân có $10$ chữ số (các chữ số của số đó hoặc là 0 hoặc là $1$) có chữ số đầu tiên là $1$. Tính xác suất để số được chọn không có ba chữ số liên tiếp nào đều bằng $0$.
\choice
{$\dfrac{1}{2}$}
{$\dfrac{137}{256}$}
{\True$\dfrac {69}{128}$}
{$\dfrac{273}{512}$}
\loigiai{
Gọi $u_n$ là số các dãy nhị phân có $n$ số hạng thỏa yêu cầu đề bài.\\
Ta thấy các chữ số đầu tiên bên trái của số nhị phân đó phải có dạng 11; 101 hoặc 1001. \\
Nên $u_{n+1}= u_n + u_{n-1} + u_{n-2}, \forall n \geq 3$. \\
Ta tính trực tiếp \\
$u_1 = 1; \; u_2=2 ; \; u_3 =4 $. \\
Suy ra $u_4 =7 \Rightarrow u_5 =13; \; u_6=24; \; u_7=44; \; u_8 =81 ; \; u_9 =149$. \\
Suy ra $u_{10} = 274$. \\
Mà số phần tử của không gian mẫu là $\left| \Omega \right| =2^9$.\\
Nên xác suất $\mathrm{P}= \dfrac{274}{2^9} = \dfrac{137}{256}$.
}
\end{ex}