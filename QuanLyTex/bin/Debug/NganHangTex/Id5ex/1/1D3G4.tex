%!Cau!%
\begin{ex}%[Đề tập huấn tỉnh Lai Châu,2019]%[Nguyễn Trung Kiên, dự án 12-EX-5-2019]%[1D3G4-7]
	Cho tập $X=\{6;7;8;9\}$. Gọi $E$ là tập hợp các số tự nhiên có 2018 chữ số lập từ các chữ số của tập $X$. Chọn ngẫu nhiên một số trong tập $E$, tính xác suất để chọn được số chia hết cho $3$.
	\choice
	{\True $\dfrac{1}{3}\left(1+\dfrac{1}{2^{4035}}\right)$}
	{$\dfrac{1}{3}\left(1+\dfrac{1}{2^{2017}}\right)$}
	{$\dfrac{1}{3}\left(1+\dfrac{1}{2^{4036}}\right)$}
	{$\dfrac{1}{3}\left(1+\dfrac{1}{2^{2018}}\right)$}
	\loigiai
	{Gọi $A_n$ là tập hợp các số tự nhiên có $n$ chữ số lập từ các chữ số của $X$ và là số chia hết cho $3$; $B_n$ là tập hợp các số tự nhiên có $n$ chữ số lập từ các chữ số của $X$ và là số không chia hết cho $3$.\\
		Với mỗi số thuộc $A_n$, có hai cách thêm vào cuối một chữ số $6$ hoặc chữ số $9$ để được số thuộc $A_{n+1}$ và có hai cách để thêm một chữ số $7$ hoặc chữ số $8$ vào cuối để được số thuộc $B_{n+1}$.\\
		Với mỗi số thuộc $B_n$ có một cách thêm vào cuối một chữ số $7$ hoặc chữ số $8$ để được số thuộc $A_{n+1}$ và có ba cách thêm một chữ số $6$, $9$ hoặc $7$ hoặc $8$ vào cuối để được số thuộc $B_{n+1}$.\\
		Như vậy
		\allowdisplaybreaks
		\begin{eqnarray*}
			&&\heva{&n(A_{n+1})=2n(A_n)+n(B_n)\\&n(B_{n+1})=2n(A_n)+3n(B_n)} \Leftrightarrow \heva{&3n(A_{n+1})-n(B_{n+1})=4n(A_n)\\&n(A_{n+1})=2n(A_n)+n(B_n)}\\
			&\Rightarrow& \heva{&n(B_n)=3n(A_n)-4n(A_{n-1})\\&n(A_{n+1})=2n(A_n)+n(B_n)}
			\Rightarrow n(A_{n+1})=5n(A_n)-4n(A_{n-1}), \left(\forall n\in \mathbb{N}, n\geq 2\right).
		\end{eqnarray*}
		Suy ra $n(A_{n+2})=5n(A_{n+1})-4n(A_n), (\forall n\in\mathbb{N}^*)$\\
		Hay $n(A_{n+2})-n(A_{n+1})=4\left[n(A_{n+1})-n(A_n)\right], (\forall n\in\mathbb{N}^*)$.\\
		Ta có $n(A_1)=2$, $n(B_1)=2$, $n(A_2)=6$, $n(B_2)=10$, $n(A_3)=22$.\\
		Xét dãy số $(u_n)$ với $u_n=n(A_{n+1})-n(A_n), \forall n\in\mathbb{N}^*$. Ta có $\heva{&u_1=4\\&u_{n+1}=4u_n, \forall n\in \mathbb{N}^*.}$\\
		Dễ thấy $(u_n)$ là cấp số nhân với công bội $q=4$ nên $u_n=4^n, (\forall n\in\mathbb{N}^*)$.\\
		Ta được $n(A_{n+1})-n(A_n)=4^n \Leftrightarrow n(A_{n+1})-\dfrac{4^{n+1}}{3}=n(A_n)-\dfrac{4^n}{3}, (\forall n\in\mathbb{N}^*)$.\\
		Suy ra $n(A_n)-\dfrac{4^n}{3}=n(A_1)-\dfrac{4}{3} \Leftrightarrow n(A_n)=\dfrac{4^n+2}{3}, (\forall n\in\mathbb{N}^*)$.\\
		Số phần tử của tập $E$ là $4^{2018}$, trong đó có $\dfrac{4^{2018}+2}{3}$ số chia hết cho $3$. Vì vậy xác suất lấy ngẫu nhiêm một số của tập $E$ được số chia hết cho $3$ là $p=\dfrac{1}{3}\left(1+\dfrac{1}{2^{4035}}\right)$.}
\end{ex}%!Cau!%
\begin{ex}%[Thi thử, Sở GD và ĐT - Vĩnh Phúc, 2019]%[Đổ Viết Lân, 12EX9]%[1D3G4-7]
Cho tập $A=\{1;2;3;4;\ldots;100\}$ Gọi $S$ là tập các tập con của $A$, mỗi tập con này gồm $3$ phần tử và có tổng các phần tử bằng $91$. Chọn ngẫu nhiên một phần tử từ $S$. Tính xác xuất chọn được một tập hợp có ba phần tử lập thành cấp số nhân.
\choice
{$\dfrac{3}{645}$}
{\True $\dfrac{4}{645}$}
{$\dfrac{2}{1395}$}
{$\dfrac{1}{930}$}
\loigiai{
Gọi một phần tử của $S$ là $\{x;y;z\}$ với $x,y,z$ khác nhau và $x+y+z = 91$.\\
Mỗi cách chọn bộ ba số $\{x;y;z\}$ thỏa mãn $x+y+z = 91$ tương ứng 1-1 với một cách đặt hai chữ số $1$ vào giữa $91$ chữ số $0$ liên tiếp sao cho mỗi số $1$ nằm giữa hai số $0$. Có $\mathrm{C}_{90}^2=4005$ cách.\\
Bây giờ ta sẽ tìm trong $4005$ cách trên có bao nhiêu cách có hai số trong ba số $x;y;z$ trùng nhau. Giả sử $x=z$ ta có $2x+y = 91 \Rightarrow x = \dfrac{91 - y}{2}$. Do $x$ là số tự nhiên nên $y$ là $45$ số lẻ từ $1$ đến $89$. Vậy có tất cả $45\times 3 = 135$ trường hợp có hai số trùng nhau.\\
Do cách chọn phần tử của $S$ là một tập con của $A$ nên số phần tử của $S$ là $\dfrac{4005-135}{6} = 645$.
Gọi bộ ba số tạo thành cấp số nhân (tăng) là $a;aq;aq^2$ với $q = \dfrac{b}{c} > 1$ và $(b;c) = 1$.\\
Ta có $aq^2 = \dfrac{ab^2}{c^2}\in\mathbb{N}$ nên $a \vdots c^2 \Rightarrow a = mc^2\,(m\in\mathbb{N})$. Do đó ba số cần tìm là $mc^2;mbc;mb^2$ và $mc^2+mbc+mb^2=91$.\\
Ta có $91 = m(b^2+bc+c^2) \Rightarrow m \in \{1;7;13\}$.\\
Nếu $m=13$, ta có $b^2+c^2+bc=7 > 3c^2$. Suy ra $c=1$ và $b=2$. Vậy ta có bộ $\{13;26;52\}$.\\
Nếu $m=7$, ta có $b^2+c^2+bc=13 > 3c^2$. Suy ra $c=1$ và $b=3$. Vậy ta có bộ $\{7;21;63\}$.\\
Nếu $m=1$, ta có $b^2+c^2+bc=91 > 3c^2$. Suy ra $c\in\{1;2;3;4;5\}$.\\
Thay lần lượt $c$ ta có $\heva{&c=1\\&b=9}$ và $\heva{&c=5\\&b=6}$. Vậy ta có các bộ số $\{1;9;81\}$ và $\{25;30;36\}$.\\
Do đó xác suất cần tìm là $\dfrac{4}{645}$.
}
\end{ex}