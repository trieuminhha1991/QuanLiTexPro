%!Cau!%
\begin{ex}%[KSCL, Sở GD và ĐT - Thanh Hóa, 2018]%[Bùi Ngọc Diệp, 12EX-5]%[1H3G5-2]
	Cho tứ diện $ABCD$ có $AB = 3a$, $AC = a \sqrt{15}$, $BD = a \sqrt{10}$, $CD = 4a$. Biết rằng góc giữa đường thẳng $AD$ và mặt phẳng $(BCD)$ bằng $45^{\circ}$, khoảng cách giữa hai đường thẳng $AD$ và $BC$ bằng $\dfrac{5a}{4}$ và hình chiếu của $A$ lên mặt phẳng $(BCD)$ nằm trong tam giác $BCD$. Tính độ dài đoạn thẳng $AD$ biết rằng $AD > a$.
	\choice
	{$\dfrac{5a \sqrt{2}}{4}$}
	{\True  $2a$}
	{$2 \sqrt{2} a$}
	{$\dfrac{3a \sqrt{2}}{2}$}
	\loigiai{
		Ta chứng minh $AD \perp BC$. Thật vậy, xét tích vô hướng
		\begin{eqnarray*}
			\overrightarrow{AD} \cdot \overrightarrow{BC} = \overrightarrow{AD} \cdot \left( \overrightarrow{AC} - \overrightarrow{AB} \right) &= &  \overrightarrow{AD} \cdot \overrightarrow{AC} - \overrightarrow{AD} \cdot \overrightarrow{AB}\\
			&= & \dfrac{AD^2 + AC^2 - CD^2}{2} - \dfrac{AD^2 + AB^2 - BD^2}{2} \\
			&= & \dfrac{AC^2 + BD^2 - CD^2 - AB^2}{2} \\
			& = & \dfrac{15a^2 + 10a^2 - 16a^2 - 9a^2}{2} = 0 \Rightarrow AD \perp BC.
		\end{eqnarray*}
		\immini{Dựng $AH \perp (BCD)$ tại $H$ nằm trong tam giác $BCD$. Gọi $M$ là giao điểm của $DH$ và $BC$ suy ra $M$ nằm giữa $B$ và $C$. \\
			Do $\heva{& BC \perp AH \\& BC \perp AD} \Rightarrow BC \perp (AHD) \Rightarrow BC \perp DM$. \\
			Trong mặt phẳng $(ADM)$, dựng $MN \perp AD$ tại $N$ \\ $\Rightarrow \heva{& MN \perp BC \\& MN \perp AD} \Rightarrow MN$ là đoạn vuông góc chung của $AD$ và $BC$ $\Rightarrow MN = \dfrac{5a}{4}$. }
		{\begin{tikzpicture}[scale=0.7, line join = round, line cap = round]
			\tikzset{label style/.style={font=\footnotesize}}
			\tkzDefPoints{0/0/B,7/0/D,2/-3/C,3/5/A}
			\coordinate (H) at ($(A)+(0,-6)$);
			\tkzInterLL(D,H)(B,C)
			\tkzGetPoint{M}
			\coordinate (N) at ($(A)!1/3!(D)$);
			\tkzDrawPolygon(A,B,C,D)
			\tkzDrawSegments(A,C A,M)
			\tkzDrawSegments[dashed](B,D D,M B,N M,N A,H)
			\tkzDrawPoints(A,B,C,D,H,M,N)
			\tkzLabelPoints[above](A)
			\tkzLabelPoints[below](C,H)
			\tkzLabelPoints[left](B,M)
			\tkzLabelPoints[right](D,N)
			\end{tikzpicture}}
		$\widehat{ADH} = 45^{\circ} $ là góc giữa $AD$ và mặt phẳng $(BCD)$. \\
		Ta có $DM = MN \cdot \sqrt{2} = \dfrac{5a \sqrt{2}}{4} \Rightarrow BM = \sqrt{BD^2 - DM^2} = \dfrac{a \sqrt{10}}{4}$. \\
		$AN = \sqrt{AB^2 - BN^2} = \sqrt{AB^2 - \left( BM^2 + MN^2 \right)} = \sqrt{9a^2 - \left( \dfrac{110a^2}{16} + \dfrac{25a^2}{16} \right)} = \dfrac{3a}{4}$. \\
		$DN = MN = \dfrac{5a}{4}$. \\
		Nếu $N$ nằm giữa $A$ và $D$ thì $AD = AN + DN = 2a$. \\
		Nếu $A$ nằm giữa $N$ và $D$ thì $AD = DN - AN = \dfrac{a}{2}$ (loại).}
\end{ex}%!Cau!%
\begin{ex}%[Thi thử, Sở GD và ĐT -Lạng Sơn, 2019]%[Trần Duy Khương, 12EX5-2019]%[1H3G5-4]
	Cho hình lập phương $ABCD.A’B’C’D’$ cạnh a . Gọi $M, N$ lần lượt là trung điểm của	$AC$ và $B'C'$. Khoảng cách giữa hai đường thẳng $MN$ và $B'D'$ bằng 
	\choice
	{$a\sqrt{5}$}
	{$\dfrac{a\sqrt{5}}{5}$}
	{$3a$}
	{\True $\dfrac{a}{3}$}
	\loigiai{\immini{Gọi $O,N,P$ lần lượt là trung điểm các cạnh $B'D',BC', C'D'$. Vì $B'D'\parallel NP$ nên $$\mathrm{d}(B'D',MN)=\mathrm{d}(B'D',(MNP))=\mathrm{d}(O,(MNP)).$$ Tứ diện $O.MNP$ có $OM,ON,OP$ đôi một vuông góc, do đó $$\dfrac{1}{\mathrm{d}(O,(MNP))^2}=\dfrac{1}{OM^2}+\dfrac{1}{ON^2}+\dfrac{1}{OP^2}$$ $\Rightarrow \mathrm{d}(O,(MNP))=\dfrac{a}{3}$. Vậy $\mathrm{d}(B'D',MN)=\dfrac{a}{3}$.}{\begin{tikzpicture}[>=stealth,scale=1.2, line join = round, line cap = round]
			\tkzDefPoints{0/0/A',3/0/B', 1/1/D', 4/1/C', 0/3/A, 3/3/B, 1/4/D, 4/4/C}
			\coordinate (M) at ($(A)!1/2!(C)$);
			\coordinate (N) at ($(B')!1/2!(C')$);
			\coordinate (O) at ($(B')!1/2!(D')$);
			\coordinate (P) at ($(C')!1/2!(D')$);
			\tkzDrawSegments(A,D D,C A,B B,C A,A' B,B' C,C' A',B' B',C' A,C)
			\tkzDrawSegments[dashed](A',D' C',D' D,D' B,D' A,D' M,O O,N O,P M,P N,P M,N C',D' B',D')
			\tkzDrawPoints[fill=black](A,B,C,D,A',B',C',D',M,N,O,P)
			\tkzLabelPoints[above](C,D)
			\tkzLabelPoints[above  right](P)
			\tkzLabelPoints[below](A',B',D',N,O)
			\tkzLabelPoints[left](A)
			\tkzLabelPoints[right](C',B,M)
			\end{tikzpicture}
	}}
\end{ex}%!Cau!%
\begin{ex}%[Đề tập huấn Sở Ninh Bình, 2019]%[Nguyễn Văn Hải, dự án(12EX-5-2019)]%[1H3G5-4]
	Cho hình chóp đều $S.ABC$ có cạnh đáy bằng $a\sqrt{6}$, khoảng cách giữa hai đường thẳng $SA$ và $BC$ bằng $\dfrac{3a}2$. Tính thể tích khối chóp $S.ABC$.	
	\choice
	{\True $\dfrac{a^3\sqrt{6}}2$}
	{$\dfrac{a^3\sqrt{6}}{8}$}
	{$\dfrac{a^3\sqrt{6}}{12}$}
	{$\dfrac{a^3\sqrt{6}}{4}$}
	\loigiai{
	\immini{Gọi $F$ là trung điểm của $BC$, $G$ là hình chiếu vuông góc của $F$ trên $SA$.\\
	Khi đó $BC \perp (SAF) \Rightarrow BC \perp FG$ hay $FG$ là đường vuông góc chung của hai đường thẳng chéo nhau $SA$ và $BC$.\\
	Vì $S.ABC$ là hình chóp đều nên khoảng cách giữa hai đường thẳng $SA$ và $BC$ là độ dài đoạn $FG=\dfrac{3a}2$.\\
	Mà $FA$ là đường cao của tam giác đều cạnh bằng $a\sqrt{6}$ nên $FA = \dfrac{a\sqrt{6} \cdot \sqrt{3}}2 = \dfrac{3a\sqrt2}2$.
	}
	{
	\begin{tikzpicture}[scale=1, font=\footnotesize, line join=round, line cap=round, >=stealth]
		\tkzDefPoint(0,0){B}
		\tkzDefShiftPoint[B](0:5){A}
		\tkzDefShiftPoint[B](-60:2){C}
		\tkzDefShiftPoint[B](60:4){S}
		\tkzDefMidPoint(S,A) \tkzGetPoint{G}
		\tkzDefMidPoint(B,C) \tkzGetPoint{F}
		\tkzDefMidPoint(B,A) \tkzGetPoint{N'}
		\tkzInterLL(A,F)(C,N') \tkzGetPoint{H}	
		\tkzDrawSegments(S,B S,C S,A B,C C,A S,F)
		\tkzDrawSegments[dashed](B,A B,H C,H A,F G,F S,H)
		\tkzLabelPoint[above](S){\footnotesize $S$}
		\tkzLabelPoint[left](B){\footnotesize $B$}
		\tkzLabelPoint[left](C){\footnotesize $C$}
		\tkzLabelPoint[right](A){\footnotesize $A$}
		\tkzLabelPoint[below](H){\footnotesize $H$}
		\tkzLabelPoint[right](G){\footnotesize $G$}
		\tkzLabelPoint[left](F){\footnotesize $F$}
		\tkzDrawPoints(S,A,B,C,F,G,H)
		\end{tikzpicture}
		}
		\noindent Từ đó suy ra $AG = \sqrt{FA^2 - FG^2} = \sqrt{\left(\dfrac{3a\sqrt2}2 \right)^2 - \left(\dfrac{3a}2\right)^2} = \dfrac{3a}2$.\\
	Như vậy tam giác $AGF$ vuông cân tại $G$.\\
	Suy ra tam giác $SHA$ vuông cân tại $H$.\\
	Do đó $SH = AH = \dfrac2{3} AF = \dfrac2{3} \cdot \dfrac{3a\sqrt2}2 = a\sqrt2$.\\
	Vậy thể tích của khối chóp $S.ABC$  là $V = \dfrac{1}{3} SH \cdot S_{\Delta ABC} = \dfrac{1}{3} \cdot a\sqrt2 \cdot \dfrac{(a\sqrt{6})^2 \sqrt{3}}{4} = \dfrac{a^3 \sqrt{6}}2$.
	}
\end{ex}%!Cau!%
\begin{ex}%[Đề tập huấn số 2, Sở GD và ĐT Quảng Ninh, 2019]%[Đỗ Đường Hiếu, 12EX5-19]%[1H3G5-3]
	Cho hình lăng trụ đứng $ABC.A'B'C'$ có $AB=1$, $AC=2$, $AA'=3$ và $\widehat{BAC}=120^\circ$. Gọi $M$, $N$ lần lượt là các điểm trên cạnh $BB'$, $CC'$ sao cho $BM=3B'M$, $CN=2C'N$. Tính khoảng cách từ điểm $M$ đến mặt phẳng $(A'BN)$.
	\choice
	{\True $\dfrac{9\sqrt{138}}{184}$}
	{$\dfrac{3\sqrt{138}}{46}$}
	{$\dfrac{9\sqrt{3}}{16\sqrt{46}}$}
	{$\dfrac{9\sqrt{138}}{46}$}
	\loigiai{
		\begin{center}
			\begin{tikzpicture}[line cap=round, line join=round,font=\footnotesize,>=stealth, scale=1]
			\tikzset{label style/.style={font=\footnotesize}}
			\tkzDefPoints{0/0/A, -1/-1.5/B, 4/-1/C, 0/3/A'}
			\tkzDefPointBy[translation = from A to A'](B)\tkzGetPoint{B'}
			\tkzDefPointBy[translation = from A to A'](C)\tkzGetPoint{C'}
			\coordinate (N) at ($(C)!2/3!(C')$);
			\coordinate (M) at ($(B)!3/4!(B')$);
			\tkzInterLL(B,N)(B',C')\tkzGetPoint{D}
			\coordinate (E) at ($(A')!-1/2!(D)$);
			\coordinate (H) at ($(E)!1/3!(B)$);
			%\tkzDefMidPoint(A,B)\tkzGetPoint{M}
			\tkzDrawSegments[dashed](A,C A,B A,A' A',N)
			\tkzDrawSegments(B,C A',B' B',C' C',A' B,B' C,C' B,D D,E B',E B,E C',D B',H)
			
			\tkzDrawPoints[fill=black](A,B,C,A',B',C',N,M,D,E,H)
			\tkzLabelPoints[above](A',B',C',D,E)
			\tkzLabelPoints[below](A,B,C,H)
			\tkzLabelPoints[right](M,N)
			\tkzMarkRightAngles(B',E,A B',H,B)
			\end{tikzpicture}
		\end{center}
		Ta có $BC^2=AB^2+AC^2-2\cdot AB\cdot AC\cos \widehat{BAC}=1^2+2^2-2\cdot 1\cdot 2\cos 120^\circ =7$. Suy ra $BC=\sqrt{7}$.\\
		Ta cũng có $\cos \widehat{ABC}=\dfrac{AB^2+BC^2-AC^2}{2\cdot AB\cdot BC}=\dfrac{1^2+\sqrt{7}^2-2^2}{2\cdot 1\cdot \sqrt{7}}=\dfrac{2}{\sqrt{7}}$, suy ra $\cos \widehat{A'B'C'}=\dfrac{2}{\sqrt{7}}$.\\
		Gọi $D=BN\cap B'C'$, suy ra $\dfrac{DC'}{DB'}=\dfrac{C'N}{BB'}=\dfrac{1}{3}$, nên $DB'=\dfrac{3}{2}B'C'=\dfrac{3\sqrt{7}}{2}$.\\
		Từ đó ta có $A'D^2=A'B'^2+B'D^2-2\cdot A'B'\cdot B'D\cos \widehat{A'B'D}=1^2+\left(\dfrac{3\sqrt{7}}{2}\right)^2-2\cdot 1\cdot \dfrac{3\sqrt{7}}{2}\cdot\dfrac{2}{\sqrt{7}}=\dfrac{43}{4}$. Suy ra $A'D=\dfrac{\sqrt{43}}{2}$.\\
		Kẻ $B'E\perp A'D$ và $B'H\perp BE$, suy ra $B'H\perp (A'BN)$. 
		Do đó $\mathrm{d}\left(B',(A'BN)\right)=B'H$.\\
		Từ $\cos \widehat{A'B'C'}=\dfrac{2}{\sqrt{7}}\Rightarrow \sin \widehat{A'B'C'}=\dfrac{\sqrt{3}}{\sqrt{7}}$. \\
		Do đó
		$S_{A'B'D}=\dfrac{1}{2}\cdot A'B'\cdot B'D\cdot \sin \widehat{A'B'D}=\dfrac{1}{2}\cdot 1\cdot \dfrac{3\sqrt{7}}{2}\cdot \dfrac{\sqrt{3}}{\sqrt{7}}=\dfrac{3\sqrt{3}}{4}$.\\
		$B'E=\dfrac{2S_{A'B'D}}{A'D}=\dfrac{2\cdot\dfrac{3\sqrt{3}}{4} }{\dfrac{\sqrt{43}}{2}}=\dfrac{3\sqrt{3}}{\sqrt{43}}$.\\
		$\dfrac{1}{B'H^2}=\dfrac{1}{B'E^2}+\dfrac{1}{B'B^2}=\dfrac{1}{\left(\dfrac{3\sqrt{3}}{\sqrt{43}}\right)^2}+\dfrac{1}{3^2}=\dfrac{46}{27}\Rightarrow B'H=\sqrt{\dfrac{27}{46}}$.\\
		Từ $BM=3B'M$ suy ra $\mathrm{d}\left(M,(A'BN)\right)=\dfrac{3}{4}\mathrm{d}\left(B',(A'BN)\right)=\dfrac{3}{4}\cdot B'H=\dfrac{3}{4}\cdot \sqrt{\dfrac{27}{46}}=\dfrac{9\sqrt{138}}{184}$.
	}
\end{ex}%!Cau!%
\begin{ex}%[Hải Phòng, 2018]%[Phan Anh Tiến, 12-EX-05]%[1H3G5-4]
	Cho hình chớp $S.ABCD$ có đáy là hình vuông $ABCD$, $SAD$ là tam giác đều và nằm trong mặt phẳng vuông góc với mặt phẳng đáy. Biết rằng diện tích mặt cấu ngoại tiếp của khối chóp $S.ABCD$ là $4\pi$. Khoảng cách giữa hai đường thẳng $SD$ và $AC$ gần với giá trị nào sau đây nhất?
	\choice
	{$ \dfrac{2}{7}$}
	{$ \dfrac{3}{7} $}
	{\True $ \dfrac{6}{7} $}
	{$ \dfrac{4}{7} $}
	\loigiai{\immini{Gọi $x>0$ là cạnh hình vuông và $H$ là trung điểm $AD$, vì $SAD$ đều nên $SH\perp AD$.\\
			Mặt khác, $(SAD)\perp (ABCD)$ nên $SH\perp (ABCD)$. Khi đó $SH=\dfrac{x\sqrt{3}}{2}$.\\
			Gọi $O$, $G$ lần lượt là tâm đường tròn $ABCD$, $SAD$.\\ $d_1$, $d_2$ lần lượt là hai trục của hai đường tròn $ABCD$, $SAD$ (với $d_1$ qua $O$ song song $SH$, $d_2$ qua $G$ song song $OH$). Gọi $I$ là giao điểm của $d_1$ và $d_2$ khi đó, $I$ là tâm mặt cầu ngoại $S.ABCD$.\\
			
		}
		{\begin{tikzpicture}[scale=0.7, line join = round, line cap = round]
			\tikzset{label style/.style={font=\footnotesize}}
			\tkzDefPoints{0/0/D,6/0/C,2.5/2/A}
			\coordinate (B) at ($(A)+(C)-(D)$);
			\coordinate (H) at ($(A)!0.5!(D)$);
			\coordinate (S) at ($(H)+(0,6)$);
			
			\tkzDrawPolygon(S,B,C,D)
			\tkzDrawSegments(S,C)
			\coordinate (O) at ($(A)!0.5!(C)$);
			\coordinate (G) at ($(S)!2/3!(H)$);
			\coordinate (I) at ($(O)+(0,2)$);
			\tkzDrawSegments[dashed](A,S A,B A,D S,H A,C B,D G,I I,O)
			\tkzDrawPoints(D,C,A,B,H,S,O,G,I)
			\tkzLabelPoints[above](S,I)
			\tkzLabelPoints[below](A,D,C,O)
			\tkzLabelPoints[left](H,G)
			\tkzLabelPoints[right](B)
		\end{tikzpicture}
}
	Ta có $R=SI=\sqrt{SG^2+GI^2}$, mà $S=4\pi R^2\Rightarrow R=1$ hay $x=\dfrac{2\sqrt{21}}{7}$.\\
	Dựng hình bình hành $ADEF$, kẻ $HK\perp ED$ và $HP\perp SK$ suy ra $HP\perp (SED)$.\\
	Ta có $AC\parallel (SED)$ nên $\mathrm{d}(AC,SD)=\mathrm{d}(AC,(SDE))=\mathrm{d}(A,(SDE))=2\mathrm{d}(H,(SDE))=2HP$.\\
	Mà $\dfrac{1}{HP^2}=\dfrac{1}{SH^2}+\dfrac{1}{KH^2}\Rightarrow HP=\dfrac{3}{7}$. Vậy $\mathrm{d}(AC,SD)=\dfrac{6}{7}$.
		}
\end{ex}%!Cau!%
\begin{ex}%[HK2, THPT Nguyễn Huệ, Vĩnh Phúc, 2019]%[Thịnh Trần, dự án(12EX-5-2019)]%[1H3G5-4]
	Cho hình lập phương  $ABCD.A'B'C'D'$ cạnh $a$. Gọi $M$, $N$ lần lượt là trung điểm của $BC$ và $DD'$. Tính theo $a$ khoảng cách giữa hai đường thẳng $MN$ và $BD$.
	\choice
	{$\sqrt{3}a$}
	{$\dfrac{\sqrt{3}a}{2}$}
	{$\dfrac{\sqrt{3}a}{3}$}
	{\True $\dfrac{\sqrt{3}a}{6}$}
	\loigiai{
		\immini{
			Gọi $O$, $O'$ lần lượt là tâm hình vuông $ABCD$ và $A'B'C'D'$.
			Gọi $P$, $Q$ lần lượt là trung điểm của $CD$, $BB'$, ta có $MP\parallel NQ \parallel BD$. Mặt khác
			$BD\perp (AA'C'C)$ nên $MP\perp (AA'C'C)$.\\
			Gọi $I$, $J$ lần lượt là giao điểm của $MP$ và $AC$, $OO'$ và $NQ$. Ta có $(AA'C'C)$ cắt $(MPNQ)$ theo giao tuyến $IJ$. Ta tính được $OI=\dfrac{a\sqrt{2}}{4}$, $OJ=\dfrac{a}{2}$.\\
			Kẻ $OH \perp IJ$ tại $H$ suy ra $OH\perp (MPNQ)$.\\	
			$\triangle OIJ$ vuông tại $O$ nên $\dfrac{1}{OH^2}=\dfrac{1}{OI^2}+\dfrac{1}{OJ^2}=\dfrac{12}{a^2}$\\
			$\Rightarrow OH=\dfrac{a\sqrt{3}}{6}$.\\
		}{
			\begin{tikzpicture}[scale=1, font=\footnotesize, line join=round, line cap=round, >=stealth]
			\tikzset{label style/.style={font=\footnotesize}}
			\tkzDefPoints{0/0/C',4.3/0/D',1/1/B'}
			\coordinate (A') at ($(B')+(D')-(C')$);
			\coordinate (M) at ($(C)!0.5!(B)$);
			\coordinate (N) at ($(D)!0.5!(D')$);
			\coordinate (P) at ($(C)!0.5!(D)$);
			\coordinate (Q) at ($(B)!0.5!(B')$);
			\coordinate (O) at ($(B)!0.5!(D)$);
			\coordinate (O') at ($(B')!0.5!(D')$);
			\coordinate (I) at ($(C)!0.5!(O)$);
			\coordinate (J) at ($(O')!0.5!(O)$);
			\coordinate (H) at ($(I)!0.25!(J)$);
			
			\tkzMarkRightAngles[size=0.2,fill=gray!20](C,O,O' O,H,J)
			\tkzDefSquare(C',D')    \tkzGetPoints{D}{C}
			\tkzDefSquare(B',A')    \tkzGetPoints{A}{B}
			\tkzDrawPolygon(A,B,C,C',D',A')
			\tkzDrawSegments(D,D' D,A P,M P,N C,D D,B C,D C,A)
			\tkzDrawSegments[dashed](B',B C',B' M,N B',A' Q,N M,Q O,O' B,B' C',A' I,J O,H)
			\tkzDrawPoints[fill=black](A',B',C',D',C,D,B,A,M,N,P,Q,O,O',I,J,H)
			\tkzLabelPoints[above](A,B,P,O,I)
			\tkzLabelPoints[below](D',C',O')
			\tkzLabelPoints[right](A',D,N)
			\tkzLabelPoints[left](C,B',M,Q)
			\tkzLabelPoints[below left](J,H)
			\end{tikzpicture}	
		}
		\noindent 
		Vì $BD \parallel MP$ nên $BD \parallel (MNP)$.\\
		Vậy $\mathrm{d}\left(BD, MN\right)=\mathrm{d}\left(BD, (MNP)\right)	=\mathrm{d}\left(O, (MNP)\right)=OH=\dfrac{a\sqrt{3}}{6}$.
	}
\end{ex}%!Cau!%
\begin{ex}%[Đề Tập Huấn -4, Sở GD và ĐT - Hải Phòng, 2019]%[Trần Xuân Thiện, 12EX5]%[1H3G5-4]
	Cho hình chóp $S.ABCD$ có đáy $ABCD$ là hình vuông, $SAD$ là tam giác đều và nằm trong mặt
	phẳng vuông góc với mặt phẳng đáy. Biết rằng diện tích mặt cầu ngoại tiếp khối chóp $S.ABCD$ là $4\pi$. Khoảng cách giữa hai đường thẳng $SD$ và $AC$ gần với giá trị nào nhất sau đây?
	\choice
	{$ \dfrac{2}{7} $ dm}
	{$ \dfrac{3}{7} $ dm}
	{$ \dfrac{4}{7} $ dm}
	{\True $ \dfrac{6}{7} $ dm}
	\loigiai{
		\immini[0.05]{		
	Gọi $ x > 0 $ là cạnh của hình vuông $ ABCD $ và $ H $ là trung điểm cạnh $ AD $.\\
	- Do $ \Delta SAD$ đều nên $ SH \perp AD$.\\
	- Mặt khác $ (SAD) \perp (ABCD)$ theo giao tuyến $ AD $ nên $ SH \perp (ABCD)$.\\
	Ta có $ SH = \dfrac{x\sqrt{3}}{2}$.\\
	}{	
			\begin{tikzpicture}[scale=0.6, line join = round, line cap = round]
			\tikzset{label style/.style={font=\footnotesize}}
			\tkzDefPoints{0/0/D,7/0/C,3/3/A}
			\coordinate (B) at ($(A)+(C)-(D)$);
			\coordinate (H) at ($(A)!0.5!(D)$);
			\coordinate (S) at ($(H)+(0,7)$);
			\coordinate (G) at ($(S)!0.7!(H)$);
			\tkzDefPointBy[translation = from A to C](D)
			\tkzGetPoint{E}
			\tkzDefPointBy[homothety = center E ratio 1.3](D)
			\tkzGetPoint{K}
			\tkzDefPointBy[projection= onto S--K](H)
			\tkzGetPoint{P}
			\tkzInterLL(A,C)(B,D)
			\tkzGetPoint{O}
			\tkzDefPointBy[translation = from H to O](G)
			\tkzGetPoint{I}
			\tkzDefPointBy[homothety = center G ratio 1.7](I)
			\tkzGetPoint{M}
			\tkzDefPointBy[homothety = center O ratio 2.9](I)
			\tkzGetPoint{N}
			\tkzDrawPolygon(S,B,C,D)
			\tkzDrawSegments(S,C D,E C,E D,K S,K I,M I,N)
			\tkzDrawSegments[dashed](A,S A,B A,D S,H A,C B,D H,O H,K H,P G,I O,I)
			\tkzDrawPoints(D,C,A,B,H,S,G,E,K,P,I)
			\tkzLabelPoints[above](S)
			\tkzLabelPoints[above right](I)
			\tkzLabelPoints[below](A,D,C,O)
			\tkzLabelPoints[below right](H)
			\tkzLabelPoints[left](G,K,P)
			\tkzLabelPoints[right](B,E)
			\tkzMarkRightAngle(H,P,K)
			\tkzMarkRightAngle(G,H,O)
			\tkzMarkRightAngle(I,O,H)
			\tkzLabelSegment[right](I,N){$d_1$}
			\tkzLabelSegment(I,M){$d_2$}
			\end{tikzpicture}	
	}	
+ Gọi $O = AC \cap BD$ và $ G $ là trọng tâm $\Delta SAD $ thì $O$, $G$ lần lượt là tâm của đường tròn ngoại tiếp hình vuông $ ABCD $ và tam giác $ SAD $, đồng thời $d_1, d_2 $ lần lượt là $2$ trục đường tròn ngoại tiếp hình vuông $ ABCD $ và tam giác $SAD$ ($ d_1 $ qua $O$ và song song $ SH $, $d_1$ qua $G$ và song song $OH$).\\
$\Rightarrow I = d_1 \cap d_2 $ là tâm mặt cầu ngoại tiếp khối chóp $ S.ABCD $.\\
$\Rightarrow R = SI = \sqrt{SG^2 + GI^2}$.\\
+ Ta lại có $ S = 4 \pi R^2 \Rightarrow R = 1 = \sqrt{\left(\dfrac{x}{\sqrt{3}}\right)^2 + \left(\dfrac{x}{2}\right)^2} \Rightarrow x = \dfrac{2\sqrt{21}}{7}$ dm.\\
+ Dựng hình bình hành $ADEC$.\\
- Kẻ $HK \perp ED$ và $ HP \perp SK \Rightarrow HP \perp (SED)$.\\
Ta có $ AC \parallel (SED)$.\\
$\Rightarrow d(AC; SD) = d(AC; (SED)) = d(A; (SED)) = 2d(H; (SED)) = 2HP$.\\
Do $\Delta SHK $ vuông tại $H$ nên $ \dfrac{1}{HP^2} = \dfrac{1}{SH^2} + \dfrac{1}{KH^2} \Rightarrow HP = \dfrac{3}{7}$ dm.	
	}
\end{ex}%!Cau!%
\begin{ex}%[Phát triển đề LTV, lần 3, 2019; Mai Sương]%[1H3G5-4]
	Cho hình chóp $S.ABC$ có đáy là tam giác vuông cân tại $B, AB=a$. Gọi $M$ là trung điểm của $AC$. Biết hình chiếu vuông góc của $S$ lên mp$(ABC)$ là điểm $N$ thỏa mãn $\vec{BM}=3\vec{MN}$ và góc giữa hai mặt phẳng $(SAB)$ và $(SBC)$ là $60^\circ$. Tính khoảng cách giữa hai đường thẳng $AB$ và $SM$ theo $a$.
	\choice
	{$\dfrac{\sqrt{17}a}{51}$}
	{$\dfrac{\sqrt{17}a}{34}$}
	{\True $\dfrac{2\sqrt{17}a}{17}$}
	{$\dfrac{\sqrt{17}a}{68}$}
	\loigiai{
		\immini{
			Trong mp$(SBC)$, dựng $CK\perp SB$, $(K \in SB)$. \\
			Ta có\\
			$\heva {&BM\perp AC \\&SN\perp AC} \Rightarrow AC\perp (SBN) \Rightarrow AC \perp SB$. \\
			$\heva {&SB\perp AC \\&SB\perp CK} \Rightarrow SB \perp (ACK) \Rightarrow SB \perp AK$. \\
			Mặt khác $MK \perp SB$, $MK < MB$, mà $MB=MC$ nên $\widehat{MKC} > 45^{\circ}$.\\
			Tương tự $\widehat{MKA} > 45^{\circ}$.\\
			Suy ra \\
			$\left((SAB),(SBC)\right)=(AK,KC)=60^\circ \\ \Rightarrow \heva{&\widehat{AKC}=120^\circ\\ &\widehat{MKC}=\widehat{MKA}=60^\circ .}$\\
			Gọi $L$ là trung điểm $BC$, kẻ $NH\perp ML$ tại $H$, $NE\perp SH$ tại $E$.
		}{
			\begin{tikzpicture}[scale=1.2, line join=round, line cap=round,font=\footnotesize,>=stealth]
			\tkzDefPoints{-2/0/A,4/0/B,1.5/-2/C}
			\tkzDefMidPoint(A,C)\tkzGetPoint{M}
			\coordinate(N) at ($(B)!4/3!(M)$);
			\tkzDefLine[perpendicular=through N,K=0.9](A,B)\tkzGetPoint{S}
			\coordinate(K) at ($(B)!2/3!(S)$);
			\tkzDefMidPoint(C,B)\tkzGetPoint{L}
			\coordinate(H) at ($(L)!4/3!(M)$);
			\coordinate(E) at ($(S)!0.7!(H)$);
			\tkzMarkRightAngles[size=0.15](N,H,L B,K,C A,B,C S,N,B N,E,H)
			\tkzDrawPoints[fill=black](S,A,B,C,H,L,M,N,K,E)
			\tkzLabelPoints[left](A)
			\tkzLabelPoints[above right](K,H)
			\tkzLabelPoints[above](S)
			\tkzLabelPoints[right](E,B)
			\tkzLabelPoints[below](C,M,N,L)
			\tkzInterLL(S,N)(A,C) \tkzGetPoint{x}
			\tkzDrawSegments(S,A S,B S,C B,C C,K S,N M,N M,C A,x)
			\tkzDrawSegments[dashed](A,B x,M A,K B,N L,H N,H S,H N,E)
			
			\end{tikzpicture}}
		$\Rightarrow \mathrm{d}(AB,SM)=\mathrm{d}(AB,(SML))=\mathrm{d}(B,(SML))=3\mathrm{d}(N,(SML))=3NE$.
		\begin{itemize}
			\item $HN=\dfrac{1}{3}BI =\dfrac{1}{6}BC=\dfrac{a}{6}.$
			\item $BM=AM=\dfrac{AC}{2}=\dfrac{a}{\sqrt{2}}.$
			\item $MK$=$AM \cdot \cot 60^{\circ}=\dfrac{a}{\sqrt{2}} \cdot \dfrac{1}{\sqrt{3}}=\dfrac{a}{\sqrt{6}}.$
			\item $BN=\dfrac{4}{3} BM=\dfrac{2a\sqrt{2}}{3}.$
			\item  $\triangle MBK \backsim \triangle SBN$.\\
			$\Rightarrow \dfrac{MK}{SN}=\dfrac{BM}{SB}\Leftrightarrow 
			\dfrac{\frac{a}{\sqrt{6}}}{SN}=\dfrac{\frac{a}{\sqrt{2}}}{\sqrt{SN^2+\left(\frac{2a\sqrt{2}}{3}\right)}}
			\Rightarrow SN=\dfrac{2a}{3}$.
			\item $\dfrac{1}{NE^2}=\dfrac{1}{SN^2}+\dfrac{1}{NH^2}
			\Rightarrow NE=\dfrac{2a\sqrt{17}}{51}$.
		\end{itemize}
		Vậy $\mathrm{d}(AB,SM)=3NE=\dfrac{2a\sqrt{17}}{17}$.
	}
\end{ex}%!Cau!%
\begin{ex}%[Đề KSCL lớp 12 môn Toán Sở giáo dục và đào tạo Thanh Hóa, 2018-2019]%[Cao Thành Thái, dự án 12-EX-8-2019]%[1H3G5-3]
 Cho hình chóp $S.ABCD$ có đáy $ABCD$ là hình thoi cạnh $a$. Tam giác $ABC$ đều, hình chiếu vuông góc $H$ của đỉnh $S$ trên mặt phẳng $(ABCD)$ trùng với trọng tâm của tam giác $ABC$. Đường thẳng $SD$ hợp với mặt phẳng $(ABCD)$ góc $30^\circ$. Tính khoảng cách $d$ từ $B$ đến mặt phẳng $(SCD)$ theo $a$.
 \choice
  {$d=a\sqrt{3}$}
  {$d=\dfrac{2a\sqrt{21}}{21}$}
  {\True $d=\dfrac{a\sqrt{21}}{7}$}
  {$d=\dfrac{2a\sqrt{5}}{3}$}
 \loigiai
  {
  \immini
  {
  Vì $H$ là hình chiếu vuông góc của $S$ lên $(ABCD)$ nên $HD$ là hình chiếu vuông góc của $SD$ lên $(ABCD)$.\\
  Vậy góc tạo bởi $SD$ và $(ABCD)$ bằng góc giữa $SD$ và $HD$ chính là $\widehat{SDH}$. Khi đó $\widehat{SDH} = 30^\circ$.\\
  Gọi $O$ là giao điểm của $AC$ và $BD$.\\
  Vì $H$ là trọng tâm của tam giác $ABC$ nên $$BH= \dfrac{2}{3}BO = \dfrac{2}{3}\cdot \dfrac{a\sqrt{3}}{2} = \dfrac{a\sqrt{3}}{3}.$$
  }
  {
  \begin{tikzpicture}[line join=round, line cap=round, >=stealth,font=\footnotesize, scale=0.8]
   \tkzDefPoints{0/0/A, 5/0/D, -1.2/-2/B}
   \tkzDefPointBy[translation=from A to D](B)\tkzGetPoint{C}
   \tkzDefMidPoint(A,B)\tkzGetPoint{M}
   \tkzCentroid(A,B,C)\tkzGetPoint{H}
   \tkzDefShiftPoint[H](90:5){S}
   \tkzDefBarycentricPoint(C=3,S=2)\tkzGetPoint{K}
   \tkzDefMidPoint(A,C)\tkzGetPoint{O}
   \pgfresetboundingbox
   \tkzDrawPoints[fill=black](A,B,C,D,H,S,K,O)
   \tkzDrawSegments(S,B S,C S,D B,C C,D)
   \tkzDrawSegments[dashed](S,A A,B A,D A,C C,M S,H B,D H,K)
   \node[above] at (S){$S$};
   \node[left] at (A){$A$};
   \node[below left] at (B){$B$};
   \node[below right] at (C){$C$};
   \node[right] at (D){$D$};
   \node[below] at (H){$H$};
   \node[below] at (O){$O$};
   \node[right] at (K){$K$};
   \tkzMarkRightAngles(H,K,C S,H,M C,O,D)
  \end{tikzpicture}
  }
  \noindent
  Mặt khác $BH = \dfrac{2}{3}BO = \dfrac{2}{3} \cdot \dfrac{1}{2}BD = \dfrac{1}{3}BD$.\\
  Suy ra $DH = \dfrac{2}{3}BD = \dfrac{4}{3}BO = \dfrac{4}{3}\cdot \dfrac{a\sqrt{3}}{2} = \dfrac{2a\sqrt{3}}{3}$.\\
  Trong tam giác vuông $SHD$ ta có $SH = DH\tan \widehat{SDH} = \dfrac{2a\sqrt{3}}{3} \cdot \dfrac{\sqrt{3}}{3} = \dfrac{2a}{3}$.\\
  Ta có $HC \perp AB$, $AB \parallel CD$ nên $HC \perp CD$.\\
  Lại có $CD \perp SH$. Do đó $CD \perp (SHC)$, suy ra $(SHC) \perp (SCD)$.\\
  Kẻ $HK \perp SC$ tại $K$. Suy ra $HK \perp (SCD)$.\\
  Vậy $\mathrm{d}(H,(SCD)) = HK = \dfrac{HC \cdot SH}{\sqrt{HC^2 + SH^2}} = \dfrac{2a\sqrt{21}}{21}$.\\
  Ta lại có
  \allowdisplaybreaks
  \begin{eqnarray*}
   \dfrac{\mathrm{d}(B,(SCD))}{\mathrm{d}(H,(SCD))} = \dfrac{BD}{DH} = \dfrac{3}{2} \Rightarrow \mathrm{d}(B,(SCD)) = \dfrac{3}{2}\mathrm{d}(H,(SCD)) = \dfrac{3}{2} \cdot \dfrac{2a\sqrt{21}}{21} = \dfrac{a\sqrt{21}}{7}.
  \end{eqnarray*}
  }
\end{ex}%!Cau!%
\begin{ex}%[Đề thi thử L2, Liên trường Nghệ An, 2019]%[Nguyễn Đắc Giáp, dự án 12EX8]%[1H3G5-3]
	Cho hình chóp tam giác $S.ABC$ có đáy $ABC$ là tam giác đều cạnh $2a$ và $\widehat{SBA}=\widehat{SCA}=90^\circ$. Biết góc giữa đường thẳng $SA$ và mặt đáy bằng $45^\circ$. Tính khoảng cách từ điểm $B$ đến mặt phẳng $(SAC)$.
	\choice
	{$\dfrac{\sqrt{15}}{5}a$}
	{\True $\dfrac{2\sqrt{15}}{5}a$}
	{$\dfrac{2\sqrt{15}}{3}a$}
	{$\dfrac{2\sqrt{51}}{15}a$}
	\loigiai{
		\immini{
			Ta có $\mathrm{d}(B,(SAC))=\dfrac{3V_{SABC}}{S_{SAC}}$. \\
			Gọi $H$ là hình chiếu vuông góc của $S$ trên $(ABC)$.\\
			Ta có $\heva{&AB\perp BS\\&AB\perp SH}\Rightarrow AB\perp (SBH)\Rightarrow AB\perp BH.$\\
			Tương tự ta chứng minh được $AC\perp CH$.\\
			Do đó ta chứng minh được $\triangle ABH=\triangle ACH$.\\
			Suy ra $AH$ là đường phân giác trong góc $A$ của tam giác đều $ABC$ và\\ $AH=\dfrac{AB}{\cos 30^\circ}=\dfrac{4a\sqrt{3}}{3}.$\\
			Vì góc giữa đường thẳng $SA$ và mặt đáy bằng $45^\circ$ nên ta có $$\widehat{SAH}=45^\circ\Rightarrow \heva{&SH=AH=\dfrac{4a\sqrt{3}}{3}\\&AC=\dfrac{4a\sqrt{6}}{3}.}$$
			Tam giác $SAC$ vuông tại $C$ nên
			$SC=\sqrt{SA^2-AC^2}=\dfrac{2a\sqrt{15}}{3}.$\\
			Diện tích tam giác $SAC$ là $S_{\triangle SAC}=\dfrac{1}{2}AC\cdot SC=\dfrac{2a^2\sqrt{15}}{3}$. \\
			Thể tích khối chóp $S.ABC$ là 			$V_{SABC}=\dfrac{1}{3}SH\cdot S_{\triangle ABC}=\dfrac{4}{3}a^3$.\\
			Vậy $\mathrm{d}(B,(SAC))=\dfrac{3V_{SABC}}{S_{\triangle SAC}}=\dfrac{2\sqrt{15}}{5}a$.
			
			
		}
		{
			\begin{tikzpicture}[line join = round, line cap = round,>=stealth,font=\footnotesize,scale=1]
			\tkzDefPoints{0/0/A,1/-2/B,4/0/C}
			\tkzDefMidPoint(B,C)\tkzGetPoint{M}
			\coordinate (H) at ($(A)!1.33!(M)$);
			\coordinate (S) at ($(H)+(0,4.5)$);
			\tkzDrawPolygon(S,A,B,H)
			\tkzDrawSegments(S,B S,C B,H C,H)
			\tkzDrawSegments[dashed](B,C A,C A,H)
			\tkzDrawPoints[fill=black](S,A,B,C,H)
			\tkzLabelPoints[above](S)
			\tkzLabelPoints[below](B)
			\tkzLabelPoints[left](A)
			\tkzLabelPoints[right](C,H)
			\tkzMarkRightAngles(S,H,A A,B,S A,C,S)
			
			\end{tikzpicture}	
		}
	}
\end{ex}%!Cau!%
\begin{ex}%[Thi thử lần 1, THPT Văn Giang - Hưng Yên, 2019]%[Đỗ Đường Hiếu, 12EX-8-2019]%[1H3G5-4]
	Cho hình chóp $S.ABC$ có đáy là tam giác đều cạnh $a$, góc giữa $SC$ và $\mathrm{mp}(ABC)$ là $45^\circ$. Hình chiếu của $S$ lên $\mathrm{mp}(ABC)$ là điểm $H$ thuộc $AB$ sao cho $HA=2HB$. Tính khoảng cách giữa hai đường thẳng $SA$ và $BC$.
	\choice
	{$\dfrac{a\sqrt{210}}{45}$}
	{\True $\dfrac{a\sqrt{210}}{20}$}
	{$\dfrac{a\sqrt{210}}{15}$}
	{$\dfrac{a\sqrt{210}}{30}$}
	\loigiai{
		\immini{Dựng hình bình hành $ABCD$, khi đó $ABCD$ là hình thoi cạnh $a$ và $BC\parallel AD\Rightarrow BC\parallel (SAD)$.\\
			Do đó 
			$$\mathrm{d}\left(SA;BC\right) =\mathrm{d}\left[BC;(SAD)\right]=\mathrm{d}\left[B;(SAD)\right].$$
			Từ $\dfrac{BA}{HA}=\dfrac{3}{2}\Rightarrow \mathrm{d}\left[B;(SAD)\right]=\dfrac{3}{2}\mathrm{d}\left[H;(SAD)\right]$.\\
			Ta có $SH\perp (ABC)$ nên suy ra
			$$\widehat{\left(SC;(ABC)\right)}=\widehat{\left(SC;HC\right)}=\widehat{SCH}.$$
			Suy ra $\widehat{SCH}=45^\circ$.\\
		}
		{\begin{tikzpicture}[scale=0.8, font=\footnotesize, line join=round, line cap=round, >=stealth]
			\tikzset{label style/.style={font=\footnotesize}}
			\tkzDefPoints{0/0/A,-2.5/-1.5/B,2.5/-3/C}
			\coordinate (H) at ($(A)!2/3!(B)$);
			\coordinate (O) at ($(A)!1/2!(C)$);
			\coordinate (D) at ($(B)!2!(O)$);
			\coordinate (S) at ($(H)+(0,4.5)$);
			\coordinate (K) at ($(A)!-0.15!(D)$);
			\coordinate (I) at ($(S)!0.7!(K)$);
			\tkzDefMidPoint(B,C)\tkzGetPoint{M}
			\tkzDrawSegments(B,C S,B S,C C,D S,D)
			\tkzDrawSegments[dashed](A,B S,A S,H A,C H,C A,D H,K A,K S,K H,I)
			\tkzDrawPoints[fill=black,size=4](A,B,C,S,H,O,D,K,I)
			\tkzLabelPoints[above](S)
			\tkzLabelPoints[below](B,H,A,C)
			\tkzLabelPoints[right](I,D)
			\tkzLabelPoints[above right](K)
			\tkzMarkRightAngles[size=0.2](S,H,C H,K,A S,H,K S,I,H)
			\end{tikzpicture}}
		\noindent
		$HC^2=HB^2+BC^2-2HB\cdot BC\cdot\cos\widehat{HBC}=\left(\dfrac{a}{3}\right)^2+a^2-2\cdot \dfrac{a}{3}\cdot a\cos60^\circ=\dfrac{7a^2}{9}\Rightarrow HC=\dfrac{a\sqrt{7}}{3}$.\\
		Tam giác $SHC$ vuông tại $H$ và $\widehat{SCH}=45^\circ$ nên tam giác $SHC$ vuông cân tại $H$. Từ đó ta có $SH=HC=\dfrac{a\sqrt{7}}{3}$.\\
		Kẻ $HK\perp AD$ tại $K\Rightarrow \widehat{HAK}=60^\circ$. Do đó 
		$$HK=HA\cdot\sin \widehat{HAK}=\dfrac{2a}{3}\cdot\sin 60^\circ =\dfrac{a\sqrt{3}}{3}.$$
		Kẻ $HI\perp SK$ tại $K$, suy ra $HI\perp (SAD)\Rightarrow \mathrm{d}\left[H;(SAD)\right]=HI$.\\
		Xét tam giác $SHA$ vuông tại $H$, ta có
		\begin{eqnarray*}
			&&\dfrac{1}{HI^2}=\dfrac{1}{HK^2}+\dfrac{1}{SH^2}=\dfrac{1}{\left(\dfrac{a\sqrt{3}}{3}\right)^2}+\dfrac{1}{\left(\dfrac{a\sqrt{7}}{3}\right)^2}=\dfrac{30}{7a^2}\\
			&\Rightarrow & HI=\dfrac{a\sqrt{210}}{30}.
		\end{eqnarray*}
		Vậy $\mathrm{d}\left(SA;BC\right)=\dfrac{3}{2}\mathrm{d}\left[H;(SAD)\right]=\dfrac{3}{2}HI=\dfrac{3}{2}\cdot \dfrac{a\sqrt{210}}{30}=\dfrac{\sqrt{210}}{20}$.
		
	}
\end{ex}%!Cau!%
\begin{ex}%[Thi thử, Toán Học và Tuổi Trẻ (Đề số 3), 2019]%[Đặng Tân Hoài, 12-EX-6-2019]%[1H3G5-3]
	Cho hình lăng trụ $ABC.A'B'C'$ có đáy $ABC$ là tam giác cân tại $C$, $AB=2a$, $AA'=a$, góc giữa $BC'$ và $(ABB'A')$ là $60^{\circ}$. Gọi $N$ là trung điểm $AA'$ và $M$ là trung điểm $BB'$. Tính khoảng cách từ điểm $M$ đến mặt phẳng $(BC'N)$.	
	\choice
	{\True $ \dfrac{2a\sqrt{74}}{37} $}
	{$ \dfrac{a\sqrt{74}}{37} $}
	{$ \dfrac{2a\sqrt{37}}{37} $}
	{$ \dfrac{a\sqrt{37}}{37} $}
	\loigiai{
		\begin{center}
			\begin{tikzpicture}[scale=1, font=\footnotesize, line join = round, line cap = round, >=stealth]
			\tkzDefPoints{0/0/A,5/0/C,1/1/B}
			\coordinate (A') at ($(A)+(0.5,4)$);
			\coordinate (N) at ($(A)!0.5!(A')$);
			\tkzDefPointsBy[translation = from A to A'](B,C){B'}{C'}
			\coordinate (I) at ($(A')!0.5!(B')$);
			\coordinate (M) at ($(B)!0.5!(B')$);
			\tkzDrawPolygon(A,C,C',B',A')
			\tkzDrawSegments(A',C' N,C')
			\tkzDrawSegments[dashed](B,C B,B' A,B B,C' B,N)
			\tkzDrawPoints[fill=black](A,C,B,A',B',C',N,M,I)
			\tkzLabelPoints[above](B')
			\tkzLabelPoints[above left](I)
			\tkzLabelPoints[left](A',A,N)
			\tkzLabelPoints[right](C',C,M)
			\tkzLabelPoints[below](B)
			\end{tikzpicture}
		\end{center}
	Gọi $I$ là trung điểm của $A'B'$. Khi đó $IC' \perp (ABB'A')$, suy ra $\left(\widehat{BC',(ABB'A')}\right)=\widehat{IBC'}=60^{\circ}$.\\
	Ta có
			$IB=\sqrt{B"B^2+B'I^2}=a\sqrt{2}$, $BC'=\dfrac{IB}{\cos 60^{\circ}}=2a\sqrt{2}$,
		 $IC'=BC' \cdot \sin 60^{\circ}=a\sqrt{6}$,\\ $NB=\sqrt{AB^2+AN^2}=\dfrac{a\sqrt{17}}{2}$, $NC'=\sqrt{A'N'^2+A'C'^2}=\sqrt{A'N'^2+A'I^2+IC'^2}=\dfrac{a\sqrt{29}}{2}$.\\
			Kí hiệu $p=\dfrac{1}{2}(NB+BC'+NC')$ thì
		 $S_{\triangle BC'N}=\sqrt{p(p-NB)(p-BC')(p-NC')}=a^2\dfrac{\sqrt{111}}{4}$.\\
	 Diện tích tam giác $ABN$ là $S_{\triangle ABN}=\dfrac{a^2}{2}$.\\
	Vậy $\mathrm{d}\left(M,(BC'N)\right)=\mathrm{d}\left(A',(BC'N)\right)=\mathrm{d}\left(A,(BC'N)\right)=\dfrac{3 \cdot V_{ABC'N}}{S_{\triangle BC'N}}=\dfrac{S_{\triangle ABN} \cdot IC'}{S_{\triangle BC'N}}\dfrac{2a\sqrt{74}}{37}$.				
}
\end{ex}