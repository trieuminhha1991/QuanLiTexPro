%!Cau!%
\begin{ex}%[De tap huan, So GD&DT Dien Bien, 2019]%[Ngoc Diep, dự án EX5]%[1D2Y2-1]
	Có bao nhiêu cách chọn $5$ cầu thủ từ $11$ cầu thủ trong một đội bóng để thực hiện đá $5$ quả luân lưu $11$ m, theo thứ tự quả thứ nhất đến quả thứ năm.
	\choice
	{\True $\mathrm{A}_{11}^5$}
	{$\mathrm{C}_{11}^5$}
	{$\mathrm{A}_{12}^2 \cdot 5!$}
	{$\mathrm{C}_{10}^5$}
	\loigiai{
		Số cách chọn $5$ cầu thủ từ $11$ cầu thủ và có thứ tự là một chỉnh hợp chập $5$ của $11$ phần tử: $\mathrm{A}_{11}^5$.
	}
\end{ex}%!Cau!%
\begin{ex}%[DTH, Sở GD và ĐT - Hà Nam, 2019]%[Đào-V- Thủy, 12EX5]%[1D2Y2-1]
	Có bao nhiêu số có bốn chữ số khác nhau được tạo thành từ các chữ số $1$, $2$, $3$, $4$, $5$?
	\choice
	{$\mathrm{C}_5^4$}
	{$\mathrm{P}_4$}
	{\True $\mathrm{A}_5^4$}
	{$\mathrm{P}_5$}
	\loigiai{
		Mỗi số cần tìm là một chỉnh hợp chập $4$ của $5$ phần tử. Do đó có $\mathrm{A}_5^4$ số thỏa mãn đề bài.
	}
\end{ex}%!Cau!%
\begin{ex}%[Thi thử lần I, Sở GD&ĐT Sơn La 2019]%[Nguyễn Anh Quốc,  dự án EX5]%[1D2Y2-1]
Cho tập hợp $S$ có $20$ phần tử. Số tập con gồm $3$ phần tử của $S$ là	
	\choice
	{$\mathrm{A}^3_{20}$}
	{$\mathrm{C}^3_{20}$}
	{$20^3$}
	{$\mathrm{A}^{17}_{20}$}
	\loigiai{
	Số tập con gồm $3$ phần tử của tập $20$ phần tử là một tổ hợp chập $3$ của $20$ phần tử. Vậy ta có $\mathrm{C}^3_{20}$ tập con.	
	}
\end{ex}%!Cau!%
\begin{ex}%[Thi thử, Sở GD và ĐT -Lạng Sơn, 2019]%[Trần Duy Khương, 12EX5-2019]%[1D2Y2-1]
		Cho tập hợp $A$ có $10$ phần tử. Số tập con gồm $3$ phần tử của $A$ là
		\choice
		{$\mathrm{A}_{10}^7$}
		{$\mathrm{A}_{10}^3$}
		{\True $\mathrm{C}_{10}^3$}
		{$10^3$}
		\loigiai{Số tập con gồm $3$ phần tử của $A$ bằng $\mathrm{C}_{10}^3$.}
	\end{ex}%!Cau!%
\begin{ex}%[Đề tập huấn, Sở GD - ĐT tỉnh Quảng Bình, 2019]%[Nguyễn Tiến, dự án 12EX5]%[1D2Y2-1]
	Cho $k$, $n$ $(k<n)$ là các số nguyên dương. Mệnh đề nào sau đây \textbf{sai}?
	\choice
	{$\mathrm{C}_n^k=\dfrac{n!}{k!\cdot(n-k)!}$}
	{\True $\mathrm{A}_n^k=n!\cdot\mathrm{C}_n^k$}
	{$\mathrm{A}_n^k=k!\cdot\mathrm{C}_n^k$}
	{$\mathrm{C}_n^k=\mathrm{C}_n^{n-k}$}
	\loigiai{
		Công thức chỉnh hợp $\mathrm{A}_n^k=\dfrac{n!}{(n-k)!}$.\\
		Công thức tổ hợp $\mathrm{C}_n^k=\dfrac{n!}{k!\cdot(n-k)!}$.\\
		Ta có $\mathrm{A}_n^k=k!\cdot\mathrm{C}_n^k$ nên đáp án $\mathrm{A}_n^k=n!\cdot\mathrm{C}_n^k$ sai.
	}
\end{ex}%!Cau!%
\begin{ex}%[Đề tập huấn, Sở GD và ĐT - Vĩnh Phúc, 2019]%[Mai Sương, EX-5-2019]%[1D2Y2-1]
Cho tập hợp $M$ có $10$ phần tử. Số tập con gồm hai phần tử của $M$ là
	\choice
	{$\mathrm{A}_{10}^8$}
	{$\mathrm{A}_{10}^2$}
	{\True $\mathrm{C}_{10}^2$}
	{$10^2$}
	\loigiai{
	Mỗi cách lấy ra $2$ phần tử trong $10$ phần tử của $M$ để tạo thành tập con gồm $2$ phần tử là một tổ hợp chập $2$ của $10$ phần tử. Vậy số tập con của $M$ gồm $2$ phần tử là $\mathrm{C}_{10}^2$.	
	}
	
	\end{ex}%!Cau!%
\begin{ex}%[Tập huấn, Sở GD và ĐT lần 1, 2019]%[Lê Xuân Hòa, 12EX5]% [1D2Y2-1]
Có bao nhiêu cách chọn $6$ học sinh từ nhóm gồm $12$ học sinh?
\choice
{$\mathrm{A}_{12}^6$}
{\True $\mathrm{C}_{12}^6$}
{$6^{12}$}
{$12^6$}
\loigiai{
Số cách chọn $6$ học sinh từ nhóm có $12$ học sinh là $\mathrm{C}_{12}^6$.}
\end{ex}%!Cau!%
\begin{ex}%[Hải Phòng, 2018]%[Phan Anh Tiến, 12-EX-05]%[1D2Y2-1]
	Tính số chỉnh hợp chập $4$ của $7$ phần tử.
	\choice
	{$ 24 $}
	{$ 720 $}
	{\True $ 840 $}
	{$ 35 $}
	\loigiai{Ta có $\mathrm{A}_7^4=840$.
		}
\end{ex}%!Cau!%
\begin{ex}%[Thi thử, Chuyên Sơn La, 2018]%[Nguyễn Thanh Tâm, 12-EX-5-2019]%[1D2Y2-1]
	Cho tập hợp $S$ có $20$ phần tử. Số tập con gồm $3$ phần tử của $S$ là
	\choice
	{$\mathrm{A}_{20}^3$}
	{$\mathrm{A}_{20}^7$}
	{\True $\mathrm{C}_{20}^3$}
	{$20^3$}
	\loigiai{
		Mỗi tập con gồm $3$ phần tử là tổ hợp chập $3$ của $20$ phần tử.\\
		Vậy số tập con gồm $3$ phần tử là $\mathrm{C}_{20}^3$. 
	}
\end{ex}%!Cau!%
\begin{ex}%[Thi tập huấn, Sở GD và ĐT - Bắc Ninh, 2019]%[Nguyễn Minh Tiến, 12EX5]%[1D2Y2-1]
	Có bao nhiêu cách chọn $6$ học sinh từ nhóm $12$ học sinh?
	\choice
	{$\mathrm{A}_{12}^{6}$}
	{\True $\mathrm{C}_{12}^{6}$}
	{$6^{12}$}
	{$12^6$}
	\loigiai{
		Mỗi cách chọn $6$ học sinh từ nhóm gồm $12$ học sinh là một tổ hợp chập $6$ của $12$, do đó có $\mathrm{C}_{12}^{6}$ cách chọn.
	}
\end{ex}%!Cau!%
\begin{ex}%[Đề Tập Huấn -4, Sở GD và ĐT - Hải Phòng, 2019]%[Trần Xuân Thiện, 12EX5]%[1D2Y2-1]
	Tính số chỉnh hợp chập $4$ của $7$ phần tử?
	\choice
	{$ 24 $}
	{$ 720 $}
	{\True $ 840 $}
	{$ 35 $}
	\loigiai{
		Số chỉnh hợp chập $4$ của $7$ phần tử là $\mathrm{A}_7^4 = \dfrac{7!}{3!}=840$.
	}
\end{ex}%!Cau!%
\begin{ex}%[Đề tập huấn, Bắc Kạn, 2018-2019]%[Cao Thành Thái, 12EX5-2019]%[1D2Y2-1]
 Cho tập hợp gồm $n$ phần tử. Số các chỉnh hợp chập $k$ của $n$ phần tử là
 \choice
  {\True $\mathrm{A}_n^k$}
  {$\mathrm{C}_n^k$}
  {$n\mathrm{A}_n^k$}
  {$n\mathrm{C}_n^k$}
 \loigiai
  {
  Số các chỉnh hợp chập $k$ của $n$ phần tử là $\mathrm{A}_n^k$.
  }
\end{ex}%!Cau!%
\begin{ex}%[Tập huấn SGD Bắc Ninh, Dự án 12EX5, 2019, Chu Đức Minh]%[1D2Y2-1]
	Có bao nhiêu cách chọn $6$ học sinh từ nhóm gồm $12$ học sinh? 
	\choice
	{$\mathrm{A}^6_{12}$}
	{\True $\mathrm{C}^6_{12}$}
	{$6^{12}$}
	{$12^6$}
	\loigiai{Số cách chọn $6$ phần tử từ $12$ phần tử, không có tính thứ tự là $\mathrm{C}^6_{12}$. }
\end{ex}%!Cau!%
\begin{ex}%[Thi thử, Sở GD và ĐT - Bình Phước lần 1, 2019]%[Lê Thanh Nin, 12EX7]%[1D2Y2-1]
	Với $k$ và $n$ là hai số nguyên dương tùy ý thỏa mãn $k\le n$, mệnh đề nào sau đây đúng?
	\choice	
	{\True $\mathrm{A}_n^k=\dfrac{n!}{(n-k)!}$}
	{$\mathrm{A}_n^k=\dfrac{n!}{k!(n-k)!}$}
	{$\mathrm{A}_n^k=\dfrac{n!}{k!}$}
	{$\mathrm{A}_n^k=\dfrac{(n-k)!}{n!}$}
	\loigiai{
		Theo lý thuyết chỉnh hợp ta có $\mathrm{A}_n^k=\dfrac{n!}{(n-k)!}$.
	}
\end{ex}%!Cau!%
\begin{ex}%[Đề thi thử Chuyên Sơn La lần 1, Sơn La, 2018-2019]%[Cao Thành Thái, dự án 12-EX-7-2019]%[1D2Y2-1]
 Với $k$ và $n$ là hai số nguyên dương tùy ý thỏa mãn $k\leq n$, mệnh đề nào dưới đây đúng?
 \choice
  {\True $\mathrm{A}_n^k=\dfrac{n!}{(n-k)!}$}
  {$\mathrm{A}_n^k=\dfrac{n!}{k!(n-k)!}$}
  {$\mathrm{A}_n^k=\dfrac{n!}{k!}$}
  {$\mathrm{A}_n^k=\dfrac{k!(n-k)!}{n!}$}
 \loigiai
  {
  Với $k$ và $n$ là hai số nguyên dương tùy ý thỏa mãn $k\leq n$ thì $\mathrm{A}_n^k=\dfrac{n!}{(n-k)!}$.
  }
\end{ex}%!Cau!%
\begin{ex}%[Đề thi thử THPTQG lần 2 THPT Thoại Ngọc Hầu, An Giang, năm 2019]%[Nguyễn Thành Khang, dự án 2019-Ex-7]%[1D2Y2-1]
	Từ các chữ số $1$, $3$, $5$, $7$, $9$ có thể lập được bao nhiêu số tự nhiên có $5$ chữ số khác nhau?
	\choice
	{$3215$}
	{$3125$}
	{$25$}
	{\True $120$}
	\loigiai{
		Mỗi số tự nhiên có $5$ chữ số khác nhau tương ứng với một hoán vị của $5$ chữ số đã cho. Vậy có thể lập được $\mathrm{P}_5=120$ số thỏa mãn.
	}
\end{ex}%!Cau!%
\begin{ex}%[Thi Thử Lần 1, THPT Chuyên Lê Khiết - Quảng Ngãi, 2019]%[Dương BùiĐức, dự án 12EX7]%[1D2Y2-1] 
	Với $k$ và $n$ là hai số nguyên dương tùy ý thỏa mãn $k\le n$, mệnh đề nào dưới đây đúng?
	\choice
	{$\mathrm{A}_{n}^{k}=\dfrac{n !}{k !(n-k) !}$}
	{$\mathrm{A}_{n}^{k}=\dfrac{k !}{(n-k) !}$}
	{\True $\mathrm{A}_{n}^{k}=\dfrac{n !}{(n-k) !}$}
	{$\mathrm{A}_{n}^{k}=\dfrac{(n-k) !}{n !}$}
	\loigiai
	{
		Ta có $\mathrm{A}_{n}^{k}=\dfrac{n !}{(n-k) !}$.
	}
\end{ex}%!Cau!%
\begin{ex}%[Thi thử L1, THPT Ngô Quyền - Hà Nội, 2019]%[Phan Ngọc Toàn, dự án EX7]%[1D2Y2-1]
	Cho $A=\{1;2;3;4\}$. Từ $A$ lập được bao nhiêu số tự nhiên có $4$ chữ số đôi một khác nhau?  
	\choice
	{$32$}
	{\True $24$}
	{$256$}
	{$18$}
	\loigiai{
		Mỗi số tự nhiên có $4$ chữ số đôi một khác nhau được lập từ tập $A$ là một hoán vị của $4$ phần tử của $A$. Vậy có $\mathrm{P}_4=4!=24$ số tự nhiên lập được.
	}
\end{ex}%!Cau!%
\begin{ex}%[Thi thử L1, THPT Hậu Lộc 2, Thanh Hoá, 2019]%[Dương Phước Sang, 12EX-5-2019]%[1D2Y2-1]
	Với $k,n$ là hai số nguyên dương tùy ý thỏa mãn $k\leqslant n$, mệnh đề nào dưới đây đúng?
	\choice
	{\True $\mathrm{A}_n^k=\dfrac{n!}{(n-k)!}$}
	{$\mathrm{A}_n^k=\dfrac{n!}{k!(n-k)!}$}
	{$\mathrm{A}_n^k=\dfrac{n!}{k!}$}
	{$\mathrm{A}_n^k=\dfrac{(n-k)!}{n!}$}
	\loigiai{
		Trong $4$ công thức trên, công thức đúng là $\mathrm{A}_n^k=\dfrac{n!}{(n-k)!}$.}
\end{ex}%!Cau!%
\begin{ex}%[Thi thử, trường THPT Nguyễn Công Trứ, tỉnh Hà Tĩnh, 2019]%[Phạm Doãn Lê Bình, 12EX7-19]%[1D2Y2-1]
	Với $k$ và $n$ là hai số nguyên dương tùy ý thỏa mãn $k\le n$, mệnh đề nào dưới đây là đúng?
	\choice
	{$\mathrm{A}_n^k = \dfrac{n!}{k!(n-k)!}$}
	{\True $\mathrm{A}_n^k = \dfrac{n!}{(n-k)!}$}
	{$\mathrm{A}_n^k = \dfrac{n!}{k!}$}
	{$\mathrm{A}_n^k = \dfrac{k!}{n! (n-k)!}$}
	\loigiai{
	Theo lý thuyết, ta có công thức $\mathrm{A}_n^k = \dfrac{n!}{(n-k)!}$.
	}
\end{ex}%!Cau!%
\begin{ex}%[Thi thử, Chuyên Thái Nguyên-Thái Nguyên-Lần 2, 2019]%[Duong Xuan Loi, 12-EX-7-19]%[1D2Y2-1]	
	Cho trước $5$ chiếc ghế xếp thành một hàng ngang. Số cách xếp ba bạn $A$, $B$, $C$ vào $5$ chiếc ghế đó sao cho mỗi bạn ngồi một ghế là
	\choice
	{$\mathrm{C}_5^3$}
	{\True $\mathrm{A}_5^3$}
	{$15$}
	{$6$}
	\loigiai{
		Số cách xếp ba bạn $A$, $B$, $C$ vào $5$ chiếc ghế đó sao cho mỗi bạn ngồi một ghế là $\mathrm{A}_5^3$.
	}
\end{ex}%!Cau!%
\begin{ex}%[TT, Hội 8 trường chuyên - Khu vực đông bằng sông Hồng, 2019-L2]%[ Nguyễn Quang Dũng, dự án 12-EX-7-2019]%[1D2Y2-1]
Cho tập hợp $S$ gồm có $5$ phần tử. Số tập con gồm $2$ phần tử của $S$ là
\choice
{$30$}
{$5^2$}
{\True $\mathrm{C}_5^2$}
{$\mathrm{A}_5^2$}
\loigiai{
 Số tập con có hai phần tử của $S$ chính bằng số tổ hợp chập $2$ của $5$ phần tử $\Rightarrow$ có $\mathrm{C}_5^2$ tập.}
\end{ex}%!Cau!%
\begin{ex}%[Thi thử L1, Chuyên Lê Quý Đôn,Lai Châu, 2019]%[Nguyễn Tài Tuệ, dự án EX7]%[1D2Y2-1]
	Một học sinh trong thời gian học thi, muốn sắp xếp $7$ ngày trong tuần cho $7$ môn học (mỗi ngày một môn). Số cách sắp xếp là
	\choice
	{$7$}
	{$49$}
	{\True $7!$}
	{$7\cdot 7!$}
	\loigiai
	{
		Mỗi cách xếp $7$ môn học cho $7$ ngày trên tuần là một hoán vị của $7$ phần tử.\\
		Số cách xếp là $7!$.
	}
\end{ex}%!Cau!%
\begin{ex}%[Đề KSCL lớp 12 môn Toán Sở giáo dục và đào tạo Thanh Hóa, 2018-2019]%[Cao Thành Thái, dự án 12-EX-8-2019]%[1D2Y2-1]
 Cho tập hợp $A$ gồm có $9$ phần tử. Số tập con gồm có $4$ phần tử của tập hợp $A$ là
 \choice
  {$\mathrm{P}_4$}
  {\True $\mathrm{C}_9^4$}
  {$4\times 9$}
  {$\mathrm{A}_9^4$}
 \loigiai
  {
  Mỗi tập con gồm $4$ phần tử của tập hợp $A$ là một tổ hợp chập $4$ của $9$ phần tử. Do đó, số tập con gồm có $4$ phần tử của tập hợp $A$ là $\mathrm{C}_9^4$.
  }
\end{ex}%!Cau!%
\begin{ex}%[Thi thử, Sở GD và ĐT - Điện Biên, 2019]%[Tô Ngọc Thy, dự án EX8]%[1D2Y2-6]
	Cho $n$ và $k$ là hai số nguyên dương tùy ý thỏa mãn $k\leq n$. Mệnh đề nào dưới đây đúng?
	\choice
	{$\mathrm{A}_n^k=\dfrac{n!}{k!(n-k)!}$}
	{\True $\mathrm{C}_{n-1}^{k-1}+\mathrm{C}_{n-1}^k=\mathrm{C}_n^k$}
	{$\mathrm{C}_n^{k-1}=\mathrm{C}_n^k$}
	{$\mathrm{C}_n^k=\dfrac{n!}{(n-k)!}$}
	\loigiai{
		Ta có
		\begin{itemize}
			\item $\mathrm{A}_n^k=\dfrac{n!}{\left(n-k\right)!}$ nên đáp án $\mathrm{A}_n^k=\dfrac{n!}{k!(n-k)!}$ loại.
			\item Với $n=9$; $k=7$ ta có $\mathrm{C}_n^{k-1}=\mathrm{C}_9^6=84\neq \mathrm{C}_9^7=36$ nên đáp án $\mathrm{C}_n^{k-1}=\mathrm{C}_n^k$, $(1\leq k\leq n)$ loại.
			\item $\mathrm{C}_n^k=\dfrac{n!}{k!\left(n-k\right)!}$ nên đáp án $\mathrm{C}_n^k=\dfrac{n!}{(n-k)!}$ loại.
		\end{itemize}
		Vậy đáp án $\mathrm{C}_{n-1}^{k-1}+\mathrm{C}_{n-1}^k=\mathrm{C}_n^k$, $(1\leq k\leq n)$ là đáp án đúng.}
\end{ex}%!Cau!%
\begin{ex}%[Thi thử, Sở GD và ĐT - Hà Tĩnh, 2019]%[Nguyễn Anh Tuấn, 12-EX8-19]%[1D2Y2-1]
	Cho tập hợp $ X $ có $ n $ phần tử ($n \in \mathbb{N}^* $), số hoán vị $ n $ phần tử của tập hợp $ X $ là
	\choice
	{$ n $}
	{$ n^2 $}
	{$ n^3 $}
	{\True $ n! $}
	\loigiai{
		Theo tính chất về hoán vị:  Số hoán vị $ n $ phần tử của tập hợp $ X $ là $ n! $.	
	}
\end{ex}%!Cau!%
\begin{ex}%[Thi thử, Sở GD và ĐT - Lào Cai, 2019]%[Lê Thanh Nin, 12EX8]%[1D2Y2-1]
	Công thức tính số tổ hợp chập $k$ của $n$ phần tử là 
	\choice
	{\True$\mathrm{C}_n^k=\dfrac{n!}{(n-k)!k!}$}
	{$\mathrm{A}_n^k=\dfrac{n!}{(n-k)!}$}
	{$\mathrm{C}_n^k=\dfrac{n!}{(n-k)!}$}
	{$\mathrm{A}_n^k=\dfrac{n!}{(n-k)!k!}$}
	\loigiai{Công thức tính số tổ hợp chập $k$ của $n$ phần tử là  $\mathrm{C}_n^k=\dfrac{n!}{(n-k)!k!}$.}
\end{ex}%!Cau!%
\begin{ex}%[TT, THPT Chuyên Hà Tĩnh, 19]%[Trần Bá Huy, 12-EX-8-2019]%[1D2Y2-2]
Với $k,n$ là hai số nguyên dương tùy ý thỏa mãn $k\leq n$, mệnh đề nào dưới đây \textbf{sai?}
\choice
{$\mathrm{C}_n^k=\dfrac{n!}{k!(n-k)!}$}
{$\mathrm{A}_n^k=k!\mathrm{C}_n^k$}
{$\mathrm{C}_n^k+\mathrm{C}_n^{k-1}=\mathrm{C}_{n+1}^k$}
{\True $\mathrm{C}_n^k=k!\mathrm{A}_n^k$}
\loigiai{
Mệnh đề sai là $\mathrm{C}_n^k=k!\mathrm{A}_n^k$.
}
\end{ex}%!Cau!%
\begin{ex}%[Đề thi thử L2, Liên trường Nghệ An, 2019]%[Nguyễn Đắc Giáp, dự án 12EX8]%[1D2Y2-1]
	$\mathrm{C}_n^2$ bằng biểu thức nào sau đây?
	\choice
	{$\dfrac{n(n-1)}{3}$}
	{\True $\dfrac{n(n-1)}{2}$}
	{$\dfrac{n(n-1)}{6}$}
	{$n(n-1)$}
	\loigiai{
		Ta có $\mathrm{C}_n^2=\dfrac{n!}{2!(n-2)!}=\dfrac{n(n-1)(n-2)!}{2!(n-2)!}=\dfrac{n(n-1)}{2}.$\\
		Vậy $\mathrm{C}_n^2=\dfrac{n(n-1)}{2}.$
	}
\end{ex}%!Cau!%
\begin{ex}%[Dự án EX-8 2019]%[Phạm Tuấn]%[1D2Y2-1]
Với $k$ và $n$ là hai số nguyên dương tùy ý thỏa mãn $k \leqslant  n$ , mệnh đề nào dưới đây \textbf{sai}?
\choice
{$\mathrm{P}_n =n!$}
{\True $\mathrm{A}_n^k = \dfrac{n!}{k!(n-k)!}$}
{$\mathrm{C}_n^k=\mathrm{C}_n^{n-k}$}
{$\mathrm{C}_n^k = \dfrac{n!}{k!(n-k)!}$}
\loigiai{
$\mathrm{A}_n^k = \dfrac{n!}{k!(n-k)!}$ là mệnh đề sai.
}
\end{ex}%!Cau!%
\begin{ex}%[Thi thử L2, Chuyên Lê Quý Đôn - Đà Nẵng, 2019]%[Đinh Thanh Hoàng, 12-EX-8-2019]%[1D2Y2-1]
	Với $k$ và $n$ là hai số nguyên dương tùy ý thỏa mãn $k\leq n$, mệnh đề nào dưới đây đúng?
	\choice
	{$\mathrm{A}_n^k=\dfrac{n!}{k!}$}
	{\True $\mathrm{A}_n^k=\dfrac{n!}{(n-k)!}$}
	{$\mathrm{A}_n^k=\dfrac{(n-k)!}{k!}$}
	{$\mathrm{A}_n^k=n\cdots(n-k)$}
	\loigiai{
		Ta có công thức số các chỉnh hợp chập $k$ của $n$ phần tử là $\mathrm{A}_n^k=\dfrac{n!}{(n-k)!}$.
	}
\end{ex}%!Cau!%
\begin{ex}%[Thi thử  L2, trường Phúc Trạch-Hà Tĩnh, 2019]%[Lê Hồng Phi, 12EX8]%[1D2Y2-1]
Kí hiệu $\mathrm{C}_n^k$ (với $k$, $n$ là những số nguyên dương và $k\leq n$) có ý nghĩa là
\choice
{Chỉnh hợp chập $k$ của $n$ phần tử}
{\True Số tổ hợp chập $k$ của $n$ phần tử}
{Tổ hợp chập $k$ của $n$ phần tử}
{Số chỉnh hợp chập $k$ của $n$ phần tử}
\loigiai{
Ký hiệu	 $\mathrm{C}_n^k$ (với $k$, $n$ là những số nguyên dương và $k\leq n$) có ý nghĩa là số tổ hợp chập $k$ của $n$ phần tử. }
\end{ex}%!Cau!%
\begin{ex}%[Thi thử L1, Chuyên Nguyễn Trãi, Hải Dương, 2019]%[Đinh Thanh Hoàng, dự án EX6]%[1D2Y2-1]
	Cho $n\in \mathbb{N}$ và $n!=1$. Số giá trị của $n$ thỏa mãn giả thiết đã cho là
	\choice
	{$1$}
	{\True $2$}
	{$0$}
	{Vô số}
	\loigiai{
		Ta có $0!=1$ và $1!=1$. Vậy có $2$ giá trị của $n$ thỏa mãn.
	}
\end{ex}%!Cau!%
\begin{ex}%[Thi thử,Quảng Xương 1-Thanh Hóa-L3, 2019]%[Lê Quốc Hiệp, 12EX8-2019]%[1D2Y2-1]
	Từ các chữ số $1,~2,~3,~4,~5,~6,~7,~8$ có thể lập được bao nhiêu số tự nhiên gồm ba chữ số khác nhau?
	\choice
	{$3^8$}
	{$\mathrm{C}_8^3$}
	{\True $\mathrm{A}_8^3$}
	{$8^3$}
	\loigiai
	{
		Chọn ba chữ số khác nhau trong $8$ số đã cho và xếp thứ tự có $\mathrm{A}_8^3$ cách.\\
		Vậy có $\mathrm{A}_8^3$ số thỏa mãn đề bài.
	}
\end{ex}%!Cau!%
\begin{ex}%[Thi thử L2, Sở GD&DT Phú Thọ, 2019]%[Trần Nhân Kiệt, dự án EX9]%[1D2Y2-1]
	Có bao nhiêu cách xếp chỗ ngồi cho bốn bạn học sinh vào bốn chiếc ghế kê thành một hàng ngang?
	\choice
	{\True $24$}
	{$4$}
	{$12$}
	{$8$}
	\loigiai{
		Số cách xếp chỗ ngồi cho bốn bạn học sinh vào bốn chiếc ghế kê thành một hàng ngang là 
		$$\mathrm{P}_4=4!=24.$$
	}
\end{ex}%!Cau!%
\begin{ex}%[Thi Thủ SGD Bắc Ninh]%[Phan Anh - EX9]%[1D2Y2-1]
	Kí hiệu $\mathrm{C}_n^k$ là số các tổ hợp chập $k$ của $n$ phần tử, với $1\le k\le n$. Mệnh đề nào sau đây đúng?
	\choice
	{\True $\mathrm{C}_n^k=\dfrac{n!}{k!(n-k)!}$}
	{$\mathrm{C}_n^k=\dfrac{k!}{(n-k)!}$}
	{$\mathrm{C}_n^k=\dfrac{n!}{n!(n-k)!}$}
	{$\mathrm{C}_n^k=\dfrac{n!}{k!(n-k)!}$}
	\loigiai{Theo công thức tổ hợp ta có $\mathrm{C}_n^k=\dfrac{n!}{k!(n-k)!}$.}
\end{ex}%!Cau!%
\begin{ex}%[Đề thi thử Sở GD-ĐT Quảng Bình - 2019]%[Dự án 12-EX9-2019, Nguyễn Anh Quốc]%[1D2Y2-2]
Với $k$ và $n$ là các số nguyên dương tùy ý thỏa mãn $k\le n$, mệnh đề nào dưới đây \textbf{sai}?	
	\choice
	{$\mathrm{C}^k_n=\mathrm{C}^{n-k}_n$}
	{$\mathrm{C}^k_n=\dfrac{\mathrm{A}^k_n}{k!}$}
	{$\mathrm{C}^{k-1}_n+\mathrm{C}^k_n=\mathrm{C}^k_{n+1}$}
	{\True $\mathrm{C}^k_n=\mathrm{C}^n_k$}
	\loigiai{
	Mệnh đề sai là $\mathrm{C}^k_n=\mathrm{C}^n_k$ khi $k<n$.
	}
\end{ex}%!Cau!%
\begin{ex}%[Thi thử, Sở GD và ĐT - Đà Nẵng, 2019]%[Nguyễn Anh Tuấn, 12-EX9-19]%[1D2Y2-1]
	Trong mặt phẳng cho $ 18 $ điểm phân biệt trong đó không có ba điểm nào thẳng hàng. Số tam giác có các đỉnh thuộc $ 18 $ điểm đã cho là
	\choice
	{\True $ \mathrm{C}_{18}^3 $}
	{$ 6 $}
	{$ \mathrm{A}_{18}^3 $}
	{$ \dfrac{18!}{3} $}
	\loigiai{
	Mỗi cách lấy ba điểm trong $ 18 $ điểm không thẳng hàng cho ta một tam giác. Vậy số tam giác có các đỉnh thuộc $ 18 $ điểm đã cho là $ \mathrm{C}_{18}^3 $.}
\end{ex}%!Cau!%
\begin{ex}%[Đề thi thử THPT Chuyên Bến Tre, năm học 2018-2019]%[Tuan Nguyen, dự án 12-EX-9-2019]%[1D2Y2-1]
	Số các hoán vị của một tập hợp có $6$ phần tử là
	\choice{$6$}{$120$}{$46656$}{\True $720$}
	\loigiai{
Số các hoán vị là $6!=720$.	
}
\end{ex}%!Cau!%
\begin{ex}%[Đề-thi-thử-THPT-Quốc-gia-2019-môn-Toán-hội-các-trường-chuyên-lần-3]%[Nguyễn Thành Nhân,12EX8]%[1D2Y2-1]
Với $k,n$ là hai số tự nhiên tùy ý thỏa mãn $k\le n$, mệnh đề nào dưới đây đúng?
	\choice
	{\True $\mathrm{A}_n^k=\dfrac{n!}{(n-k)!}$}
	{  $\mathrm{A}_n^k=\dfrac{n!}{k!}$}
	{ $\mathrm{A}_n^k=\dfrac{n!}{k! \cdot (n-k)!}$}
	{ $\mathrm{A}_n^k=\dfrac{k!\cdot (n-k)!}{n!}$}
	\loigiai{ Áp dụng công thức tính số chỉnh hợp chập $k$ của $n$ phần tử, ta có $\mathrm{A}_n^k=\dfrac{n!}{(n-k)!}$. }
\end{ex}%!Cau!%
\begin{ex}%[12-EX-ĐHVinh-L3]%[Ngô Quang Anh]%[1D2Y2-1]
	Cho trước $5$ chiếc ghế xếp thành một hàng ngang. Số cách xếp ba bạn $A$, $B$, $C$ vào $5$ chiếc ghế đó sao cho mỗi bạn ngồi một ghế là
	\choice
	{$\mathrm{C}_5^3$}
	{\True $\mathrm{A}_5^3$}
	{$15$}
	{$6$}
	\loigiai{
		Số cách xếp ba bạn $A$, $B$, $C$ vào $5$ chiếc ghế đó sao cho mỗi bạn ngồi một ghế là $\mathrm{A}_5^3$.
	}
\end{ex}%!Cau!%
\begin{ex}%[12-EX-ĐHVinh-L3]%[Ngô Quang Anh]%[1D2Y2-1]
	Một nhóm học sinh có $10$ người. Cần chọn $3$ học sinh trong nhóm để làm $3$ công việc là tưới cây, lau bàn và nhặt rác, mỗi người làm một công việc. Số cách chọn là
	\choice
	{$\mathrm{C}_{10}^3$}
	{$10^3$}
	{$3\times 10$}
	{\True $\mathrm{A}_{10}^3$}
	\loigiai{
		Số cách chọn $3$ học sinh trong $10$ học sinh: $\mathrm{C}_{10}^3$ cách. \\
		Số cách xếp công việc cho $3$ học sinh: $3!$ cách. \\
		Vậy số cách chọn là: $\mathrm{C}_{10}^3 \cdot 3!=\mathrm{A}_{10}^3$.
	}
\end{ex}%!Cau!%
\begin{ex}%[Dự án 12-EX-8-2019, Nguyễn Anh Quốc]%[1D2Y2-1]
	Với $k$ và $n$ là hai số nguyên dương tùy ý thỏa mãn $k\le n$, mệnh đề nào dưới đây \textbf{sai}?
	\choice
	{$\mathrm{C}_n^k=\dfrac{n!}{k!(n-k)!}$}
	{$\mathrm{A}_n^k=\dfrac{n!}{(n-k)!}$}
	{$\mathrm{P}_n=n!$}
	{\True $\mathrm{C}_n^k=\dfrac{k!(n-k)!}{n!}$}
	\loigiai{
		Mệnh đề sai là $$\mathrm{C}_n^k=\dfrac{k!(n-k)!}{n!}.$$ 
	}
\end{ex}%!Cau!%
\begin{ex}%[Đề dự đoán số 2]%[Nguyễn Thành Khang, dự án 2019-12-Ex-8]%[1D2Y2-1]
	Với $k$ và $n$ là hai số nguyên dương tùy ý thỏa mãn $k\le n$, mệnh đề nào dưới đây là đúng?
	\choice
	{$\mathrm{A}_n^k=\dfrac{n!}{k!(n-k)!}$}
	{$\mathrm{A}_n^k=\dfrac{k!(n-k)!}{n!}$}
	{\True $\mathrm{A}_n^k=\dfrac{n!}{(n-k)!}$}
	{$\mathrm{A}_n^k=\dfrac{n!}{k!}$}
	\loigiai{
		Công thức tính chỉnh hợp chập $k$ của $n$ phần tử là $\mathrm{A}_n^k=\dfrac{n!}{(n-k)!}$.
	}
\end{ex}%!Cau!%
\begin{ex}%[Phát triển đề minh họa 2019]%[Ex 8 - 2019,Dũng Lê]%[1D2Y2-1]
	Cho tập hợp gồm $n$ phần tử. Số các chỉnh hợp chập $k$ của $n$ phần tử là
	\choice
	{\True $\mathrm{A}_n^k$}
	{$\mathrm{C}_n^k$}
	{$n\mathrm{A}_n^k$}
	{$n\mathrm{C}_n^k$}
	\loigiai{
		Số các chỉnh hợp chập $k$ của $n$ phần tử là $\mathrm{A}_n^k$.
	}
\end{ex}%!Cau!%
\begin{ex}%[Đề số 4, 2019]%[Phạm An Bình, 12EX8]%[1D2Y2-6]
	Với $k$ và $n$ là hai số nguyên dương tùy ý thỏa mãn $k\le n$, mệnh đề nào dưới đây đúng?
	\choice
	{\True $\mathrm{C}_n^k=\dfrac{n!}{k!(n-k)!}$}
	{$\mathrm{P}_n^k=\dfrac{n!}{k!(n-k)!}$}
	{$\mathrm{C}_n^k=\dfrac{n!}{k!}$}
	{$\mathrm{P}_n^k=\dfrac{n!}{k!}$}
	\loigiai{
		Ta có $\mathrm{C}_n^k=\dfrac{n!}{k!(n-k)!}$ và $\mathrm{P}_n^k=\dfrac{n!}{(n-k)!}$.
	}
\end{ex}%!Cau!%
\begin{ex}%[Phát triển đề số 5]%[Đoàn Minh Tân, EX8]%[1D2Y2-1]
	Với $k$ và $n$ là hai số nguyên dương tùy ý thỏa mãn $k\le n$, mệnh đề nào dưới đây là đúng?
	\choice
	{$\mathrm{A}_n^k = \dfrac{n!}{k!(n-k)!}$}
	{\True $\mathrm{A}_n^k = \dfrac{n!}{(n-k)!}$}
	{$\mathrm{A}_n^k = \dfrac{n!}{k!}$}
	{$\mathrm{A}_n^k = \dfrac{k!}{n! (n-k)!}$}
	\loigiai{
		Theo lý thuyết, ta có công thức $\mathrm{A}_n^k = \dfrac{n!}{(n-k)!}$.
	}
\end{ex}