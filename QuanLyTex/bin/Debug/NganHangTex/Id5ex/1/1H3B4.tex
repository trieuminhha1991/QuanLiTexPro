%!Cau!%
\begin{ex}%[De tap huan, So GD&DT Dien Bien, 2019]%[Ngoc Diep, dự án EX5]%[1H3B4-3]
	Cho hình lăng trụ đều $ABC.A'B'C'$ có $AB =\sqrt{3}$ và $AA' =1$. Tính góc tạo bởi đường thẳng $AC'$ và mặt phẳng $(ABC)$.
	\choice
	{$45^\circ$}
	{$60^\circ$}
	{\True $30^\circ$}
	{$75^\circ$}
	\loigiai{
		\immini{
			Ta có $\left( AC', (ABC)\right) = (AC',AC) = \widehat{CAC'} $. \\
			Lại có $\tan\widehat{CAC'} = \dfrac{CC'}{AC} =\dfrac{1}{\sqrt{3}}$, suy ra $\widehat{CAC'} =30^{ \circ }$.}{
			\begin{tikzpicture}[>=stealth,scale=0.7, line join=round, line cap=round,thick,font=\footnotesize]
			\tkzDefPoints{0/0/A, 2/-2/B, 5/0/C, 0/4/z}
			\coordinate (A') at ($(A)+(z)$);
			\coordinate (B') at ($(B)+(z)$);
			\coordinate (C') at ($(C)+(z)$);
			\tkzDrawPolygon(A',B',C')
			\tkzDrawSegments(A',A A,B B,C B,B' C,C')
			\tkzDrawSegments[dashed](A,C)
			\tkzLabelPoints[below right](B, C, B', C')
			\tkzLabelPoints[left](A, A')
			\tkzDrawPoints(A,B,C,A',B',C') 
			\end{tikzpicture}}
	}
\end{ex}%!Cau!%
\begin{ex}%[Hải Phòng, 2018]%[Phan Anh Tiến, 12-EX-05]%[1H3B4-3]
	Cho hình chóp đều $S.ABCD$ có cạnh đáy bằng $a$, cạnh bên bằng $a\sqrt{2}$. Góc giữa đường thẳng $SB$ và mặt phẳng $(ABCD)$ bằng bao nhiêu?
	\choice
	{$ 30^{\circ}$}
	{$ 45^{\circ} $}
	{\True $ 60^{\circ} $}
	{$ 90^{\circ} $}
	\loigiai{\immini{Gọi $O$ là tâm hình vuông $ABCD$, ta có $SO\perp(ABCD)$.\\
		Khi đó $\left( SB,(ABCD)\right) =\widehat{SBO}$, với $\cos\widehat{SBO}=\dfrac{BO}{SB}=\dfrac{\frac{a\sqrt{2}}{2}}{a\sqrt{2}}=\dfrac{1}{2}$. Suy ra $\left( SB,(ABCD)\right) =60^{\circ}$.
	}
		{\begin{tikzpicture}[scale=0.7, line join = round, line cap = round]
			\tikzset{label style/.style={font=\footnotesize}}
			\tkzDefPoints{0/0/D,5/0/C,2/2/A}
			\coordinate (B) at ($(A)+(C)-(D)$);
			\tkzInterLL(A,C)(B,D)    \tkzGetPoint{O}
			\coordinate (S) at ($(O)+(0,5)$);
			\tkzDrawPolygon(S,B,C,D)
			\tkzDrawSegments(S,C)
			\tkzDrawSegments[dashed](A,S A,B A,D A,C B,D S,O)
			\tkzDrawPoints(D,C,A,B,O,S)
			\tkzLabelPoints[above](S)
			\tkzLabelPoints[left](A,D)
			\tkzLabelPoints[right](B,C)
			\tkzLabelPoints[above right](O)
		\end{tikzpicture}
}
		}
\end{ex}%!Cau!%
\begin{ex}%[Đề ĐK 12 Nguyễn Khuyến, HCM, ngày 24 tháng 03 năm 2019]%[Vinh Vo, 12EX7-2019]%[1H3B4-6]
	Cho lăng trụ tam giác đều $ ABC.A'B'C' $ có tất cả các cạnh đều bằng $ a $. Tính cô-sin của góc tạo bởi hai đường thẳng $ BC $ và $ AB' $.
	\choice
	{$ \dfrac{1}{2} $}
	{$ \dfrac{3}{4} $}
	{$ \dfrac{2}{3} $}
	{\True $ \dfrac{ \sqrt{2} }{4} $}
	\loigiai{
	\immini{
		Ta thấy $ (AB',BC) = (AB',B'C') $.\\
		Ta có \allowdisplaybreaks{
			\begin{eqnarray*}
				\cos (AB',BC) & = & \left | \cos \widehat{AB'C'} \right | \\
				& = & \dfrac{ \left | AB'^2 + B'C'^2 - AC'^2 \right | }{ 2AB' \cdot B'C'} \\
				& = & \dfrac{ \sqrt{2} }{4}.
			\end{eqnarray*}	
		}
	}{
		\begin{tikzpicture}
		\tkzDefPoint(0,0){A}
		\tkzDefShiftPoint[A](-10:3){C}
		\tkzDefShiftPoint[A](-150:2){B}
		\tkzDefShiftPoint[A](90:2){A'}
		
		\tkzDefPointsBy[translation = from A to A'](B,C){B',C'}
		
		
		\tkzDrawPoints[fill = black](A,B,C,A',B',C')
		\tkzDrawSegments[dashed](A,B A,A'  A,C A,C'  A,B')
		\tkzDrawSegments(B,B' C,C'  B,C A',B' B',C' C',A')
		\tkzLabelPoints[above](A',B',C')
		\tkzLabelPoints[below](A,B,C)
		\end{tikzpicture}
	}	
}
\end{ex}%!Cau!%
\begin{ex}%[Đề ĐK 12 Nguyễn Khuyến, HCM, ngày 24 tháng 03 năm 2019]%[Vinh Vo, 12EX7-2019]%[1H3B4-3]
	Cho hình chóp tứ giác đều $ S.ABCD $ có tất cả các cạnh đều bằng $ a $. Gọi $ \varphi $ là góc tạo bởi mặt bên và mặt đáy của hình chóp. Giá trị của $ \cos \varphi $ là 
	\choice
	{$ \dfrac{ \sqrt{3} }{2} $}
	{$ \dfrac{1}{ \sqrt{2} } $}
	{\True $ \dfrac{1}{ \sqrt{3} } $}
	{$ \dfrac{2}{3} $}
	\loigiai{
	\immini{
	Gọi $ O $ là tâm hình vuông $ ABCD $.\\
	Gọi $ I $ là trung điểm $ CD $.\\
	Khi đó, góc giữa mặt bên $ (SCD) $ và $ (ABCD) $ là $ \widehat{SIO} $.\\
	Ta có $ \heva{& OI = \dfrac{a}{2} \\ & SI = \dfrac{a \sqrt{3} }{2}} \Rightarrow \cos \varphi  = \cos \widehat{SIO} = \dfrac{ OI }{SI} = \dfrac{1}{ \sqrt{3} } $.	
	}{
		\begin{tikzpicture}[>=stealth,line cap=round,line join=round,x=1cm,y=1cm,scale=1]
		\tkzDefPoint(0,0){A}
		\tkzDefShiftPoint[A](0:3){D}
		\tkzDefShiftPoint[A](-150:2){B}
		\tkzDefShiftPoint[B](0:3){C}
		\tkzDefMidPoint(A,C)
		\tkzGetPoint{O}
		\tkzDefMidPoint(D,C)
		\tkzGetPoint{I}
		\tkzDefShiftPoint[O](90:3){S}
		\tkzDrawPoints[fill = black](A,B,C,D,O,S,I)
		\tkzDrawSegments(S,B S,C S,D B,C C,D S,I)
		\tkzDrawSegments[dashed](S,A A,B A,D S,O A,C B,D O,I)
		\tkzLabelPoints[below](B,C,D,I)
		\node at ($ (A) + (0,0.1) $)[left]{$ A $};
		\node at ($ (O) + (0,0) $)[below]{$ O $};
		\tkzLabelPoints[above](S)
		\end{tikzpicture}
	}	
}
\end{ex}%!Cau!%
\begin{ex}%[Thi thử, trường THPT Nguyễn Công Trứ, tỉnh Hà Tĩnh, 2019]%[Phạm Doãn Lê Bình, 12EX7-19]%[1H3B4-3]
	Cho hình chóp $S.ABCD$ có đáy $ABCD$ là hình thoi cạnh $a$ và có $SA=SB=SC=a$. Góc giữa hai mặt phẳng $(SBD)$ và $(ABCD)$ bằng
	\choice
	{$30^\circ$}
	{\True $90^\circ$}
	{$60^\circ$}
	{$45^\circ$}
	\loigiai{
	\immini{
	Gọi $H$ là hình chiếu vuông góc của $S$ lên mặt phẳng $(ABCD)$.
	Do $SA=SB=SC$ nên $H$ là tâm đường tròn ngoại tiếp tam giác $ABC$. Mà tam giác $ABC$ có $BD$ là đường trung trực của $AC$ nên $H\in BD$. \\
	Do $SH\subset (SBD)$ và $SH \perp (ABCD)$ nên $(SBD)\perp (ABCD)$. Suy ra góc giữa $(SBD)$ và $(ABCD)$ bằng $90^\circ$.
	}{
	\begin{tikzpicture}[scale=1, font=\footnotesize, line join=round, line cap=round, >=stealth]
	\tkzDefPoints{0/0/A,3/0/B,2/-1/C}
	\tkzDefPointBy[translation = from B to C](A)\tkzGetPoint{D}
	\tkzDefPointBy[homothety = center B ratio 0.333](D)\tkzGetPoint{H}
	\coordinate (S) at ($(H)+(0,3)$);
	\tkzDrawSegments(S,C S,B S,D B,C C,D)
	\tkzDrawSegments[dashed](A,B S,H A,C B,D S,A A,D)
	\tkzDrawPoints[fill=black](C,A,B,S, D, H)
	\tkzLabelPoints[above](S)
	\tkzLabelPoints[below](A,B,C,H,D)
	\tkzMarkRightAngle[size=0.2](S,H,B)			
	\end{tikzpicture}
	}}
\end{ex}%!Cau!%
\begin{ex}%[Thi thử L2, THPT  Ngô Quyền-Hải Phòng, 2019]%[KV Thanh, 12EX7]%[1H3B4-3]
Cho hình chóp $S.ABC$ có đáy là tam giác đều cạnh $a$, $SA\perp (ABC)$, góc giữa hai mặt phẳng $(SBC)$ và $(ABC)$ bằng $60^{\circ}$. Đô dài cạnh $SA$ bằng 
\choice
{\True $\dfrac{3a}{2}$}
{$\dfrac{a}{2}$}
{$a\sqrt{3}$}
{$\dfrac{a}{\sqrt{3}}$}
\loigiai{
\immini{
Gọi $H$ là trung điểm của $BC$. Khi đó, ta có\\
 $AH\perp BC$ và $SH\perp BC$.\\
Suy ra $\left((SBC),(ABC)\right)=\widehat{SHA}=60^{\circ}$.\\
Do tam giác $ABC$ đều cạnh $a$ nên $AH=\dfrac{a\sqrt{3}}{2}$.\\
Vậy $SA=AH\cdot\tan 60^{\circ}=\dfrac{a\sqrt{3}}{2}\cdot\sqrt{3}=\dfrac{3a}{2}$.
}
{
\begin{tikzpicture}[scale=1, font=\footnotesize, line join=round, line cap=round, >=stealth] 
	\tkzDefPoints{0/0/A, 0/3/S, 4/0/C, 1/-2/B}
	\tkzDefMidPoint(B,C)\tkzGetPoint{H}
	\tkzDrawSegments(S,A A,B B,C S,B S,C S,H)
	\tkzDrawSegments[dashed](A,H A,C)
	\tkzDrawPoints[fill=black](S,A,B,C,H)
	\tkzLabelPoints[left](S,A)
	\tkzLabelPoints[right](B,C,H)
	\tkzMarkRightAngles(S,A,B S,A,C S,H,B A,H,B)
	\tkzMarkAngles[arc=l,size=0.7cm](S,H,A)
	\tkzLabelAngle[pos=0.9](S,H,A){$60^{\circ}$}
	\tkzMarkSegments[mark=|](B,H H,C)
	\end{tikzpicture}
}
}
\end{ex}%!Cau!%
\begin{ex}%[Thử sức trước kì thi đề số 4, THTT, 2019]%[Trần Hòa, 12EX-7-2019]%[1H3B4-4]
	Cho tứ diện đều $ABCD$ có cạnh $a$, gọi $G$ là trọng tâm tam giác $ABC$. Cắt tứ diện bởi mặt phẳng $(GCD)$ được thiết diện có diện tích là
	\choice
	{$\dfrac{a^2\sqrt{3}}{4}$}
	{$\dfrac{a^2\sqrt{2}}{2}$}
	{$\dfrac{a^2\sqrt{2}}{6}$}	
	{\True $\dfrac{a^2\sqrt{2}}{4}$}
	\loigiai{
	\immini{$CG$ cắt $AB$ tại $I$ là trung điểm của $AB$. Thiết diện là tam giác cân $ICD$. Ta có
	$IC=ID=\dfrac{a\sqrt{3}}{2}$.\\
	Gọi $E$ là trung điểm $CD$ suy ra\\
	$IE=\sqrt{IC^2-EC^2}=\sqrt{\dfrac{3a^2}{4}-\dfrac{a^2}{4}}=\dfrac{a\sqrt{2}}{2}$.\\
	$S_{ICD}=\dfrac{1}{2}MH\cdot CD=\dfrac{1}{2}\cdot\dfrac{a\sqrt{2}}{2}\cdot a=\dfrac{a^2\sqrt{2}}{4}$.}{
	\begin{tikzpicture}[scale=1,font=\footnotesize,line join = round, line cap = round, >= stealth]
\tkzDefPoints{0/0/B, 4/0/x, .5/-1.5/y, 0/4/z}
\coordinate (D) at ($(B)+(y)$);
\coordinate (C) at ($(B)+(x)$);
\coordinate (E) at ($(D)!0.5!(C)$);
\coordinate (H) at ($(B)!2/3!(E)$);
\coordinate (A) at ($(H)+(z)$);
\coordinate (I) at ($(A)!0.5!(B)$);
\coordinate (G) at ($(I)!1/3!(C)$);
\tkzDrawPolygon(A,B,D)
\tkzDrawSegments[dashed](C,I I,E B,C)
\tkzDrawSegments[](A,C C,D D,I)
\tkzLabelPoints[left](B,I)
\tkzLabelPoints[right](C)
\tkzLabelPoints[above](A,G)
\tkzLabelPoints[below](D,E)
\tkzDrawPoints[fill=black](A,B,C,D,I,E,G)
\tkzMarkRightAngle(A,I,C)
\tkzMarkRightAngle(B,I,D)
\tkzMarkRightAngle(D,E,I)
\end{tikzpicture}
	
	}	
	}
\end{ex}%!Cau!%
\begin{ex}%[Thi thử, Sở GD và ĐT - Quảng Nam, 2019]%[Nguyện Ngô, 12EX8]%[1H3B4-3]
Cho hình lập phương $ABCDA'B'C'D'$. Gọi $\varphi$ là góc giữa hai mặt phẳng $(A'BD)$ và $(ABC)$. Tính $\tan\varphi$.
\choice
{$\tan\varphi=\dfrac{1}{\sqrt{2}}$}
{\True $\tan\varphi=\sqrt{2}$}
{$\tan\varphi=\sqrt{\dfrac{2}{3}}$}
{$\tan\varphi=\sqrt{\dfrac{3}{2}}$}
\loigiai{
\immini
{
\begin{itemize}
\item Gọi cạnh của hình lập phương là $a$, tâm của đáy là $O$.
\item Ta có $\heva{&(A'DB)\cap (ABC)=DB\\&OA'\perp DB\\&OA\perp DB}\\
\Rightarrow ((A'BD),(ABC))=(OA',OA)=\widehat{A'OA}$.
\item $\tan\varphi=\tan\widehat{A'OA}=\dfrac{AA'}{OA}=\dfrac{a}{\dfrac{a\sqrt{2}}{2}}=\sqrt{2}$.
\end{itemize}
}
{\begin{tikzpicture}[scale=0.6, font=\footnotesize, line join=round, line cap=round, >=stealth]
\tkzDefPoints{0/0/A', 8/0/B', -2/-2.5/D', 0/-6/A}
\tkzDefPointBy[translation = from A' to B'](D') \tkzGetPoint{C'}
\tkzDefPointBy[translation = from A' to A](B') \tkzGetPoint{B}
\tkzDefPointBy[translation = from A' to A](C') \tkzGetPoint{C}	
\tkzDefPointBy[translation = from A' to A](D') \tkzGetPoint{D}
\tkzDefMidPoint(A,C)\tkzGetPoint{O}
\tkzDrawSegments(A',B' B',C' C',D' D',A' D',D C',C B',B D,C B,C)
\tkzDrawSegments[dashed](A',A A',C A,D A,B A',B A',D D,B A,C A',O)
\tkzDrawPoints[fill=black](A',B',C',D',A,B,C,D,O)
\tkzLabelPoints[above left](A')
\tkzLabelPoints[left](A,D')
\tkzLabelPoints[right](B)
\tkzLabelPoints[below left](C',D)
\tkzLabelPoints[above right](B')
\tkzLabelPoints(C)
\tkzLabelPoints[below](O)
\end{tikzpicture}
}
}
\end{ex}%!Cau!%
\begin{ex}%[Đề KSCL lớp 12 môn Toán Sở giáo dục và đào tạo Thanh Hóa, 2018-2019]%[Cao Thành Thái, dự án 12-EX-8-2019]%[1H3B4-3]
 Cho hình chóp đều $S.ABCD$ có tất cả các cạnh bằng $a$. Gọi $M$ là trung điểm của $SC$. Tính góc $\varphi$ giữa hai mặt phẳng $(MBD)$ và $(ABCD)$.
 \choice
  {$\varphi = 60^\circ$}
  {$\varphi = 30^\circ$}
  {\True $\varphi = 45^\circ$}
  {$\varphi = 90^\circ$}
 \loigiai
  {
  \immini
  {
  Gọi $O$ là tâm của hình vuông $ABCD$. Khi đó $SO \perp (ABCD)$.\\
  Vì $BD \perp SO$ và $BD \perp AC$ nên $BD \perp (SAC)$.\\
  Suy ra $BD \perp OM$.\\
  Lại có $OC \perp BD$ và $(MBD)\cap (ABCD)=BD$.\\
  Vậy góc giữa $(MBD)$ và $(ABCD)$ bằng góc giữa $OM$ và $OC$.\\
  Ta có $OM=CM=\dfrac{a}{2}$ và $OC=\dfrac{a\sqrt{2}}{2}$.\\
  Trong tam giác $OMC$ có $OM^2 + CM^2 = \dfrac{a^2}{2} = OC^2$.\\
  Vậy tam giác $OMC$ vuông cân tại $M$.
  }
  {
  \begin{tikzpicture}[line join=round, line cap=round, >=stealth,font=\footnotesize, scale=0.8]
   \tkzDefPoints{0/0/A, 4/0/D, -1.5/-2/B}
   \tkzDefPointBy[translation=from A to D](B)\tkzGetPoint{C}
   \tkzDefMidPoint(B,D)\tkzGetPoint{O}
   \tkzDefShiftPoint[O](90:5){S}
   \tkzDefMidPoint(S,C)\tkzGetPoint{M}
   \tkzDrawPoints[fill=black](A,B,C,D,O,S,M)
   \tkzDrawSegments(S,B S,C S,D B,C C,D M,B M,D)
   \tkzDrawSegments[dashed](S,A A,C B,D A,B A,D S,O M,O)
   \node[above] at (S){$S$};
   \node[left] at (A){$A$};
   \node[below left] at (B){$B$};
   \node[below right] at (C){$C$};
   \node[right] at (D){$D$};
   \node[below] at (O){$O$};
   \node[above right] at (M){$M$};
  \end{tikzpicture}
  }
  \noindent
  Do đó góc tạo bởi $OM$ và $OC$ bằng $\widehat{COM} = 45^\circ$ hay $\varphi = 45^\circ$.
  }
\end{ex}%!Cau!%
\begin{ex}%[Chuyên ĐHSPHN - 19]%[Phan Anh - EX8]%[1H3B4-3]
Cho hình lập phương $ABCD.A'B'C'D'$. Góc giữa hai mặt phẳng $(BCD'A')$ và $(ABCD)$ bằng
\choice
{\True $45^\circ$}
{$30^\circ$}
{$90^\circ$}
{$60^\circ$}
\loigiai{\immini{Ta có $\heva{&(BCD'A')\cap(ABCD)=BC\\&BC\perp(ABB'A')\\&(BCD'A')\cap(ABB'A')=A'B\\&(ABCD)\cap(ABB'A')=AB.}$\\
		Nên suy ra $$((BCD'A'),(ABCD))=(AB;A'B)=45^\circ.$$}
	{\begin{tikzpicture}[scale=0.6, font=\footnotesize, line join=round, line cap=round, >=stealth]
\tkzDefPoints{0/0/A,-2/-1.5/B, 5/0/D, 0/5/A'}
\coordinate (C) at ($(B)+(D)-(A)$);
\coordinate (B') at ($(B)+(A')-(A)$);
\coordinate (C') at ($(C)+(A')-(A)$);
\coordinate (D') at ($(D)+(A')-(A)$);
\tkzDrawSegments(B,B' C,C' D,D' A',B' B',C' C',D' D',A' B,C C,D C,D')
\tkzDrawSegments[dashed](A,A' A,B A,D A',B)
\tkzDrawPoints[fill=black](A,B,C,D,A',B',C',D')
\tkzLabelPoints[above left](B',C',A',A)
\tkzLabelPoints[above right](D,D')
\tkzLabelPoints[below left](B)
\tkzLabelPoints[below right](C)
\end{tikzpicture}}
}
\end{ex}%!Cau!%
\begin{ex}%[thi thử, THPT Triệu Thái, Vĩnh Phúc]%[Phan Quốc Trí, dự án 12EX-8-2019]%[1H3B4-3]
	Cho hình lập phương $ABCD.A'B'C'D'$. Góc giữa hai mặt phẳng $(ADD'A')$ và $(ABC'D')$ bằng	
	\choice
	{$30^{\circ}$}
	{$60^{\circ}$}
	{$45^{\circ}$}
	{\True $90^{\circ}$}
	\loigiai{
		\immini{
			Ta có $\heva{&AB \perp AA'\\&AB \perp AD} \Rightarrow AB \perp (ABB'A') $\\ $\Rightarrow (ADD'A') \perp (ABC'D')$.\\
			Do đó  góc giữa hai mặt phẳng $(ADD'A')$ và $(ABC'D')$ bằng $90^{\circ}$.	
		}{
			\begin{tikzpicture}[scale=1, font=\footnotesize, line join=round, line cap=round, >=stealth]
			\tkzDefPoints{0/0/A, 3/0/B, -1/-1./D, 3/2/E}
			\tkzDefPointBy[translation=from A to D](B)\tkzGetPoint{C}
			\tkzDefPointBy[translation=from B to E](B)\tkzGetPoint{B'}
			\tkzDefPointBy[translation=from B to E](C)\tkzGetPoint{C'}
			\tkzDefPointBy[translation=from B to E](D)\tkzGetPoint{D'}
			\tkzDefPointBy[translation=from B to E](A)\tkzGetPoint{A'}
			\tkzDrawPolygon(A',B',C',D')
			\tkzDrawSegments(B,B' B,C C,C' C,D D,D')
			\tkzDrawSegments[dashed](A,B A,A' A,D A,D' B,C')
			\tkzLabelPoints(C,D)
			\tkzLabelPoints[above](B',C',D')
			\tkzLabelPoints[left](A')
			\tkzLabelPoints[below](A,B)
			\tkzDrawPoints[fill=black](A,B,C,D,A',B',C',D')
			\end{tikzpicture}
		}		
	}
\end{ex}%!Cau!%
\begin{ex}%[Thi thử L2, THPT Hà Huy Tập - Hà Tĩnh, 2019]%[Phan Ngọc Toàn, dự án EX8]%[1H3B4-3] 
Cho hình chóp $S.ABCD$ có đáy $ABCD$ là hình vuông cạnh $a$, $SA\perp (ABCD)$ và  $SA=\dfrac{a\sqrt{6}}{3}$. Tính góc giữa $SC$ và $(ABCD)$.
	\choice
	{\True $30^\circ$}	
	{ $45^\circ$}	
	{$60^\circ$}	
	{$30^\circ$}
	\loigiai{
\immini{Vì $SA\perp (ABCD)$ nên $AC$ là hình chiếu vuông góc của $SC$ lên $(ABCD)$, suy ra góc giữa $SC$ và $(ABCD)$ là $\widehat{SCA}$. \\
Do đó $\tan \widehat{SCA}=\dfrac{SA}{AC}=\dfrac{\sqrt{3}}{3}\Rightarrow \widehat{SCA}=30^\circ$.
}{
\begin{tikzpicture}[scale=0.3, font=\footnotesize, line join=round, line cap=round, >=stealth]
\tkzDefPoints{0/0/D,7/0/C,3/3/A}
\coordinate (B) at ($(A)+(C)-(D)$);
\coordinate (S) at ($(A)+(0,6)$);
\tkzDrawPolygon(S,B,C,D)
\tkzDrawSegments(S,C)
\tkzDrawSegments[dashed](A,S A,B A,D A,C)
\tkzDrawPoints(D,C,A,B,S)
\tkzLabelPoints[above](S)
\tkzLabelPoints[left](A,D)
\tkzLabelPoints[right](B,C)
\tkzMarkAngles[size=1.5cm,arc=ll,mark=|](S,C,A)
\end{tikzpicture}
} }
\end{ex}%!Cau!%
\begin{ex}%[Thi thử L2, THPT Hà Huy Tập - Hà Tĩnh, 2019]%[Phan Ngọc Toàn, dự án EX8]%[1H3B4-1]
Có một khối đá trắng hình lập phương được sơn đen toàn bộ mặt ngoài. Người ta xẻ khối đá đó thành $125$ khối đá nhỏ bằng nhau và cũng là hình lập phương. Hỏi có bao nhiêu khối đá nhỏ mà không có mặt nào bị sơn đen?
\choice
{$45$}
{$48$}
{$36$}
{\True $27$} 
\loigiai{
Ta có $ 125-(2\cdot 25+3\cdot 16)=27 $.
}
\end{ex}%!Cau!%
\begin{ex}%[Đề thi thử THPT Quốc gia môn Toán năm 2019, Sở giáo dục và đào tạo Đà Nẵng]%[Cao Thành Thái, dự án 12-EX-9-2019]%[1H3B4-3]
	Trong hình chóp tam giác đều có góc giữa cạnh bên và mặt đáy bằng $60^{\circ}$, tang của góc giữa mặt bên và mặt đáy bằng
	\choice
	{$\dfrac{\sqrt{3}}{6}$}
	{$\sqrt{3}$}
	{$\dfrac{\sqrt{3}}{2}$}
	{\True $2\sqrt{3}$}
	\loigiai
	{
		\immini
		{
			Kí hiệu hình chóp $S.ABC$ là hình chóp tam giác đều.\\
			Gọi $M$ là trung điểm của $BC$, $G$ là trọng tâm của tam giác $ABC$.\\
			Khi đó $SG \perp (ABC)$, do đó góc giữa $SC$ với $(ABC)$ bằng góc giữa $SC$ và $GC$ chính là $\widehat{SCG} = 60^\circ$.\\
			Đặt $AB=a$. Ta có $AM = \dfrac{a\sqrt{3}}{2}$, $GC = \dfrac{a\sqrt{3}}{3}$, $GM = \dfrac{a\sqrt{3}}{6}$.\\
			Trong tam giác vuông $SGC$ ta có $$\tan\widehat{SCG} = \dfrac{SG}{GC} \Rightarrow SG = GC\cdot \tan\widehat{SCG} = \dfrac{a\sqrt{3}}{3} \cdot \sqrt{3} = a.$$
			Ta có $BC\perp AM$ và $BC\perp SG$ nên $BC \perp (SAM)$, suy ra $BC \perp SM$.\\
			Lại có $(SBC) \cap (ABC) = BC$.
		}
		{
			\begin{tikzpicture}[line join=round, line cap=round, >=stealth,font=\footnotesize, scale=1]
			\tkzDefPoints{0/0/A, 4/0/C, 2.5/-1.5/B}
			\tkzCentroid(A,B,C)\tkzGetPoint{G}
			\tkzDefShiftPoint[G](90:4){S}
			\tkzDefMidPoint(B,C)\tkzGetPoint{M}
			\tkzDrawPoints[fill=black, size=3pt](A,B,C,G,S,M)
			\tkzDrawSegments(S,A S,B S,C S,M A,B B,C)
			\tkzDrawSegments[dashed](S,G A,C A,M G,C)
			\tkzMarkSegments[mark=|, size=0.1cm](B,M C,M)
			\tkzMarkRightAngles[size=0.15](S,G,A A,M,B S,M,C)
			\node[above] at (S){$S$};
			\node[below] at (G){$G$};
			\node[below] at (B){$B$};
			\node[left] at (A){$A$};
			\node[right] at (C){$C$};
			\node[right] at (M){$M$};
			\end{tikzpicture}
		}
		\noindent
		Vậy góc giữa $(SBC)$ và $(ABC)$ bằng góc giữa $SM$ và $AM$, chính là $\widehat{SMG}$.\\
		Trong tam giác vuông $SMG$ ta có $\tan \widehat{SMG} = \dfrac{SG}{GM} = \dfrac{a}{\dfrac{a\sqrt{3}}{6}} = \dfrac{6}{\sqrt{3}} = 2\sqrt{3}$.
	}
\end{ex}%!Cau!%
\begin{ex}%[Đề thi thử THPT Chuyên Bến Tre, năm học 2018-2019]%[Tuan Nguyen, dự án 12-EX-9-2019]%[1H3B4-3]
	Cho hình lập phương $ABCD.A'B'C'D'$. Góc giữa hai mặt phẳng $(DA'B')$ và $(DC'B')$ bằng
	\choice{$45^\circ$}{$30^\circ$}{\True $60^\circ$}{$90^\circ$}
	\loigiai{
\immini
{
Không mất tính tổng quát, giả sử cạnh hình lập phương bằng $a$.\\
Dễ thấy $A'B'=B'C'=a, A'D=C'D=a\sqrt{2}, B'D=a\sqrt{3}$.\\
Ta có $A'C'\perp (BDD'B')$ nên $A'C'\perp B'D$.\\
Kẻ $A'H\perp B'D$ thì $B'D\perp (A'HC')$, vậy $B'D\perp C'H$.\\
Khi đó góc giữa hai mặt phẳng $(DA'B')$ và $(DC'B')$ bằng góc giữa hai đường thẳng $A'H$ và $C'H$.\\
Xét tam giác $A'HC'$ ta có\\
$A'C'=a\sqrt{2}, A'H=C'H=\dfrac{A'D\cdot A'B'}{B'D}=\dfrac{a\sqrt{6}}{3}$.\\
Vậy $\cos\widehat{A'HC'}=\dfrac{A'H^2+C'H^2-A'C'^2}{2A'H\cdot C'H}=-\dfrac{1}{2}\Rightarrow \widehat{A'HC'}=120^\circ$.\\
Vậy góc giữa hai mặt phẳng $(DA'B')$ và $(DC'B')$ bằng $60^\circ$.
}
{
	\begin{tikzpicture}[scale=.7, line join = round, line cap = round,>=stealth]
\tikzset{label style/.style={font=\footnotesize}}
\tkzDefPoints{0/0/D',5/0/C',2/2/A'}
\coordinate (B') at ($(A')+(C')-(D')$);
\tkzDefSquare(D',C')    \tkzGetPoints{C}{D}
\tkzDefSquare(A',B')    \tkzGetPoints{B}{A}
\coordinate (H) at ($(D)!.6!(B')$);
\tkzDrawPolygon(A,B,B',C',D',D)
\tkzDrawSegments(C,B C,D C,C' D,C')
\tkzDrawSegments[dashed](A',A A',B' A',D' D,A' D,B' A',H C',H A',C')
\tkzDrawPoints[fill=black](A',B',C',D',C,D,B,A,H)
\tkzLabelPoints[above](A,B,H)
\tkzLabelPoints[below](D',C')
\tkzLabelPoints[left](A',D)
\tkzLabelPoints[right](C,B')
\tkzMarkRightAngles[](D,H,A' C',H,B')
\end{tikzpicture}

}	
}
\end{ex}%!Cau!%
\begin{ex}%[Thi thử, Toán học tuổi trẻ - Đề 6, 2019]%[Phan Văn Thành, 12EX9]%[1H3B4-3]
	Cho hình hộp chữ nhật $ABCD.A'B'C'D'$ có $AB = 3a$, $AD = a\sqrt{3}$, $AA' = 2a$. Góc giữa đường thẳng $AC'$ với mặt phẳng $(A'B'C')$ bằng
	\choice
	{$ 60^\circ  $}
	{$ 45^\circ  $}
	{$  120^\circ $}
	{\True $  30^\circ $}
	\loigiai{
	\immini{Vì $AA' \perp (A'B'C')$\\$ \Rightarrow (AC',(A'B'C')) = (AC', A'C') = \widehat{AC'A'}$.\\
	Xét $\triangle A'B'C'$ vuông tại $B'$, ta có 
$$A'C' = \sqrt{A'B'^2 + B'C'^2} = \sqrt{9a^2 + 3a^2} = 2a\sqrt{3}.$$
Xét $\triangle AA'C'$ vuông tại $A'$, ta có
$$\tan \widehat{AC'A'} = \dfrac{AA'}{A'C'} = \dfrac{2a}{2a\sqrt{3}} = \dfrac{\sqrt{3}}{3} \Rightarrow \widehat{AC'A'} = 30^\circ.$$
Vậy $(AC',(A'B'C')) = \widehat{AC'A'} = 30^\circ$.}{	\begin{tikzpicture}[scale=0.7, line join=round, line cap=round, >=stealth]
\tikzset{label style/.style={font=\footnotesize}}
\tkzDefPoints{0/0/A,6/0/B,-3/-2/D}
\coordinate (C) at ($(B)+(D)-(A)$);
\coordinate (A') at ($(A) - (0,5)$);
\tkzDefPointsBy[translation = from A to A'](B,C,D){B'}{C'}{D'}
\tkzDrawPolygon(A,B,B',C',D',D)
\tkzDrawSegments(C,B C,D C,C')
\tkzDrawSegments[dashed](A',A A',B' A',D' A,C' A',C')
\tkzDrawPoints[fill=black](A,B,D,C,A',B',C',D')
\tkzLabelPoints[above](A,B,C)
\tkzLabelPoints[below](D',C')
\tkzLabelPoints[left](A',D)
\tkzLabelPoints[right](B')
\tkzMarkAngles[size=1cm,arc=ll](A,C',A')
\end{tikzpicture}}

	}
\end{ex}%!Cau!%
\begin{ex}%[Dự án 12-EX-8-2019, Nguyễn Anh Quốc]%[1H3B4-3]
	Cho hình lập phương $ABCD.A'B'C'D'$. Góc giữa hai mặt phẳng $(A'B'CD)$ và $(CDD'C')$ bằng
	\choice
	{$30^\circ$}
	{$60^\circ$}
	{\True $45^\circ$}
	{$90^\circ$}
	\loigiai{
		\immini{Ta thấy $(A'B'CD)\cap (CDD'C')=CD$, $B'C\perp CD$, $CC'\perp CD$ nên góc giữa hai mặt phẳng $(A'B'CD)$ và $(C'CDD')$ là góc giữa $B'C$ và $CC'$ là $\widehat{B'CC'}=45^\circ$. 
		}
		{\begin{tikzpicture}[scale=0.8,>=stealth, font=\footnotesize, line join=round, line cap=round]
			\tkzDefPoints{0/0/A,-1.3/-1.1/B,2/-1.1/C}
			\coordinate (D) at ($(A)+(C)-(B)$);
			\coordinate (A') at ($(A)+(0,2.5)$);
			\tkzDefPointsBy[translation=from A to A'](B,C,D){B'}{C'}{D'}
			\tkzDrawPolygon(A',B',B,C,D,D')
			\tkzDrawSegments(B',C' C',D' C,C' B',C)
			\tkzDrawSegments[dashed](A,B A,D A,A' A',D)
			\tkzDrawPoints[fill=black,size=4](A,B,D,C,A',B',C',D')
			\tkzLabelPoints[above](A',D')
			\tkzLabelPoints[below](A,B,C)
			\tkzLabelPoints[left](B')
			\tkzLabelPoints[right](C',D)
			\tkzMarkAngles[size=0.7cm,arc=l](C',C,B')
			
			\end{tikzpicture}}
	}
\end{ex}%!Cau!%
\begin{ex}%[Đề dự đoán số 2]%[Nguyễn Thành Khang, dự án 2019-12-Ex-8]%[1H3B4-3]
	Cho lăng trụ $ABC.A'B'C'$ có đáy $ABC$ là tam giác đều cạnh $a$ và $A'A=A'B=A'C=\dfrac{a\sqrt{15}}{6}$. Góc giữa hai mặt phẳng $(ABB'A')$ và $(ABC)$ bằng
	\choice
	{$30^{\circ}$}
	{\True $45^{\circ}$}
	{$60^{\circ}$}
	{$75^{\circ}$}
	\loigiai{ 
		\immini{
			Gọi $M$ là trung điểm $AB$, $H$ là trọng tâm tam giác $ABC$. Khi đó $A'H\perp(ABC)$ và $(A'HM) \perp AB$.\\
			Suy ra $\left((ABB'A');(ABC)\right)=\widehat{A'MH}$.\\
			Ta có $MH=\dfrac{CM}{3}=\dfrac{a\sqrt{3}}{6}$ và $CH=2MH=\dfrac{a\sqrt{3}}{3}$,\\
			suy ra $A'H=\sqrt{A'C^2-CH^2}=\dfrac{a\sqrt{3}}{6}$.\\
			Xét $\triangle A'MH$, ta có $\tan\widehat{A'MH}=\dfrac{A'H}{MH}=1$.\\
			Vậy $\left((ABB'A');(ABC)\right)=45^\circ$.
		}{
			\begin{tikzpicture}[scale=1, font=\footnotesize, line join=round, line cap=round,>=stealth]
			\tkzInit[xmin=-0.5, xmax=5.5, ymin=-2, ymax=4]
			\tkzClip
			\tkzDefPoints{0/0/A,0.8/-1.5/B,3.7/0/C,1.5/3/A'}
			\tkzDefMidPoint(A,B)\tkzGetPoint{M}
			\tkzCentroid(A,B,C)\tkzGetPoint{H}
			\tkzDefPointBy[translation=from A to B](A')\tkzGetPoint{B'}
			\tkzDefPointBy[translation=from A to C](A')\tkzGetPoint{C'}
			\tkzDrawPoints[fill=black](A,B,C,A',B',C',M,H)
			\tkzDrawSegments(A,B B,C B,A' A,A' B,B' C,C' A',B' B',C' C',A' A',M)
			\tkzDrawSegments[dashed](C,M A',H A',C A,C)
			\tkzLabelPoints[above](A',C')
			\tkzLabelPoints[below](B,H,C)
			\tkzLabelPoints[left](A,M,B')
			\end{tikzpicture}
		}
	}
\end{ex}%!Cau!%
\begin{ex}%[Đề số 4, 2019]%[Phạm An Bình, 12EX8]%[1H3B4-3]
	Cho hình lập phương $ABCD.A'B'C'D'$. Góc giữa hai mặt phẳng $(ABB'A')$ và $(ACC'A')$ là
	\choice
	{\True $45^\circ$}
	{$90^\circ$}
	{$30^\circ$}
	{$60^\circ$}
	\loigiai{
		\immini{
			Ta có $\heva{&(ABB'A')\cap (ACC'A')=AA'\\&A'C' \perp AA'\\&A'B'\perp AA'}$\\
			$\Rightarrow [(ABB'A'),(ACC'A')]=\widehat{B'A'C'}=45^\circ$.
		}{
			\begin{tikzpicture}[scale=1, font=\footnotesize, line join=round, line cap=round, >=stealth]
			\pgfmathsetmacro\a{2}
			\pgfmathsetmacro\b{\a*2/3}
			\pgfmathsetmacro\h{\a}
			
			\tkzDefPoint(0,0){A}
			\tkzDefShiftPoint[A](0:\a){B}
			\tkzDefShiftPoint[A](-140:\b){D}
			\coordinate (C) at ($(D)+(B)-(A)$);
			\tkzDefShiftPoint[A](90:\h){A'}
			\coordinate (B') at ($(B)+(A')-(A)$);
			\coordinate (C') at ($(C)+(A')-(A)$);
			\coordinate (D') at ($(D)+(A')-(A)$);
			\tkzDrawPolygon(A',B',B,C,D,D',A')
			\tkzDrawSegments(C',B' C',D' C',C A',C')
			\tkzDrawSegments[dashed](A,B A,D A,A' A,C)
			\tkzDrawPoints[fill=black](A,B,C,D,A',B',C',D')
			\tkzLabelPoints[right](B,B')
			\tkzLabelPoints[left](A,A',D,D')
			\tkzLabelPoints[below right=-.1](C,C')
			\tkzMarkRightAngles(B,A,A')
			\end{tikzpicture}
		}
	}
\end{ex}%!Cau!%
\begin{ex}%[Đề dự đoán số 6 từ câu 1 đến 35]%[Đăng Tạ, dự án EX-8]%[1H3B4-3]
	Cho hình lập phương $ABCD.MNPQ$. Góc giữa hai mặt phẳng $\left( MNCD\right)$ và  $\left( ABPQ\right)$ là
	\choice
	{$30^{\circ}$}
	{$45^{\circ}$}
	{$60^{\circ}$}
	{\True $90^{\circ}$}
	\loigiai{
		\immini{
			$\heva{&CD\perp AD\\&CD\perp DQ} \Rightarrow CD\perp \left(ADQ \right) \Rightarrow CD \perp AQ \quad (1)$.\\
			Mặt khác $ADQM$ là hình vuông nên $DM \perp AQ \quad (2)$.\\
			Từ $(1)$ và $(2)$ suy ra $AQ \perp \left(MNCD \right)$. Mà $AQ \subset \left( ABPQ\right) $ nên $\left( ABPQ\right) \perp \left( MNCD\right)$.\\
			Vậy góc giữa hai mặt phẳng $\left( MNCD\right)$ và  $\left( ABPQ\right)$ là $90^{\circ}$.
		}{
			\begin{tikzpicture}[scale=.7, line join = round, line cap = round]
			\tikzset{label style/.style={font=\footnotesize}}
			\tkzDefPoints{0/0/Q,5/0/P,2/2/M}
			\coordinate (N) at ($(M)+(P)-(Q)$);
			\tkzDefSquare(Q,P)    \tkzGetPoints{C}{D}
			\tkzDefSquare(M,N)    \tkzGetPoints{B}{A}
			\tkzDrawPolygon(A,B,N,P,Q,D)
			\tkzDrawSegments(C,B C,D C,P C,N B,P)
			\tkzDrawSegments[dashed](M,A M,N M,Q D,M A,Q)
			\tkzDrawPoints(M,N,P,Q,C,D,B,A)
			\tkzLabelPoints[above](A,B)
			\tkzLabelPoints[below](Q,P)
			\tkzLabelPoints[left](M,D)
			\tkzLabelPoints[right](C,N)
			\end{tikzpicture}
		}
	}
\end{ex}