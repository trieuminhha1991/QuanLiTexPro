%!Cau!%
\begin{ex}%[Đề tập huấn, Sở GD và ĐT - Quảng Ninh, 2019]%[Lê Hồng Phi, 12EX5]%[1H3G3-3]
	\immini{Cho hình chóp tứ giác đều $S.ABCD$ với tất cả các cạnh bằng $a$. Gọi $G$ là trọng tâm tam giác $SCD$ (tham khảo hình vẽ bên). Giá trị tan góc giữa $AG$ và $(ABCD)$ bằng 
		\choice
		{\True $\dfrac{\sqrt{17}}{17}$}
		{$\dfrac{\sqrt{5}}{3}$}
		{$\sqrt{17}$}
		{$\dfrac{\sqrt{5}}{5}$}}{\begin{tikzpicture}[scale=0.8, font=\footnotesize, line join=round, line cap=round,>=stealth]
		\tkzDefPoints{0/0/B, 5/0/C, 7/1.5/D, 0/4/h};
		\coordinate (A) at ($(B)+(D)-(C)$);
		\coordinate (O) at ($(A)!0.5!(C)$);
		\coordinate (S) at ($(O)+(h)$);
		\coordinate (I) at ($(C)!0.5!(D)$);
		\coordinate (G) at ($(S)!0.67!(I)$);	
		\coordinate (Q) at ($(O)!0.67!(I)$);	
		\tkzDrawPolygon(S,B,C,D);
		\tkzDrawSegments(S,C S,I);
		\tkzDrawSegments[dashed](S,A A,B A,D S,O O,I A,G A,C B,D G,Q);
		\tkzDrawPoints[fill=black](S,A,B,C,D,O,I,G,Q);
		\tkzLabelPoints[above](S);
		\tkzLabelPoints[below](B,C,O,Q);
		\tkzLabelPoints[right](D,G,I);
		\tkzLabelPoints[above left](A);
		\end{tikzpicture}}
	\loigiai
	{\immini{Gọi $O$ là tâm của hình vuông $ABCD$. Khi đó, $SO\perp (ABCD)$.\\
			Gọi $I$ là trung điểm của $CD$. Ta tính được $SI=\dfrac{a\sqrt{3}}{2}$, $SG=\dfrac{2}{3}SI=\dfrac{a\sqrt{3}}{3}$ và $$SO=\sqrt{SI^2-OI^2}=\sqrt{\left(\dfrac{a\sqrt{3}}{2}\right)^2-\left(\dfrac{a}{2}\right)^2}=\dfrac{a\sqrt{2}}{2}.$$}{\begin{tikzpicture}[scale=0.8, font=\footnotesize, line join=round, line cap=round,>=stealth]
			\tkzDefPoints{0/0/B, 5/0/C, 7/1.5/D, 0/4/h};
			\coordinate (A) at ($(B)+(D)-(C)$);
			\coordinate (O) at ($(A)!0.5!(C)$);
			\coordinate (S) at ($(O)+(h)$);
			\coordinate (I) at ($(C)!0.5!(D)$);
			\coordinate (G) at ($(S)!0.67!(I)$);	
			\coordinate (Q) at ($(O)!0.67!(I)$);	
			\tkzDrawPolygon(S,B,C,D);
			\tkzDrawSegments(S,C S,I);
			\tkzDrawSegments[dashed](S,A A,B A,D S,O O,I A,G A,C B,D G,Q A,Q A,I);
			\tkzDrawPoints[fill=black](S,A,B,C,D,O,I,G,Q);
			\tkzLabelPoints[above](S);
			\tkzLabelPoints[below](B,C,O,Q);
			\tkzLabelPoints[right](D,G,I);
			\tkzLabelPoints[above left](A);
			\end{tikzpicture}}\noindent
	Gọi $Q$ là hình chiếu vuông góc của $G$ trên $(ABCD)$. Ta có $Q\in OI$ và $GQ=\dfrac{1}{3}SO=\dfrac{a\sqrt{2}}{6}$.\\
	Ta có 
	\begin{align*}
	AG^2&=SA^2+SG^2-2\cdot  SA\cdot SG\cdot\cos \widehat{ASG}=SA^2+SG^2-2\cdot SA\cdot SG\cdot\dfrac{SA^2+SI^2-AI^2}{2\cdot SA\cdot SI}\\
	&=SA^2+SG^2-\dfrac{2(SA^2+SI^2-(AD^2+ID^2))}{3}\\
	&=a^2+\dfrac{a^2}{3}-\dfrac{2\left(a^2+\dfrac{3a^2}{4}-a^2-\dfrac{a^2}{4}\right)}{3}=a^2.
	\end{align*}
	Khi đó, $AQ=\sqrt{AG^2-GQ^2}=\sqrt{a^2-\dfrac{2a^2}{36}}=\dfrac{a\sqrt{34}}{6}$.\\
	Vì $AG\cap (ABCD)=A$ và $GQ\perp (ABCD)$ nên góc giữa $AG$ và $(ABCD)$ là $\widehat{GAQ}$.\\
	Vậy $\tan\widehat{GAQ}=\dfrac{GQ}{AQ}=\dfrac{\dfrac{a\sqrt{2}}{6}}{\dfrac{a\sqrt{34}}{6}}=\dfrac{\sqrt{17}}{17}$.	
	}
\end{ex}