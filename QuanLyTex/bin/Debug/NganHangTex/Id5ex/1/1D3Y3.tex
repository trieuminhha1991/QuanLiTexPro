%!Cau!%
\begin{ex}%[DTH, Sở GD và ĐT - Hà Nam, 2019]%[Đào-V- Thủy, 12EX5]%[1D3Y3-1]
	Cho một cấp số cộng $(u_n)$ có $u_1= \dfrac{1}{3}$, $u_8= 26$. Tìm công sai $d$.
	\choice
	{\True $d= \dfrac{11}{3}$}
	{$d= \dfrac{10}{3}$}
	{$d= \dfrac{3}{10}$}
	{$d= \dfrac{3}{11}$}
	\loigiai{
		Ta có $u_8= u_1+ 7d \Leftrightarrow 26= \dfrac{1}{3}+ 7d \Leftrightarrow d=\dfrac{11}{3}$.
	}
\end{ex}%!Cau!%
\begin{ex}%[Thi thử, Sở GD và ĐT -Lạng Sơn, 2019]%[Trần Duy Khương, 12EX5-2019]%[1D3Y3-1]
		Cho cấp số cộng $1,8,15,22,29,\ldots$ Công sai của cấp số cộng này là 
		\choice
		{\True $7$}
		{$8$}
		{$9$}
		{$10$}
		\loigiai{Công sai của cấp số cộng này là $d=8-1=7$.	}
	\end{ex}%!Cau!%
\begin{ex}%[2-DTH-14-NINHBINH-19]%[Nguyễn Thế Anh, dự án EX5]%[1D3Y3-1]
	Cho cấp số cộng $(u_n)$ có $u_1=2018, u_2=2020$.Tìm công sai $d$ của cấp số cộng.
	\choice
	{$-2$}
	{$4038$}
	{\True $2$}
	{$-4038$}
	\loigiai{ Ta có $d=u_2-u_2=2.$
	}
	\end{ex}%!Cau!%
\begin{ex}%[Đề tập huấn Sở Ninh Bình, 2019]%[Nguyễn Văn Hải, dự án(12EX-5-2019)]%[1D3Y3-2]
Cho cấp số cộng $\left(u_n\right)_{n\geq 1}$ có $u_1=2018$, $u_2=2020$. Tìm công sai $d$ của cấp số cộng.
\choice
{$-2$}
{$4038$}
{\True $2$}
{$-4038$}
\loigiai{
Ta có $d=u_2-u_1=2$.
}
\end{ex}%!Cau!%
\begin{ex}%[Đề tập huấn, Bắc Kạn, 2018-2019]%[Cao Thành Thái, 12EX5-2019]%[1D3Y3-2]
 Công thức nào sau đây đúng với cấp số cộng có số hạng đầu $u_1$, công sai $d$?
 \choice
  {$u_n=u_1+d$}
  {$u_n=u_1+(n+1)d$}
  {$u_n=u_1-(n+1)d$}
  {\True $u_n=u_1+(n-1)d$}
 \loigiai
  {
  Công thức số hạng tổng quát của cấp số cộng $(u_n)$ có số hạng đầu $u_1$, công sai $d$ là $u_n=u_1+(n-1)d$.
  }
\end{ex}%!Cau!%
\begin{ex}%[Phát triển đề THPT Lương Thế Vinh, lần 3, 2019; Đoàn Mạnh Hùng]%[1D3Y3-1]
	Cho cấp số cộng $\left(u_n\right)$ có $u_1=-3$, $u_6=27$. Tính công sai $d$.
	\choice
	{$d=7$}
	{$d=8$}
	{$d=5$}
	{\True $d=6$}
	\loigiai{
		Ta có $27=u_6=u_1+5d=-3+5d\Rightarrow d=6$.
	}
\end{ex}%!Cau!%
\begin{ex}%[Thi thử lần 2, THPT Nguyễn Đức Cảnh, 2019]%[Nguyễn Anh Tuấn, 12EX7]%[1D3Y3-5]
	Một cấp số cộng $ (u_n) $ có $ 10 $ số hạng, biết $ u_1=3 $, $ u_{10}=67 $. Tính tổng các số hạng của cấp số cộng $ (u_n) $.	
	\choice
	{\True $ 350 $}
	{$ 700 $}
	{$ 175 $}
	{$ 330 $}
	\loigiai{
		Gọi $ S $ là tổng của cấp số cộng $ (u_n) $. Ta có
		$$S=\dfrac{u_1+u_{10}}{2} \cdot 10=\dfrac{3+67}{2} \cdot 10=350.$$
	}
\end{ex}%!Cau!%
\begin{ex}%[Thi thử, Chuyên Thái Nguyên-Thái Nguyên-Lần 2, 2019]%[Duong Xuan Loi, 12-EX-7-19]%[1D3Y3-3]
	Cho cấp số cộng $(u_n)$ biết $u_1=-5$, $d=2$. Số $81$ là số hạng thứ bao nhiêu?
	\choice
	{$50$}
	{$100$}
	{\True $44$}
	{$75$}
	\loigiai{
		$u_n=u_1+(n-1)d\Leftrightarrow 81=-5+(n-1)2\Leftrightarrow n=44$.
	}
\end{ex}%!Cau!%
\begin{ex}%[Thi thử L2, THPT  Ngô Quyền-Hải Phòng, 2019]%[KV Thanh, 12EX7]%[1D3Y3-3]
Cho cấp số cộng $(u_n)$ có $u_1=-5$ và công sai $d=3$. Số $100$ là số hạng thứ bao nhiêu của cấp số cộng đã cho?
\choice
{$20$}
{\True $36$}
{$35$}
{$15$}
\loigiai{
Ta có công thức số hạng của cấp số cộng $u_n=u_1+(n-1)d$.\\
$100$ là số hạng thứ $n$ khi và chỉ khi $100=-5+(n-1)\cdot 3\Leftrightarrow 3n=108\Leftrightarrow n=36$.\\
Vậy $100$ là số hạng thứ $36$.
}
\end{ex}%!Cau!%
\begin{ex}%[TT, Hội 8 trường chuyên - Khu vực đông bằng sông Hồng, 2019-L2]%[ Nguyễn Quang Dũng, dự án 12-EX-7-2019]%[1D3Y3-3]
Cho cấp số cộng $(u_n)$ có số hạng đầu $u_1=2$ và công sai $d=3$. Giá trị của $u_5$ bằng
\choice
{$11$}
{$5$}
{\True $14$}
{$15$}
\loigiai{
Ta có $u_n=u_1+(n-1)d\Rightarrow u_5=u_1+4d=2+3\times 4=14$.}
\end{ex}%!Cau!%
\begin{ex}%[Thi thử, Sở GD và ĐT - BRVT, 2019]%[Phan Văn Thành, 12EX8]%[1D3Y3-3]
	Cho cấp số cộng $(u_n)$, biết $u_2 = 3$ và $u_4 = 7$. Giá trị của $u_{2019}$ bằng
	\choice
	{$ 4040 $}
	{$ 4400 $}
	{$ 4038 $}
	{\True $ 4037 $}
	\loigiai{
		Ta có $u_4 = u_2 + 2d \Leftrightarrow d = 2$. Khi đó $u_{2019} = u_2 + 2017d = 3 + 2017 \cdot 2 = 4037$.
	}
\end{ex}%!Cau!%
\begin{ex}%[Thi thử, Sở GD và ĐT - Lào Cai, 2019]%[Lê Thanh Nin, 12EX8]%[2D1B1-1]%[Thi thử, Sở GD và ĐT - Lào Cai, 2019]%[Lê Thanh Nin, 12EX8]%[1D2Y2-1]


\begin{ex}%[Thi thử, Sở GD và ĐT - Lào Cai, 2019]%[Lê Thanh Nin, 12EX8]%[1D3Y3-2]
	Công thức nào sau đây là đúng với một cấp số cộng có số hạng đầu $u_1$, công sai $d$ và số tự nhiên $n\ge 2$. 
	\choice
	{$u_n=u_1-(n-1)d$}
	{$u_n=u_1+(n+1)d$}
	{\True $u_n=u_1+(n-1)d$}
	{$u_n=u_1+d$}
	\loigiai{ Theo công thức số hạng tổng quát của cấp số công ta có $u_n=u_1+(n-1)d$.
	}
\end{ex}%!Cau!%
\begin{ex}%[Thi thử, Sở GD và ĐT - Đà Nẵng, 2019]%[Nguyễn Anh Tuấn, 12-EX9-19]%[1D3Y3-3]
	Cho cấp số cộng $ (u_n) $ có số hạng đầu $ u_1 =2$, số hạng thứ ba $ u_3=8 $. Giá trị công sai bằng
	\choice
	{$ 5 $}
	{$ 10 $}
	{$ 4 $}
	{\True $ 3 $}
	\loigiai{
	Gọi $ d $ là công sai của cấp số cộng, suy ra $ u_3=u_1+2d \Leftrightarrow d=\dfrac{u_3-u_1}{2}=\dfrac{8-2}{2}=3$.}
\end{ex}%!Cau!%
\begin{ex}%[Thi thử L1, Chuyên Bến Tre, 2019]%[Lê Quốc Hiệp,12EX-9-2019]%[1D3Y3-3]
	Cho cấp số cộng $(u_n)$ có $u_1=-2$ và công sai $d=3$. Tìm số hạng $u_{10}$.
	\choice
	{$u_{10}=28$}
	{$u_{10}=-2\cdot 3^9$}
	{\True $u_{10}=25$}
	{$u_{10}=-29$}
	\loigiai
	{
		Ta có $u_{10}=u_1+9\cdot d=-2+9\cdot3=25$.
	}
\end{ex}%!Cau!%
\begin{ex}%[Dự án 12-EX-8-2019, Nguyễn Anh Quốc]%[1D3Y3-3]
	Cho cấp số cộng $\left(u_n\right)$ có số hạng đầu $u_1=-3$ và công sai $d=2$. Giá trị của $u_5$ bằng
	\choice
	{\True $5$}
	{$11$}
	{$-48$}
	{$-10$}
	\loigiai{Theo công thức cấp số cộng ta có
		$$u_5=u_1+4d=-3+4\cdot2=5.$$
	}
\end{ex}%!Cau!%
\begin{ex}%[Phát triển đề minh họa 2019]%[Ex 8 - 2019,Dũng Lê]%[1D3Y3-1]
	Cho một cấp số cộng $(u_n)$ có $u_1= \dfrac{1}{3}$, $u_8= 26$. Tìm công sai $d$.
	\choice
	{\True $d= \dfrac{11}{3}$}
	{$d= \dfrac{10}{3}$}
	{$d= \dfrac{3}{10}$}
	{$d= \dfrac{3}{11}$}
	\loigiai{
		Ta có $u_8= u_1+ 7d \Leftrightarrow 26= \dfrac{1}{3}+ 7d \Leftrightarrow d=\dfrac{11}{3}$.
	}
\end{ex}