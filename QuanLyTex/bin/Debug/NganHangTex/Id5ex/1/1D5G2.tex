%!Cau!%
\begin{ex}%[DTH, Sở GD và ĐT - Hà Nam, 2019]%[Đào-V- Thủy, 12EX5]%[1D5G2-5]
	Cho hàm số $f(x)=x^3+6x^2+9x+3$ có đồ thị $(C)$. Tồn tại hai tiếp tuyến của $(C)$ phân biệt và có cùng hệ số góc $k$, đồng thời đường thẳng đi qua các tiếp điểm của hai tiếp tuyến đó cắt các trục $Ox$, $Oy$ tương ứng tại $A$ và $B$ sao cho $OA=2017\cdot OB$. Hỏi có bao nhiêu giá trị của $k$ thoả mãn yêu cầu bài toán?
	\choice
	{$0$}
	{$1$}
	{\True $2$}
	{$3$}
	\loigiai
	{Gọi $M_1\left(x_1; f(x_1)\right)$; $M_2\left(x_2; f(x_2)\right)$ là hai tiếp điểm mà tại đó tiếp tuyến có cùng hệ số góc. Ta có $y'=3x^2+12x+9$.\\
		Khi đó $k=3x_1^2+12x_1+9=3x_2^2+12x_2+9\Leftrightarrow \left(x_1-x_2\right)\left(x_1+x_2+4\right)=0\Leftrightarrow x_1+x_2=-4 \quad (1)$.\\
		Hệ số góc của đường thẳng $M_1M_2$ là $k'=\pm \dfrac{OA}{OB}=\pm \dfrac{1}{2017}=\dfrac{f(x_2)-f(x_1)}{x_2-x_1}$\\
		$\Leftrightarrow \pm\dfrac{1}{2017}=\left(x_1+x_2\right)^2-x_1x_2+6(x_1+x_2)\Leftrightarrow \hoac{x_1x_2=\dfrac{2016}{2017}=P\\ x_1x_2=\dfrac{2018}{2017}=P} (2)$.\\
		Với $\heva{x_1+x_2=-4=S\\x_1x_2=\dfrac{2016}{2017}=P}$ có $S^2>4P$ nên tồn tại hai cặp $x_1$, $x_2\Rightarrow$ tồn tại $1$ giá trị $k$.\\
		Với $\heva{x_1+x_2=-4=S\\x_1x_2=\dfrac{2018}{2017}=P}$ có $S^2>4P$ nên tồn tại hai cặp $x_1$, $x_2\Rightarrow$ tồn tại $1$ giá trị $k$.\\
		Vậy có $2$ giá trị $k$ thoả mãn yêu cầu bài toán.
	}
\end{ex}%!Cau!%
\begin{ex}%[Thi thử THPTQG 2019 môn Toán lần 2 trường Nho Quan A – Ninh Bình,2019]%[Nguyễn Thành Nhân,12EX8]%[1D5G2-5]
Cho hàm số $y=x^3-2019x$ có đồ thị $(C)$. Gọi $M_1$ là điểm trên $(C)$ có hoành độ $x_1=1$. Tiếp tuyến của $(C)$ tại $M_1$ cắt $(C)$ tại điểm $M_2$ khác $M_1$, tiếp tuyến của $(C)$ tại $M_2$ cắt $(C)$ tại $M_3$ khác $M_2,\ldots,$ tiếp tuyến của $(C)$ tại  $M_{n-1}$ cắt $(C)$ tại $M_n$ khác $M_{n-1}\,(n=4,5,6,\ldots)$. Gọi $(x_n;y_n)$ là tọa độ của điểm $M_n$. Tìm $n$ để $2019x_n+y_n+2^{2013}=0$. 
\choice
	{   $n=685$  }
	{   $n=679$  }
	{\True $n=672$   }
	{  $675$  }
	
	\loigiai{  Ta có $y'=3x^2-2019$
	Gọi $M_k(x_k;y_k)$ là điểm thuộc $(C)$. Khi đó phương trình tiếp tuyến của đồ thị $(C)$ tại điểm $M_k$ là\\
	$$y-y_k=y'(x_k)(x-x_k)\Leftrightarrow y=(3x_k^2-2019)(x-x_k)+(x_k^3-2019x_k).$$
	Khi đó hoành độ của điểm $M_{k+1}$ là nghiệm của phương trình 
	\begin{align*}
	& (3x_k^2-2019)(x-x_k)+(x_k^3-2019x_k)=x^3-2019x \\
	 & \Leftrightarrow   (x-x_k)(x^2+x_k\cdot x-2x_k^2)=0 \\
	& \Leftrightarrow  \heva{&x=x_k\\&x=-2x_k.}
	\end{align*}
	Do $M_{k+1} \not\equiv  M_k$ nên $x_{k+1} =-2x_k$. Từ đó ta có một dãy $\left\{x_n\right\}$ các hoành độ tiếp điểm đươc xác định như sau
	$$x_1=1,x_2=-2,x_3=4,\ldots , x_n=(-2)^{n-1}.$$
	Do đó 
	\begin{eqnarray*} 
	& & 2019x_n+y_n+2^{2013}=0\\ 
&\Rightarrow &2019x_n +x_n^3-2019x_n+2^{2013}=0\\ 
	&\Leftrightarrow & (-2)^{3(n-1)}+2^{2013}=0 \\
	&\Leftrightarrow & n=672. 
	\end{eqnarray*} 
	 }	
\end{ex}%!Cau!%
\begin{ex}%[Thi thử, Trần Đại Nghĩa - Đắc Lắk, 2019]%[Trần Nhân Kiệt, 12EX8-2019]%[1D5G2-3]
	Cho hàm số $y=-x^3+mx^2-x-4m$ có đồ thị $(C_m)$ và $A$ là điểm cố định có hoành độ âm của $(C_m)$. Giá trị của $m$ để tiếp tuyến tại $A$ của $(C_m)$ vuông góc với đường phân giác của góc phần tư thứ nhất là
	\choice
	{$m=2$}
	{\True $m=-3$}
	{$m=-\dfrac{7}{2}$}
	{$m=-6$}
\loigiai{
Gọi $A(x;y)$ là điểm cố định của $(C_m)$ với $x<0$. Khi đó 
\begin{eqnarray*}
&&y=-x^3+mx^2-x-4m,\,\forall m\in \Bbb{R}\\
&\Leftrightarrow& m(x^2-4)-x^3-x-y=0,\,\forall m\in \Bbb{R}\\
&\Leftrightarrow& \heva{&x^2-4=9\\&-x^3-x-y=0}\\
&\Leftrightarrow& \heva{&\hoac{&x=2\quad (\text{loại})\\&x=-2}\\&y=-x^3-x}\\
&\Leftrightarrow& \heva{&x=-2\\&y=10}\Rightarrow A(-2;10).
\end{eqnarray*}
Ta có $y'=-3x^2+2mx-1\Rightarrow y'(-2)=-4m-13$.\\
Tiếp tuyến tại $A$ của $(C_m)$ có hệ số góc là $y'(-2)=-4m-13$.\\
Vì tiếp tuyến tại $A$ của $(C_m)$ vuông góc với đường phân giác của góc phần tư thứ nhất $y=x$ nên
$$y'(-2)\cdot 1=-1\Leftrightarrow -4m-13=-1\Leftrightarrow m=-3.$$

}
\end{ex}%!Cau!%
\begin{ex}%[Thi thử, Trần Đại Nghĩa - Đắc Lắk, 2019]%[Trần Nhân Kiệt, 12EX8-2019]%[1D5G2-5]
	Cho hàm số $y=\dfrac{x-1}{2\left(x+1\right)}$ có đồ thị là $(C)$. Gọi điểm $M\left(x_0;y_0\right)$ với $x_0>-1$ là điểm thuộc $(C)$, biết tiếp tuyến của $(C)$ tại điểm $M$ cắt trục hoành, trục tung lần lượt tại hai điểm phân biệt $A$, $B$ và tam giác $OAB$ có trọng tâm $G$ nằm trên đường thẳng $d\colon 4x+y=0$. Hỏi giá trị của $x_0+2y_0$ bằng bao nhiêu?
	\choice
	{$\dfrac{5}{2}$}
	{$\dfrac{7}{2}$}
	{$-\dfrac{5}{2}$}
	{\True $-\dfrac{7}{2}$}
	\loigiai{
		Ta có $y'=\dfrac{1}{(x+1)^2}$, $x\neq -1$.\\
		Gọi $\Delta$ là tiếp tuyến của $(C)$ tại $M(x_0;y_0)$, $x_0>-1$.\\
		Suy ra $\Delta\colon y=y'(x_0)(x-x_0+y_0)\Rightarrow \Delta\colon y=\dfrac{1}{(x_0+1)^2}(x-x_0)+\dfrac{x_0-1}{2(x_0+1)}$.\\
		Ta có
		\begin{itemize}
			\item  $A=\Delta\cap Ox\Rightarrow y=0\Rightarrow \dfrac{1}{(x_0+1)^2}(x-x_0)+\dfrac{x_0-1}{2(x_0+1)}=0\Leftrightarrow x=\dfrac{-x_0^2+2x_0+1}{2}$\\
			$\Rightarrow A\left(\dfrac{-x_0^2+2x_0+1}{2};0\right)$.
			\item $B=\Delta\cap Oy\Rightarrow x=0\Rightarrow y=-\dfrac{x_0}{(x_0+1)^2}+\dfrac{x_0-1}{2(x_0+1)}=\dfrac{x_0^2-2x_0-1}{2(x_0+1)^2}$\\
			$\Rightarrow B\left(0;\dfrac{x_0^2-2x_0-1}{2(x_0+1)^2}\right)$.
		\end{itemize}
		Tọa độ trọng tâm $G$ của tam giác $OAB$ là $G\left(\dfrac{-x_0^2+2x_0+1}{6};\dfrac{x_0^2-2x_0-1}{6(x_0+1)^2}\right)$. Khi đó 
		\begin{eqnarray*}
			&&G\in d\\
			&\Leftrightarrow& 4\cdot \dfrac{-x_0^2+2x_0+1}{6}+\dfrac{x_0^2-2x_0-1}{6(x_0+1)^2}=0\\
			&\Leftrightarrow& \hoac{&x_0^2-2x_0-1=0\quad (\text{loại vì } A\equiv B\equiv O)\\&\dfrac{1}{(x_0+1)^2}=4}\\
			&\Leftrightarrow& x_0+1=\pm \dfrac{1}{2}\\
			&\Leftrightarrow& \hoac{&x_0=-\dfrac{1}{2}>-1\\&x_0=-\dfrac{3}{2}<-1.}
		\end{eqnarray*}
		Suy ra $x_0=-\dfrac{1}{2}$, $y_0=-\dfrac{3}{2}$.\\
		Vậy $x_0+2y_0=-\dfrac{1}{2}+2\cdot \left(-\dfrac{3}{2}\right)=-\dfrac{7}{2}$.
	}
\end{ex}%!Cau!%
\begin{ex}%[Thi thử THPT-QG lần 3 THCS-THPT Lương-Thế-Vinh-Hà-Nội]%[Ex 9 - 2019, Dũng Lê]%[1D5G2-5]
	Cho hàm số $y=f(x)$ có đồ thị $(C)$, biết tiếp tuyến của đồ thị $(C)$ tại điểm có hoành độ $x=0$ là đường thẳng $y=3x-3$. Giá trị của $\lim\limits_{x\to 0}\dfrac{3x}{f(3x)-5f(4x)+4f(7x)}$ là
	\choice
	{$\dfrac{1}{10}$}
	{$\dfrac{3}{31}$}
	{$\dfrac{3}{25}$}
	{\True $\dfrac{1}{11}$}
	\loigiai{
		Ta có
		\begin{eqnarray*}
			\lim\limits_{x\to 0}\dfrac{3x}{f(3x)-5f(4x)+4f(7x)}&=&\lim\limits_{x\to 0}\dfrac{3x}{(f(3x)-f(0))-5(f(4x)-f(0))+4(f(7x)-f(0))}\\
			&=&\lim\limits_{x\to 0}\dfrac{3}{3\frac{f(3x)-f(0)}{3x-0}-20\frac{f(4x)-f(0)}{4x-0}+28\frac{f(7x)-f(0)}{7x-0}}\\
			&=&\dfrac{3}{3\lim\limits_{3x\to 0}\frac{f(3x)-f(0)}{3x-0}-20\lim\limits_{4x\to 0}\frac{f(4x)-f(0)}{4x-0}+28\lim\limits_{7x\to 0}\frac{f(7x)-f(0)}{7x-0}}\\
			&=&\dfrac{3}{3f'(0)-20f'(0)+28'(0)}\\
			&=&\dfrac{3}{11f'(0)}.
		\end{eqnarray*}
		Tiếp tuyến của đồ thị hàm số $y=f(x)$ tại điểm có hoành độ $x=0$ là $y=3x-3$, suy ra $f'(0)=3$. \\
		Do đó $\lim\limits_{x\to 0}\dfrac{3x}{f(3x)-5f(4x)+4f(7x)}=\dfrac{3}{11f'(0)}=\dfrac{1}{11}$.
	}
\end{ex}