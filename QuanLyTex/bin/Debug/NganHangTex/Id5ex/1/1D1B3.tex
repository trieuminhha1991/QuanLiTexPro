%!Cau!%
\begin{ex}%[Đề thi thử THPTQG lần 2 THPT Thoại Ngọc Hầu, An Giang, năm 2019]%[Nguyễn Thành Khang, dự án 2019-Ex-7]%[1D1B3-3]
	Cho phương trình $(2\sin x+1)(\sqrt{3}\cos x+2\sin x)=2\sin^2x+3\sin x+1$. Tính tổng tất cả các nghiệm thuộc đoạn $[0;2\pi]$ của phương trình đã cho.
	\choice
	{\True $\dfrac{7\pi}{2}$}
	{$2\pi$}
	{$\dfrac{16\pi}{3}$}
	{$\pi$}
	\loigiai{
		Phương trình đã cho tương đương với
		\allowdisplaybreaks{
			\begin{eqnarray*}
				&& (2\sin x+1)(\sqrt{3}\cos x+2\sin x)=(2\sin x+1)(\sin x+1)\\
				&\Leftrightarrow& (2\sin x+1)(\sqrt{3}\cos x+\sin x-1)=0 \\
				&\Leftrightarrow& (2\sin x+1)\left(\dfrac{\sqrt{3}}{2}\cos x+\dfrac{1}{2}\sin x-\dfrac{1}{2}\right)=0 \\
				&\Leftrightarrow& (2\sin x+1)\left(\cos\left(x-\dfrac{\pi}{6}\right)-\dfrac{1}{2}\right)=0\\
				&\Leftrightarrow& \hoac{&\sin x=-\dfrac{1}{2} \\ &\cos\left(x-\dfrac{\pi}{6}\right)=\dfrac{1}{2}} \\
				&\Leftrightarrow& \hoac{&x=-\dfrac{\pi}{6}+k2\pi \\ &\pi-x=-\dfrac{\pi}{6}+k2\pi \\ &x-\dfrac{\pi}{6}=\pm\dfrac{\pi}{3}+k2\pi},k\in\mathbb{Z}\\
				&\Leftrightarrow& \hoac{&x=-\dfrac{\pi}{6}+k2\pi \\ &x=\dfrac{7\pi}{6}+k2\pi \\ &x=\dfrac{\pi}{2}+k2\pi},k\in\mathbb{Z}
			\end{eqnarray*}
		}
		Mà $x\in[0;2\pi]$ nên $x\in\left\{\dfrac{11\pi}{6};\dfrac{7\pi}{6};\dfrac{\pi}{2}\right\}$, suy ra tổng các nghiệm thỏa mãn là $\dfrac{7\pi}{2}$.
	}
\end{ex}%!Cau!%
\begin{ex}%[Thi thử L2, THPT Nguyễn Trung Thiên - Hà Tĩnh, 2019]%[Nguyễn Thành Nhân,12EX7]%[1D1B3-1]
Số nghiệm thuộc khoảng $\left(0;2019\right)$ của phương trình $ \sin ^4 \dfrac{x}{2} +\cos ^4 \dfrac{x}{2} =1-2 \sin x$ là    
	\choice
	{ \True $642$  }
	{$643$ }
	{$641$ }
	{$644$}
	\loigiai{ Phương trình đã cho tương đương
	\begin{eqnarray*}
& & 1-\dfrac{1}{2} \sin ^2 x =1-2 \sin x\\
&\Leftrightarrow & \sin ^2 x- 4\sin x =0\\
&\Leftrightarrow & \hoac{& \sin x =0 \\& \sin x =4\, \text{(vô nghiệm)}.}\\
&\Leftrightarrow & x= k \pi \left( k \in \mathbb{Z}\right).
\end{eqnarray*}
Vì $ x\in \left(0;2019\right)$ nên $0<k \pi < 2019 \Leftrightarrow 0<k< \dfrac{2019}{\pi}$. Vì $k \in \mathbb{Z}$ nên có $642$ giá trị $k$ thỏa mãn.
	 }
\end{ex}%!Cau!%
\begin{ex}%[Thi thử,Quảng Xương 1-Thanh Hóa-L3, 2019]%[Lê Quốc Hiệp, 12EX8-2019]%[1D1B3-1]
	Nghiệm dương bé nhất của phương trình $2\sin^2x+5\sin x-3=0$ là
	\choice
	{\True $x=\dfrac{\pi}{6}$}
	{$x=\dfrac{3\pi}{2}$}
	{$x=\dfrac{5\pi}{6}$}
	{$x=\dfrac{\pi}{2}$}
	\loigiai
	{
		Ta có $2\sin^2x+5\sin x-3=0\Leftrightarrow\hoac{&\sin x=\dfrac{1}{2}\\&\sin x=-3\text{ (vô nghiệm)}}\Leftrightarrow \hoac{&x=\dfrac{\pi}{6}+k2\pi\\&x=\dfrac{5\pi}{6}+k2\pi},~(k\in\mathbb{Z})$.\\
		Vậy nghiệm dương bé nhất là $\dfrac{\pi}{6}$.
	}
\end{ex}%!Cau!%
\begin{ex}%[Thi Thủ SGD Bắc Ninh]%[Phan Anh - EX9]%[1D1B3-3]
	Số giá trị nguyên của tham số $m$ để phương trình $\sin x+(m-1)\cos x=2m-1$ có nghiệm là
	\choice
	{$0$}
	{$3$}
	{\True $2$}
	{$1$}
	\loigiai{Phương trình đã cho có nghiệm khi và chỉ khi
		$$1^2+(m-1)^2\ge(2m-1)^2\Leftrightarrow3m^2-2m-1\le0\Leftrightarrow-\dfrac{1}{3}\le m\le1.$$
		Vậy có $2$ giá trị nguyên của $m$ làm cho phương trình có nghiệm.}
\end{ex}%!Cau!%
\begin{ex}%[Thi thử, Toán học tuổi trẻ - Đề 6, 2019]%[Phan Văn Thành, 12EX9]%[1D1B3-4]
	Từ phương trình $(1 + \sqrt{5})(\sin x - \cos x) + \sin 2x - 1 - \sqrt{5} = 0$ ta tìm được $\sin \left(x - \dfrac{\pi}{4}\right)$ có giá trị bằng
	\choice
	{$ \dfrac{\sqrt{3}}{2} $}
	{$ -\dfrac{\sqrt{2}}{2} $}
	{\True $ \dfrac{\sqrt{2}}{2} $}
	{$ -\dfrac{\sqrt{3}}{2} $}
	\loigiai{
	Đặt $t = \sin x - \cos x$, điều kiện $-\sqrt{2} \leq t \leq \sqrt{2}$.\\
	Khi đó phương trình đã cho trở thành
	$$(1 + \sqrt{5})t + 1 - t^2 - 1 - \sqrt{5} = 0 \Leftrightarrow -t^2 + (1 + \sqrt{5})t - \sqrt{5} = 0 \Leftrightarrow \hoac{&t = \sqrt{5}\\&t = 1.}$$
	So với điều kiện thì $t = 1$ thỏa yêu cầu bài toán hay
	$$\sin x - \cos x = 1 \Leftrightarrow \dfrac{1}{\sqrt{2}}\sin x - \dfrac{1}{\sqrt{2}}\cos x = \dfrac{\sqrt{2}}{2} \Leftrightarrow \sin \left(x - \dfrac{\pi}{4}\right) = \dfrac{\sqrt{2}}{2}.$$
	}
\end{ex}