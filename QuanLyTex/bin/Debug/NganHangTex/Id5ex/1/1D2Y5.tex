%!Cau!%
\begin{ex}%[KSCL, Sở GD và ĐT - Thanh Hóa, 2018]%[Bùi Ngọc Diệp, 12EX-5][1D2Y5-1]
	Cho $A$ và $B$ là hai biến cố xung khắc. Mệnh đề nào sau đây đúng?
	\choice
	{\True Hai biến cố $A$ và $B$ không đồng thời xảy ra}
	{Hai biến cố $A$ và $B$ đồng thời xảy ra}
	{$\mathrm{P}(A) + \mathrm{P}(B) = 1$}
	{$\mathrm{P}(A) + \mathrm{P}(B) < 1$}
	\loigiai{
		Mệnh đề đúng là "Hai biến cố $A$ và $B$ không đồng thời xảy ra".}
\end{ex}%!Cau!%
\begin{ex}%[De tap huan, So GD&DT Dien Bien, 2019]%[Ngoc Diep, dự án EX5]%[1D2Y5-1]
	Cho $A,B$ là hai biến cố xung khắc. Đẳng thức nào sau đây là đúng?
	\choice
	{\True $\mathrm{P}(A\cup B) = \mathrm{P}(A)+\mathrm{P}(B)$}
	{$\mathrm{P}(A \cup B) = \mathrm{P}(A)\cdot P(B)$}
	{$\mathrm{P}(A\cup B) = \mathrm{P}(A)-\mathrm{P}(B)$}
	{$\mathrm{P}(A\cap B) = \mathrm{P}(A)+\mathrm{P}(B)$}
	\loigiai{
		Với $A,B$ là hai biến cố xung khắc thì $\mathrm{P}(A\cup B) = \mathrm{P}(A)+\mathrm{P}(B)$.
	}
\end{ex}%!Cau!%
\begin{ex}%[Đề tập huấn tỉnh Lai Châu,2019]%[Nguyễn Trung Kiên, dự án 12-EX-5-2019]%[1D2Y5-1]
	Cho $A$ và $\overline{A}$ là hai biến cố đối nhau. Khẳng định nào sau đây là khẳng định \textbf{đúng}?
	\choice
	{$\mathrm{P}(A)=1+\mathrm{P}(\overline{A})$}
	{$\mathrm{P}(A)=\mathrm{P}(\overline{A})$}
	{\True $\mathrm{P}(A)=1-\mathrm{P}(\overline{A})$}
	{$\mathrm{P}(A)+\mathrm{P}(\overline{A})=0$}
	\loigiai
	{Áp dụng tính chất xác suất ta có $\mathrm{P}(A)=1-\mathrm{P}(\overline{A})$.}
\end{ex}%!Cau!%
\begin{ex}%[Đề thi thử THPTQG lần 2 THPT Thoại Ngọc Hầu, An Giang, năm 2019]%[Nguyễn Thành Khang, dự án 2019-Ex-7]%[1D2Y5-2]
	Năm đoạn thẳng có độ dài $1$ cm, $3$ cm, $5$ cm, $7$ cm, $9$ cm. Lấy ngẫu nhiên ba đoạn thẳng trong năm đoạn thẳng trên. Xác suất để ba đoạn thẳng lấy ra có thể tạo thành ba cạnh của một tam giác là
	\choice
	{$\dfrac{2}{5}$}
	{$\dfrac{7}{10}$}
	{$\dfrac{3}{5}$}
	{\True $\dfrac{3}{10}$}
	\loigiai{
		Số phần tử của không gian mẫu là $n(\Omega)=\mathrm{C}_5^3=10$.\\
		Gọi $A$ là biến cố ``ba đoạn thẳng lấy ra có thể tạo thành ba cạnh của một tam giác''. Khi đó các trường hợp thuận lợi cho biến cố $A$ thì độ dài các đoạn thẳng lấy ra chỉ có thể là $(3;5;7)$, $(3;7;9)$ và $(5;7;9)$ nên $n(A)=3$.\\
		Vậy xác suất cần tìm là $\mathrm{P}(A)=\dfrac{n(A)}{n(\Omega)}=\dfrac{3}{10}$.
	}
\end{ex}