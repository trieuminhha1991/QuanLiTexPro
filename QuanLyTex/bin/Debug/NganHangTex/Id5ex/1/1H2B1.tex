%!Cau!%
\begin{ex}%[Đề tập huấn, Sở GD - ĐT tỉnh Quảng Bình, 2019]%[Nguyễn Tiến, dự án 12EX5]%[1H2B1-1]
	Cho hai đường thẳng phân biệt $a$; $b$ và mặt phẳng $\left(\alpha\right)$. Hãy chọn mệnh đề \textbf{đúng} trong các mệnh đề sau
	\choice
	{Nếu $a \parallel\left(\alpha\right)$ và $b \parallel\left(\alpha\right)$ thì $a\parallel b$}
	{\True Nếu $a \parallel\left(\alpha\right)$ và $b \perp\left(\alpha\right)$ thì $a\perp b$}
	{Nếu $a \parallel\left(\alpha\right)$ và $b \perp a$ thì $b\perp \left(\alpha\right)$}
	{Nếu $a \parallel\left(\alpha\right)$ và $b \perp a$ thì $b\parallel \left(\alpha\right)$}
	\loigiai{
		\item[-] Với $\heva{& a \parallel\left(\alpha\right)\\& b \parallel\left(\alpha\right)}$ thì $a$ chưa chắc song song với $b$, vì khi $a$, $b$ cùng nằm trong một mặt phẳng thì chúng có thể cắt nhau $\Rightarrow$ đáp án sai.
		\item[-] Với $\heva{& a \parallel\left(\alpha\right)\\& b \perp a}$ thì $b$ chưa chắc vuông góc với $\left(\alpha\right)$, vì khi $b$ cùng nằm trong một mặt phẳng với $a$ thì $b\parallel\left(\alpha\right)$ $\Rightarrow$ đáp án sai.
		\item[-] Với $\heva{& a \parallel\left(\alpha\right)\\& b \perp a}$ thì $b$ chưa chắc song song với $\left(\alpha\right)$, vì $b$ có thể nằm trong mặt phẳng $\left(\alpha\right)$\\
		$\Rightarrow$ đáp án sai.
		\item[-] Với $\heva{& a \parallel\left(\alpha\right)\\& b \perp\left(\alpha\right)} \Rightarrow a\perp b$ $\Rightarrow$ đáp án đúng.
	}
\end{ex}%!Cau!%
\begin{ex}%[Đề tập huấn Sở Ninh Bình, 2019]%[Nguyễn Văn Hải, dự án(12EX-5-2019)]%[1H2B1-4]
Cho hình chóp $S.ABCD$ có đáy $ABCD$ là hình bình hành. Gọi $M$, $N$, $P$ lần lượt là trung điểm của $BC$, $CD$, $SA$. Mặt phẳng $\left(MNP\right)$ cắt hình chóp theo thiết diện  là hình
\choice
{Tam giác}
{Lục giác}
{\True Ngũ giác}
{Tứ giác}
\loigiai{
\immini{
Gọi $I$, $J$ lần lượt là giao của đường thẳng $MN$ và $AB$, $AD$.\\
Gọi $F=SB\cap PI$; $E=SD\cap PJ$.\\
Khi đó thiết diện của hình chóp cắt bởi mặt phẳng $\left(MNP\right)$ là ngũ giác $MNEPF$.}
{
\begin{tikzpicture}[scale=0.8, font=\footnotesize, line join=round, line cap=round, >=stealth,yscale=1, xscale=1.5]
% định nghĩa các điểm
\tkzDefPoints{0/0/A,-1/-1/B,2/-1/C,3/0/D,0/3/S}
\tkzDefMidPoint(B,C) \tkzGetPoint{M}
\tkzDefMidPoint(D,C) \tkzGetPoint{N}
\tkzDefMidPoint(S,A) \tkzGetPoint{P}
\tkzInterLL(M,N)(A,B) \tkzGetPoint{I}
\tkzInterLL(M,N)(A,D) \tkzGetPoint{J}
\tkzInterLL(S,B)(P,I) \tkzGetPoint{F}
\tkzInterLL(S,D)(P,J) \tkzGetPoint{E}
% vẽ các đường
\tkzDrawSegments[dashed](S,A A,I A,J P,F P,E F,B B,M E,D D,N M,N)
\fill[fill=green,opacity=0.2](P)--(E)--(N)--(M)--(F)--cycle;% tô màu
\tkzDrawSegments(S,E S,F S,C E,N E,J F,M F,I M,I M,C N,J N,C)
% Viết tên các điểm
\tkzDrawPoints(S,A,B,C,D,M,N,P,E,F,I,J)
\tkzLabelPoints[below](A,B,C,D,M,N)
\tkzLabelPoints[left](S,F,I)
\tkzLabelPoints[above right](P,E,J)
\end{tikzpicture}
}}
\end{ex}%!Cau!%
\begin{ex}%[Đề Tập Huấn -4, Sở GD và ĐT - Hải Phòng, 2019]%[Trần Xuân Thiện, 12EX5]%[1H2B1-1]
	Cho hình lăng trụ $ABC.A'B'C'$. Gọi $M,M'$ lần lượt là trung điểm của $B$C và $B'C'$ ; $G,G'$ lần lượt là trọng tâm tam giác $ABC$ và $A'B'C'$. Bốn điểm nào sau đây đồng phẳng?
	\choice
	{$ A,G,G',C' $}
	{$ A,G,M',B' $}
	{$ A',G',M,C $}
	{\True $ A,G',M',G $}
	\loigiai{
		\immini[0.05]{
			Do $MM'$ là đường trung bình trong hình bình hành $BB'C'C$ nên $\heva{&MM' = BB' = AA'\\&MM' \parallel BB' \parallel AA'.}$\\ 
			Do đó $AA'M'M$ là hình bình hành hay $4$ điểm $A,G',M',G$ đồng phẳng.
		             }{
\begin{tikzpicture}[scale=0.6, font=\footnotesize,line join=round, line cap=round,>=stealth]
\tikzset{label style/.style={font=\footnotesize}}
\tkzDefPoints{0/0/A,7/0/C,4/-3/B}
\coordinate (A') at ($(A)+(2,6)$);
\tkzDefPointsBy[translation = from A to A'](B,C){B'}{C'}
\coordinate (M) at ($(B)!0.5!(C)$);
\coordinate (M') at ($(B')!0.5!(C')$);
\coordinate (N) at ($(B)!0.5!(A)$);
\coordinate (N') at ($(B')!0.5!(A')$);
\tkzInterLL(A,M)(C,N)
\tkzGetPoint{G}
\tkzInterLL(A',M')(C',N')
\tkzGetPoint{G'}
\tkzDrawPolygon(A,B,C,C',B',A')
\tkzDrawSegments(A',C' B',B M,M' A',M' C',N')
\tkzDrawSegments[dashed](A,C A,M C,N G,G')
\tkzDrawPoints[fill=black](A,C,B,A',B',C',M,M',N,N',G,G')
\tkzLabelPoints[above](B',M',G')
\tkzLabelPoints[below](B,M,G)
\tkzLabelPoints[left](A',A)
\tkzLabelPoints[right](C',C)
\end{tikzpicture}
		}
	}
\end{ex}