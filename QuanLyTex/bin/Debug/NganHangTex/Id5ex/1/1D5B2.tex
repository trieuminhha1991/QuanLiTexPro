%!Cau!%
\begin{ex}%[HK2, THPT Nguyễn Huệ, Vĩnh Phúc, 2019]%[Thịnh Trần, dự án(12EX-5-2019)]%[1D5B2-2]
	Cho hàm số $y =\dfrac{x^2 + x}{x - 2}$ có đồ thị $(C)$. Phương trình tiếp tuyến tại điểm $A(1; -2)$ của $(C)$ là
	\choice
	{$y = -3x + 5$}
	{$y = -5x + 7$}
	{\True $y = -5x + 3$}
	{$y = -4x + 6$}
	\loigiai{
		Ta có $y'=\dfrac{(2x+1)(x-2)-(x^2+x)}{(x-2)^2}=\dfrac{x^2-4x-2}{(x-2)^2}$.\\
		Phương trình tiếp tuyến của $(C)$ tại $A(1;-2)$ là $$y=y'(1)\cdot (x-1)-2\Leftrightarrow y=-5(x-1)-2\Leftrightarrow y=-5x+3.$$	
	}
\end{ex}%!Cau!%
\begin{ex}%[THPT Đức Thọ - Hà Tĩnh - Lần 1 - 2019]%[Phan Anh - EX7]%[1D5B2-2]
Cho đồ thị $(H)\colon y=\dfrac{2x-4}{x-3}$. Lập phương trình tiếp tuyến của đồ thị $(H)$ tại giao điểm của $(H)$ và $Ox$.
	\choice
	{\True $y=-2x+4$}
	{$y=-2x-4$}
	{$y=2x$}
	{$y=2x-4$}
	\loigiai{Phương trình hoành độ giao điểm của $(H)$ với $Ox$ là $\dfrac{2x-4}{x-3}=0\Leftrightarrow x=2$.\\
	Vậy $(H)$ cắt $Ox$ tại $A(2;0)$.\\
Ta có $y'=\dfrac{-2}{(x-3)^2}$, suy ra phương trình tiếp tuyến của $(H)$ tại $A$ là
$$y=y'(2)(x-2)\Leftrightarrow y=-2(x-2)=-2x+4.$$}
\end{ex}%!Cau!%
\begin{ex}%[Thi thử, Lý Thái Tổ - Bắc Ninh, 2019]%[Nguyện Ngô, 12EX7]%[1D5B2-2]
Phương trình tiếp tuyến của đồ thị hàm số $y=\dfrac{2x-1}{x+1}$ tại điểm có hoành độ bằng $-2$ là
\choice
{$y=-3x+1$}
{$y=3x+1$}
{\True $y=3x+11$}
{$y=-3x+11$}
\loigiai{
\begin{itemize}
\item Ta có $y'(-2)=3$, $y(-2)=5$.
\item Vậy phương trình tiếp tuyến của đồ thị hàm số tại điểm có hoành độ bằng $-2$ là
\[y=3(x+2)+5\Leftrightarrow y=3x+11.\]
\end{itemize}
}
\end{ex}%!Cau!%
\begin{ex}%[Thử sức trước kì thi đề số 4, THTT, 2019]%[Trần Hòa, 12EX-7-2019]%[1D5B2-2]
	Phương trình tiếp tuyến của đồ thị hàm số $y=x^3-3x^2+1$ tại điểm $A(3;1)$ là
	\choice
	{$y=-9x-26$}
	{\True $y=9x-26$}
	{$y=-9x-3$}
	{$y=9x+2$}
	\loigiai{
		Ta có $y'=3x^2-6x\Rightarrow y'(3)=9$.\\
		Phương trình tiếp tuyến tại $A(3;1)$ là $y=9(x-3)+1\Leftrightarrow y=9x-26$.
	}
\end{ex}%!Cau!%
\begin{ex}%[Thử sức trước kì thi đề số 4, THTT, 2019]%[Trần Hòa, 12EX-7-2019]%[1D5B2-3]
	Tìm điểm $M$ có hoành độ âm trên đồ thị $(C)\colon y=\dfrac{1}{3}x^3-x+\dfrac{2}{3}$ sao cho tiếp tuyến tại $M$ vuông góc với đường thẳng $y=-\dfrac{1}{3}x+\dfrac{2}{3}$.
	\choice
	{$M(-2;-4)$}
	{$M\left(-1;\dfrac{4}{3}\right)$}
	{$M\left(2;\dfrac{4}{3}\right)$}
	{\True $M(-2;0)$}
	\loigiai{
	Ta có $y'=x^2-1$.\\
	Gọi $M(x_0;y_0)$ là hoành độ tiếp điểm. Do tiếp tuyến tại $M$ vuông góc với đường thẳng $y=-\dfrac{1}{3}x+\dfrac{2}{3}$ nên ta có $y'(x_0)=3\Leftrightarrow x_0^2-1=3\Leftrightarrow \hoac{&x_0=-2\\&x_0=2.}$\\
	Do $x_0<0$ nên $x_0=-2\Rightarrow M(-2;0).$	
	}
\end{ex}%!Cau!%
\begin{ex}%[Thi thử, Sở GD và ĐT - Quảng Nam, 2019]%[Nguyện Ngô, 12EX8]%[1D5B2-2]
Phương trình tiếp tuyến của đồ thị hàm số $y=x^3+3x-1$ tại điểm có hoành độ $x=1$ là
\choice
{\True $y=6x-3$}
{$y=6x+3$}
{$y=6x-1$}
{$y=6x+1$}
\loigiai{
\begin{itemize}
\item Ta có $y'=3x^2+3$, $y'(1)=6$, $y(1)=3$.
\item Suy ra phương trình tiếp tuyến là $y=6(x-1)+3=6x-3$.
\end{itemize}
}
\end{ex}%!Cau!%
\begin{ex}%[Thi thử, Sở GD và ĐT - Hưng Yên-Lần 1, 2019]%[Duong Xuan Loi, 12-EX-8]%[1D5B2-3] 
	Cho hàm số $y=x^3-3x^2+2$ có đồ thị $(C)$. Tìm số tiếp tuyến của đồ thị $(C)$ song song với đường thẳng $y=9x+7$.
	\choice
	{$3$}
	{$0$}
	{\True $1$}
	{$2$}
	\loigiai{
		Ta có $y'=3x^2-6x$.\\
		Vì tiếp tuyến song song với đường thẳng $y=9x+7$ nên hệ số góc của nó là $k=9$. \\
		Nên $y'=k \Leftrightarrow 3x^2-6x=9 \Leftrightarrow \hoac{&x=3\\&x=-1.}$ \\
		Với $x=3$ ta có $y(3)=2$ suy ra phương trình tiếp tuyến là $y=9(x-3)+2=9x-25$ (thỏa mãn).\\
		Với $x=-1$ ta có $y(-1)=-2$ suy ra phương trình tiếp tuyến là $y=9(x+1)-2=9x+7$ (loại).\\
		Vậy có một tiếp tuyến thỏa mãn.
	}
\end{ex}%!Cau!%
\begin{ex}%[Thi thử, Sở GD và ĐT - Hưng Yên-Lần 1, 2019]%[Duong Xuan Loi, 12-EX-8]%[1D5B2-1] 
	Cho $\left(\dfrac{3-2x}{\sqrt{4x-1}}\right)’=\dfrac{ax-b}{(4x-1)\sqrt{4x-1}}$, $\forall x>\dfrac{1}{4}$. Tính $\dfrac{b}{a}$.
	\choice
	{$4$}
	{$1$}
	{\True $-1$}
	{$-4$}
	\loigiai{
		Ta có
		\begin{eqnarray*}
			\left(\dfrac{3-2x}{\sqrt{4x-1}}\right)’&=&\dfrac{(-2)\sqrt{4x-1}-\dfrac{2}{\sqrt{4x-1}}(3-2x)}{4x-1}\\
			&=&\dfrac{(-2)(4x-1)-2(3-2x)}{(4x-1)\sqrt{4x-1}}=\dfrac{-4x-4}{(4x-1)\sqrt{4x-1}}.
		\end{eqnarray*}
		
		Vậy ta có $a=-4$, $b=4 \Rightarrow \dfrac{b}{a}=-1$.
	}
\end{ex}%!Cau!%
\begin{ex}%[Thi thử, Kinh Môn - Hải Dương, 2019]%[Lê Vũ Hải, 12EX8]%[1D5B2-3]
	Phương trình tiếp tuyến của đồ thị hàm số $y=\dfrac{x+1}{x-1}$ song song với đường thẳng $\Delta \colon y=-2x-1$ là 
	\choice
	{\True $ y=-2x+7 $}
	{$ y=-2x+4 $}
	{$ y=-2x-1 $}
	{$ y=-2x+3 $}
	\loigiai{
		Ta có $y' = \dfrac{-2}{(x-1)^2}$.\\
		Gọi $M(x_{0};y_{0})$ là tiếp điểm, do tiếp tuyến tại $M$ song song với $\Delta \colon y=-2x-1$ \\
		$\Rightarrow \dfrac{-2}{(x_{0}-1)^2} = -2 \Leftrightarrow (x_{0}-1)^{2}=1 \Leftrightarrow \hoac{& x_{0}=2 \\ & x_{0} = 0.}$\\
		Với $x_{0}=2$ suy ra phương trình tiếp tuyến $y=-2(x-2)+3\Leftrightarrow y=-2x+7$ (nhận).\\
		Với $x_{0}=0$ suy ra phương trình tiếp tuyến $y=-2(x-0)-1 \Leftrightarrow y=-2x-1$ (loại do trùng với đường thẳng $\Delta$).
	}
\end{ex}%!Cau!%
\begin{ex}%[Chuyên ĐHSPHN - 19]%[Phan Anh - EX8]%[1D5B2-6]
Trong một chuyển động thẳng, chất điểm chuyển động xác định bởi phương trình $s(t)=t^3-3t^2+3t+10$, trong đó thời gian $t$ tính bằng giây và quãng đường $s$ tính bằng mét. Gia tốc của chất điểm tại thời điểm chất điểm dừng lại là
\choice
{$-6$ m/s$^2$}
{\True $0$ m/s$^2$}
{$12$ m/s$^2$}
{$10$ m/s$^2$}
\loigiai{
Ta có $v(t)=s'(t)=3t^2-6t+3$, $v(t)=0\Leftrightarrow 3t^2-6t+3=0\Leftrightarrow t=1$.\\
Ta có $a(t)=v'(t)=6t-6\Rightarrow a(1)=0$.}
\end{ex}%!Cau!%
\begin{ex} %[Thi Thử L1, Trường THPT Phụ Dực- Thái Bình, 2019 ]%[Nguyễn Thế Anh, 12EX8-2019]%[1D5B2-3] 
		Tìm điểm $M$ trên đồ thị hàm số $y=\dfrac{x-3}{x+1}$ có đồ thị $(\mathcal{C})$,  biết tiếp tuyến của đồ thị $(\mathcal{C})$ tại $M$ song song với đường thẳng $y=4x-3$. 
		\choice
		{Không tồn tại $M$}
		{$M\left(0;-3\right)$}
		{$M\left(0;-3\right)$ hoặc $M\left(-2;5\right)$}
		{\True $M\left(-2;5\right)$}
		\loigiai{
		Ta có $y'=\dfrac{4}{{(x+1)}^2}$.\\
		Do $M\in(\mathcal{C})\colon y=\dfrac{x-3}{x+1}$ nên $ M\left(m;\dfrac{m-3}{m+1}\right)$.\\
		Vì tiếp tuyến của đồ thị $(\mathcal{C})$ tại $M$ song song với đường thẳng $y=4x-3$ nên ta có 
		\begin{eqnarray*}
		&&y'(m)=4\\&\Leftrightarrow&\dfrac{4}{(m+1)^2}=4\\&\Leftrightarrow&(m+1)^2=1\\&\Leftrightarrow&\hoac{&m+1=1\\ &m+1=-1}\\&\Leftrightarrow&\hoac{&m=0\\ &m=-2.}
		\end{eqnarray*}
		\begin{itemize}
		\item Với $m=0\Rightarrow M\left(0;-3\right)$;
		phương trình tiếp tuyến là $y=4x-3$ (loại).
		\item Với $m=-2\Rightarrow M\left(-2;5\right)$; phương trình tiếp tuyến là $y=4x+13$ (nhận).
		\end{itemize}
		Vậy $M\left(-2;5\right)$.
		}
	\end{ex}%!Cau!%
\begin{ex} %[Thi Thử L1, Trường THPT Phụ Dực- Thái Bình, 2019 ]%[Nguyễn Thế Anh, 12EX8-2019]%
	[1D5B2-3] 
		Tìm điểm $M$ trên đồ thị hàm số $y=\dfrac{x-3}{x+1}$ có đồ thị $(\mathcal{C})$,  biết tiếp tuyến của đồ thị $(\mathcal{C})$ tại $M$ song song với đường thẳng $y=4x-3$. 
		\choice
		{Không tồn tại $M$}
		{$M\left(0;-3\right)$}
		{$M\left(0;-3\right)$ hoặc $M\left(-2;5\right)$}
		{\True $M\left(-2;5\right)$}
		\loigiai{
		Ta có $y'=\dfrac{4}{{(x+1)}^2}$.\\
		Do $M\in(\mathcal{C})\colon y=\dfrac{x-3}{x+1}$ nên $ M\left(m;\dfrac{m-3}{m+1}\right)$.\\
		Vì tiếp tuyến của đồ thị $(\mathcal{C})$ tại $M$ song song với đường thẳng $y=4x-3$ nên ta có 
		\begin{eqnarray*}
		&&y'(m)=4\\&\Leftrightarrow&\dfrac{4}{(m+1)^2}=4\\&\Leftrightarrow&(m+1)^2=1\\&\Leftrightarrow&\hoac{&m+1=1\\ &m+1=-1}\\&\Leftrightarrow&\hoac{&m=0\\ &m=-2.}
		\end{eqnarray*}
		\begin{itemize}
		\item Với $m=0\Rightarrow M\left(0;-3\right)$;
		phương trình tiếp tuyến là $y=4x-3$ (loại).
		\item Với $m=-2\Rightarrow M\left(-2;5\right)$; phương trình tiếp tuyến là $y=4x+13$ (nhận).
		\end{itemize}
		Vậy $M\left(-2;5\right)$.
		}
	\end{ex}%!Cau!%
\begin{ex}%[Thi thử lần 1, THPT Văn Giang - Hưng Yên, 2019]%[Đỗ Đường Hiếu, 12EX-8-2019]%[1D5B2-2]
	Cho hàm số $y=\dfrac{2x+1}{x-1}$. Phương trình tiếp tuyến tại điểm $M(2;5)$ của đồ thị hàm số trên là
	\choice
	{$y=3x-11$}
	{\True $y=-3x+11$}
	{$y=-3x-11$}
	{$y=3x+11$}
	\loigiai{
		Ta có $y'=-\dfrac{3}{(x-1)^2}$; $y'(2)=-3$.\\
		Phương trình tiếp tuyến tại điểm $M(2;5)$ của đồ thị hàm số $y=\dfrac{2x+1}{x-1}$ là
		$$y-5=-3(x-2)\;\text{hay}\; y=-3x+11.$$
	}
\end{ex}%!Cau!%
\begin{ex}%[Thi thử L2, Thanh Chương 1 Nghệ An, 2019]%[Nguyễn Văn Nay, dự án EX8]%[1D5B2-2]
	Cho hàm số $y=\dfrac{x+1}{2x-3}$ có đồ thị $(C)$. Phương trình tiếp tuyến của đồ thị $(C)$ tại điểm có hoành độ bằng $1$ là
	\choice
	{$y=-5x-7$}
	{$y=5x+3$}
	{\True $y=-5x+3$}
	{$y=5x-3$}
	\loigiai{
		Ta có $y'=-\dfrac{5}{(2x-3)^2}$, $\forall x \ne \dfrac{3}{2}$.\\
		Hệ số góc của tiếp tuyến $y'(1)=-5$.\\
		Tung độ tiếp điểm $y(1)=-2$.\\
		Phương trình tiếp tuyến $y=-5(x-1)-2 \Leftrightarrow y=-5x+3$.	
	}
\end{ex}%!Cau!%
\begin{ex}%[Thi thử, Sở GD và ĐT - Bình Thuận, 2019]%[Huỳnh Xuân Tín, 12EX9]%[1D5B2-2]  
	Cho hàm số $y=\dfrac{x+3}{x+2}$ có đồ thị $(H)$. Gọi đường thẳng $\Delta \colon y=ax+b$ là tiếp tuyến của $(H)$ tại giao điểm của $(H)$ với trục $Ox$. Khi đó $a+b$ bằng
	\choice
	{$-\dfrac{10}{49}$}
	{$\dfrac{2}{49}$}
	{\True $-4$}
	{$2$}
	\loigiai{Ta có $(H)$ cắt trục $O x$ tại $M(-3;0)$.\\
		$y'=\dfrac{-1}{(x+2)^2} \Rightarrow y'(-3)=-1$.\\
		Khi đó tiếp tuyến tại điểm $M$ có phương trình $y=-x-3 \Rightarrow a=-1$, $b=-3$.\\
		Vậy $a+b=-4$.
	} 
\end{ex}%!Cau!%
\begin{ex}%[Đề thi thử THPT Quốc gia môn Toán năm 2019, Sở giáo dục và đào tạo Đà Nẵng]%[Cao Thành Thái, dự án 12-EX-9-2019]%[1D5B2-2]
	Phương trình tiếp tuyến của đồ thị hàm số $y=\dfrac{2x^2+1}{x}$ tại điểm có hoành độ bằng $1$ là
	\choice
	{\True $y=x+2$}
	{$y=x-2$}
	{$y=x+3$}
	{$y=3x+3$}
	\loigiai
	{
		Với mọi $x\neq 0$ ta có $y'=\dfrac{2x^2-1}{x^2}$.\\
		Gọi $(x_0;y_0)$ là tiếp điểm. Ta có $x_0=1$, suy ra $y_0 = 3$.\\
		Lại có $y'(1) = 1$.\\
		Phương trình tiếp tuyến cần tìm là $y=1(x-1)+3$ hay $y=x+2$.
	}
\end{ex}%!Cau!%
\begin{ex}%[Thi thử, Lương Thế Vinh, Sở GD và ĐT - Đồng Nai, 2019, lần 2]%[Lê Vũ Hải, 12EX9]%[1D5B2-3]
	Đường thẳng nào sau đây là tiếp tuyến của đồ thị hàm số $y=x^{3}-3x+2$?
	\choice
	{$ y=9x-12 $}
	{\True $ y=9x-14 $}
	{$ y=9x-13 $}
	{$ y=9x-11 $}
	\loigiai{
		Gọi đường thẳng  $(d)$ là tiếp tuyến của đồ thị hàm số $y=x^{3}-3x+2$ tại điểm $x_{0}$ với $y'(x_{0})=9$. \\
		Ta có $y'= 3x^{2}-3$ $\Rightarrow y'(x_{0}) = 3x^{2}_{0} -3$.\\
		Khi đó $y'(x_{0})=9 \Leftrightarrow 3x^{2}_{0} - 3= 9 \Leftrightarrow \hoac{& x_{0}=2 \\& x_{0}=-2.}$\\
		Với $x_{0}=2$ $\Rightarrow (d) \colon y=9x-14$. \\
		Với $x_{0}=-2$ $\Rightarrow (d) \colon y=9x+18$.
	}
\end{ex}%!Cau!%
\begin{ex}%[Thi thử, Toán Học và Tuổi Trẻ (Đề số 3), 2019]%[Đặng Tân Hoài, 12-EX-6-2019]%[1D5B2-3]
	Trên parabol $(P) \colon y=x^2+1$ lấy hai điểm $A(1;2)$, $B(3;10)$. Gọi $M$ là điểm di động trên cung $\stackrel\frown{AB}$ của $(P)$, $M$ khác $A$, $B$. Gọi $S_1$ là diện tích hình phẳng giới hạn bởi $(P)$ và $MA$, gọi $S_2$ là diện tích hình phẳng giới hạn bởi $(P)$ và $MB$. Gọi $(x_0;y_0)$ là tọa độ của điểm $M$ khi $S_1+S_2$ đạt giá trị nhỏ nhất. Tính $x_0^2+y_0^2$. 
	\choice
	{\True $ 29 $}
	{$ 11 $}
	{$ 109 $}
	{$ 5 $}
	\loigiai{\immini{
			Tổng $S_1+S_2$ nhỏ nhất khi $\triangle MAB$ có diện tích lớn nhất. Khi đó $M$ là tiếp điểm của $(P)$ với tiếp tuyến $d$ mà $d \parallel AB$.\\
			Đường thẳng AB có hệ số góc là $k=\dfrac{10-2}{3-1}=4$, $d$ có hệ số góc $y'(x_0)=2x_0$, suy ra $x_0=2 \Rightarrow y_0=5$.\\
			Vậy $M(2;5)$ và $x_0^2+y_0^2=29$.
		}{
		\begin{tikzpicture}[scale=1, font=\footnotesize,line join=round, line cap=round,>=stealth,x=1cm,y=0.5cm]
		\draw[->] (-1.3,0)--(4,0) node[below] {$x$};
		\draw[->] (0,-0.5)--(0,10.5) node[right] {$y$};
		\clip (-1.5,-0.7) rectangle (4,10.7); 
		\fill (0,0) circle(2pt) node[below left]{$O$};
		\fill (1,2) circle(2pt) node[below right]{$A$};
		\fill (3,10) circle(2pt) node[left]{$B$};
		\fill (2,5) circle(2pt) node[right]{$M$};
		\draw[smooth,samples=300,domain=-1.3:3.5] plot(\x,{(\x)^2+1});
		\draw[-] (2,5) -- (1,2) -- (3,10) -- (2,5);
		\end{tikzpicture}
	}
}
\end{ex}