%!Cau!%
\begin{ex}%[Đề tập huấn số 2, Sở GD và ĐT Quảng Ninh, 2019]%[Đỗ Đường Hiếu, 12EX5-19]%[1D2B2-1]
	Tổ của An và Cường có $7$ học sinh. Số cách xếp $7$ học sinh ấy theo hàng dọc mà An đứng đầu hàng, Cường đứng cuối hàng là
	\choice
	{\True $120$}
	{$100$}
	{$110$}
	{$125$}
	\loigiai{
		An đứng đầu hàng, có $1$ cách.\\
		Cường đứng cuối hàng, có $1$ cách.\\
		Xếp $5$ bạn còn lại vào giứa An và Cường, có $5!$ cách.\\
		Vậy, số cách xếp $7$ học sinh ấy theo hàng dọc mà An đứng đầu hàng, Cường đứng cuối hàng là $1\cdot 1\cdot 5!=120$.
	}
\end{ex}%!Cau!%
\begin{ex}%[Hải Phòng, 2018]%[Phan Anh Tiến, 12-EX-05]%[1D2B2-2]
	Một người có $8$ bì thư và $6$ tem thư, người đó cần gửi thư cho $3$ người bạn. Hỏi có bao nhiêu cách chọn $3$ bì thư và $3$ tem thư sau đó dán mỗi tem lên mỗi bì để gửi?
	\choice
	{$ 1120$}
	{$ 40320 $}
	{\True $ 6720 $}
	{$ 241920 $}
	\loigiai{Để thực hiện công việc người đó thực hiện liên tiếp ba bước sau
		\begin{itemize}
			\item Chọn $3$ bì thư trong $8$ bì thư có $\mathrm{C}_8^3$ cách.
			\item Chọn $3$ tem thư trong số $6$ tem thư có $\mathrm{C}_6^3$ cách.
			\item Dán $3$ tem thư vào $3$ bì thư có $3!$ cách.
		\end{itemize}
		Vậy số cách làm là $\mathrm{C}_8^3\cdot \mathrm{C}_6^3\cdot3!=6720$.
	}
\end{ex}%!Cau!%
\begin{ex}%[Đề tập huấn, Sở GD và ĐT - Quảng Ninh, 2019]%[Lê Hồng Phi, 12EX5]%[1D2B2-1]
	Tổ của An và Cường có $7$ học sinh. Số cách sắp xếp $7$ học sinh ấy theo hàng dọc mà An đứng đầu hàng, Cường đứng cuối hàng là
	\choice
	{\True $120$}
	{$100$}
	{$110$}
	{$125$}
	\loigiai
	{Sắp xếp An đứng đầu hàng, Cường đứng cuối hàng có $1$ cách.\\
	Sắp xếp $5$ bạn còn lại vào hàng có $5!$ cách.\\
	Vậy có tất cả $1\times 5!=120$ cách sắp xếp.	
	}
\end{ex}%!Cau!%
\begin{ex}%[Đề tập huấn, Sở GD và ĐT - Quảng Trị, 2018]%[Nguyễn Văn Nay, 12EX10]%[1D2B2-1]
	Từ các chữ số $1$;$2$;$3$;$4$;$5$;$6$;$7$ lập được bao nhiêu số tự nhiên có $5$ chữ số khác nhau, trong đó phải có mặt chữ số $2$?
	\choice
	{$2040$}
	{$1400$}
	{\True $1800$}
	{$1620$}
	\loigiai{
		Có $5$ cách chọn vị trí cho chữ số $2$.\\
		Có $\mathrm{A}_6^4$ cách chọn vị trí cho $4$ chữ số còn lại.\\
		Vậy có $5\cdot \mathrm{A}_6^4=1800$ cách lập số.	
	}
\end{ex}%!Cau!%
\begin{ex}%[Thi thử, Lào Cai - Phú Thọ, 2019]%[Bùi Anh Tuấn, dự án (12EX-5)]%[1D2B2-1]
	Một hội đồng gồm $ 2 $ giáo viên và $ 3 $ học sinh được chọn từ một nhóm $ 5 $ giáo viên và $ 6 $ học sinh. Hỏi có bao nhiêu cách chọn?
	\choice
	{\True $ 200 $}
	{$ 150 $}
	{$ 160 $}
	{$ 180 $}
	\loigiai{
		Chọn $ 2 $ trong $ 5 $ giáo viên có $ \mathrm{C}_5^2=10 $ cách chọn.\\
		Chọn $ 3 $ trong $ 6 $ giáo viên có $ \mathrm{C}_6^3=20 $ cách chọn.\\
		Vậy có $ 10\cdot 20=200 $ cách chọn.
	}
\end{ex}%!Cau!%
\begin{ex}%[Đề Tập Huấn -4, Sở GD và ĐT - Hải Phòng, 2019]%[Trần Xuân Thiện, 12EX5]%[1D2B2-2]
	Một người có $8$ bì thư và $6$ tem thư, người đó cần gửi thư cho $3$ người bạn. Hỏi người đó có bao nhiêu cách chọn $3$ bì thư và $3$ tem thư sau đó dán mỗi tem lên mỗi bì thư để gửi thư?
	\choice
	{$ 1120 $}
	{$ 40320 $}
	{\True $ 6720 $}
	{$ 241920 $}
	\loigiai{
		Để thực hiện công việc đó người đó phải thực hiện liên tiếp ba bước sau:
		\begin{itemize}
			\item Chọn $3$ bì thư trong $8$ bì thư có $\mathrm{C}_8^3 $ cách.
			\item Chọn 3 tem thư trong 6 tem thư có $\mathrm{C}_6^3 $ cách.
			\item Chọn 3 tem thư vào 3 bì thư có có $\mathrm{P}_3 $ cách.
		\end{itemize}
	Vậy theo quy tắc nhân, số cách người đó có thể chọn là $\mathrm{C}_8^3 \cdot \mathrm{C}_6^3 \cdot \mathrm{P}_3$.
	}
\end{ex}%!Cau!%
\begin{ex}%[Thi thử, Sở GD và ĐT - Hà Tĩnh, 2019]%[Đặng Tân Hoài, 12-EX-5-2019]%[1D2B2-1]
Có bao nhiêu số tự nhiên có $2$ chữ số khác nhau lấy từ tập $X=\{1;2;3;4;5\}$?
	\choice
	{$5^2$}
	{$P_5$}
	{\True $\mathrm{A}_5^2$}
	{$\mathrm{C}_5^2$}
	\loigiai{
	Lấy $2$ chữ số khác nhau (có tính thứ tự) từ tập có $5$ chữ số thì ta lập được một số tự nhiên mà chữ số đầu của nó khác $0$.\\
	Vậy có $\mathrm{A}_5^2$ số tự nhiên cần tìm.
	}
\end{ex}%!Cau!%
\begin{ex}%[Thi thử L1, Star Education HCM, 2019]%[Nguyễn Ngọc Dũng, dự án 12EX6]%[1D2B2-1]
Có bao nhiêu số chẵn có $4$ chữ số và các chữ số phân biệt?
\choice
{$2250$}
{$2560$}
{\True $2296$}
{$2520$}
\loigiai{
Gọi số cần tìm có dạng $\overline{a_1a_2a_3a_4}$.
\begin{enumerate}
\item Trường hợp $a_1$ có thể bằng $0$:
\begin{itemize}
\item $a_4\in \{0;2;4;6;8\}$ nên có $5$ cách chọn.
\item Chọn $3$ số còn lại có $\mathrm{A}^3_9$ cách chọn. 
\end{itemize}
Suy ra có $5\cdot \mathrm{A}^3_9$ cách chọn.
\item Trường hợp số $a_1=0$:
\begin{itemize}
\item $a_4\in \{2;4;6;8\}$ nên có $4$ cách chọn.
\item Chọn $2$ số còn lại có $\mathrm{A}^2_8$ cách chọn. 
\end{itemize}
Suy ra có $4\cdot \mathrm{A}^2_8$ cách chọn.
\end{enumerate}
Vậy số các chữ số chẵn có $4$ chữ số phân biệt là
$$5\cdot \mathrm{A}^3_9 - 4\cdot \mathrm{A}^2_8 = 2296.$$	
}
\end{ex}%!Cau!%
\begin{ex}%[THPT Đức Thọ - Hà Tĩnh - Lần 1 - 2019]%[Phan Anh - EX7]%[1D2B2-1]
Cần chọn $4$ người đi công tác từ một tổ có $40$ người, khi đó số cách chọn là
	\choice
	{\True $\mathrm{C}^4_{40}$}
	{$10$}
	{$4^{40}$}
	{$\mathrm{A}^4_{40}$}
	\loigiai{Mỗi một cách chọn $4$ người ta $40$ người là một tổ hợp chập $4$ của $40$ phần tử, vậy có $\mathrm{C}^4_{40}$ cách chọn.}
\end{ex}%!Cau!%
\begin{ex}%[Thi thử, Lý Thái Tổ - Bắc Ninh, 2019]%[Nguyện Ngô, 12EX7]%[1D2B2-1]
Một tổ có $7$ học sinh nam và $5$ học sinh nữ. Hỏi có bao nhiêu cách chọn hai bạn trực nhật sao cho có nam và nữ?
\choice
{\True $35$}
{$49$}
{$12$}
{$25$}
\loigiai{
Số cách chọn hai bạn trực nhật sao cho có nam và nữ là $\mathrm{C}^1_7\cdot\mathrm{C}^1_5=35$ cách.
}
\end{ex}%!Cau!%
\begin{ex}%[Thi thử, THPT Trần Phú - Hà Tĩnh, lần 2, 2019]%[Đỗ Đường Hiếu, 12-EX-7-2019]%[1D2B2-1]
	Cho tập hợp $M$ có $10$ phần tử. Số tập con gồm $3$ phần tử của $M$ là
	\choice
	{$\mathrm{A}_{10}^3$}
	{$3^{10}$}
	{\True $\mathrm{C}_{10}^3$}
	{$10^3$}
	\loigiai{
		Mỗi tập con gồm $3$ phần tử của $M$ là một tổ hợp chập $3$ của tập hợp $M$. Do đó, số tập con gồm $3$ phần tử của $M$ là $\mathrm{C}_{10}^3$. 	
	}
\end{ex}%!Cau!%
\begin{ex}%[2-TT-26- Đề thi thử Toán THPT Quốc gia 2019, Trường THPT Nam Tiền Hải – Thái Bình]%[Nguyễn Thế Anh, dự án EX7]%[1D2B2-1]
Có tất cả bao nhiêu số tự nhiên có hai chữ số sao cho các chữ số khác nhau và đều khác $0$?
\choice
{$9^2$}
{\True $\mathrm{A}^2_9$}
{$\mathrm{C}^2_9$}
{$90$}
\loigiai{
Số các số tự nhiên có hai chữ số thỏa mãn là số các chỉnh hợp chập $2$ của $9$ phần tử $\{1;2;\ldots ;9\}$ nên có $\mathrm{A}^2_9$ số có hai chữ số khác nhau và đều khác $0$.
}
\end{ex}%!Cau!%
\begin{ex}%[Thi thử, Chuyên Đại học Vinh, 2019]%[Huỳnh Xuân Tín, 12EX7]%[1D2B2-2]
	Mệnh đề nào sau đây \textbf{sai}?
	\choice
	{Số tập con có $4$ phần tử của tập $6$ phần tử là $\mathrm{C}_6^4$}
	{Số cách xếp $4$ quyển sách đôi một khác nhau vào $4$ trong $6$ vị trí ở trên giá là  $\mathrm{A}_6^4$}
	{\True Số cách chọn và xếp thứ tự $4$ học sinh từ nhóm $6$ học sinh là $\mathrm{C}_6^4$}
	{Số cách xếp $4$ quyển sách trong $6$ quyển sách đôi một khác nhau vào $4$ vị trí trên giá là $\mathrm{A}_6^4$}
	\loigiai{Ta lần lượt xét các mệnh đề cho
		\begin{itemize}
			\item Mệnh đề \lq\lq Số tập con có $4$ phần tử của tập $6$ phần tử là $\mathrm{C}_6^4$\rq\rq\, \textbf{ĐÚNG}.\, Vì: Lấy ngẫu nhiên $4$ phần tử từ tập $6$ phần tử ta được một tập con của $6$ phần tử. Vậy số	tập con có $4$ phần tử của tập $6$ phần tử là $\mathrm{C}_6^4$.
			\item Mệnh đề \lq\lq Số cách xếp $4$ quyển sách vào $4$ trong $6$ vị trí ở trên giá là  $\mathrm{A}_6^4$\rq\rq \,\,\textbf{ĐÚNG}.\, Vì: Mỗi cách sắp xếp $4$ quyển sách trong $6$ quyển sách là một chỉnh hợp chập $4$ của $6$
			quyển sách. Vậy số cách sắp xếp $4$ quyển sách vào $4$ vị trí trong $6$ vị trí trên giá là $\mathrm{A}_6^4$.
			\item Mệnh đề \lq\lq Số cách chọn và xếp thứ tự $4$ học sinh từ nhóm $6$ học sinh là $\mathrm{C}_6^4$\rq\rq\, \textbf{SAI}. \, Vì: Mỗi cách lựa chọn và xếp thứ tự $4$ học sinh từ nhóm $6$ học sinh là một chỉnh chập $4$ của	$6$ học sinh. Vậy số cách lựa chọn và xếp thứ tự $4$ học sinh từ nhóm $6$ học sinh là $\mathrm{A}_6^4$.
			\item Mệnh đề \lq\lq Số cách xếp $4$ quyển sách trong $6$ quyển sách vào $4$ vị trí trên giá là $\mathrm{A}_6^4$\rq\rq \,\,\textbf{ĐÚNG}.\, Vì: Mỗi cách sắp xếp $4$ quyển sách trong $6$ quyển sách vào $4$ vị trí là một chỉnh hợp chập
			$4$ của $6$ quyển sách. Vậy số cách sắp xếp $4$ quyển sách trong $6$ vào $4$ vị trí trên giá là $\mathrm{A}_6^4$.
	\end{itemize}	}
\end{ex}%!Cau!%
\begin{ex}%[Thử sức trước kì thi đề số 4, THTT, 2019]%[Trần Hòa, 12EX-7-2019]%[1D2B2-1]
	Trong kho đèn trang trí đang còn $5$ bóng đèn loại $I$, $7$ bóng đèn loại $II$, các bóng đèn đều khác nhau về màu sắc và hình dáng. Lấy ra $5$ bóng đèn bất kì. Hỏi có bao nhiêu khả năng xảy ra số bóng đèn loại $I$ nhiều hơn số bóng đèn loại $II$?
	\choice
	{\True $246$}
	{$3480$}
	{$245$}
	{$3360$}
	\loigiai{
	Số cách chọn ra $5$ bóng đèn bất kì sao cho số bóng đèn loại $I$ nhiều hơn số bóng đèn loại $II$ là 
		{\allowdisplaybreaks
	\begin{eqnarray*}
		\mathrm{C}_5^5+\mathrm{C}_5^4\cdot\mathrm{C}_7^1+\mathrm{C}_5^3\cdot\mathrm{C}_7^2=246.
	\end{eqnarray*}
	}		
	}
\end{ex}%!Cau!%
\begin{ex}%[Thử sức trước kì thi đề số 4, THTT, 2019]%[Trần Hòa, 12EX-7-2019]%[1D2B2-1]
	Từ các chữ số $0$, $1$, $2$, $3$, $4$, $5$, $6$, $7$ có thể lập được bao nhiêu số tự nhiên có $5$ chữ số đôi một khác nhau sao cho có đúng $3$ chữ số chẵn và $2$ chữ số lẻ?
	\choice
	{\True $2448$}
	{$3600$}
	{$2324$}
	{$2592$}
	\loigiai{
	Để lập được  số tự nhiên có $5$ chữ số đôi một khác nhau sao cho có đúng $3$ chữ số chẵn và $2$ chữ số lẻ có các trường hợp sau
	\begin{itemize}
        \item TH1: $3$ chữ số chẵn chọn ra có mặt chữ số $0$.
                \begin{itemize}
                \item Chọn hai chữ số chẵn khác số $0$ có $\mathrm{C}_3^2$ cách.
                \item Chọn hai chữ số lẻ có $\mathrm{C}_4^2$ cách.
                \item Số cách lập thoả mãn trong trường hợp này là $(5!-4!)\cdot\mathrm{C}_3^2\cdot \mathrm{C}_4^2=1728$.
                \end{itemize}
        \item TH2: Số lập ra không có chứa chữ số $0$, số cách lập thỏa mãn là $\mathrm{C}_3^3\cdot \mathrm{C}_4^2\cdot 5!=720$.
        \item Vậy tất cả có $1728+720=2448$ cách.
        
        \end{itemize}	
	}
\end{ex}%!Cau!%
\begin{ex}%[Chuyên ĐHSPHN - 19]%[Phan Anh - EX8]%[1D2B2-1]
Cho $n$ là số tự nhiên lớn hơn $2$. Số các chỉnh hợp chập $2$ của $n$ phần tử là
\choice
{$\dfrac{n(n-1)}{2!}$}
{$2!\cdot n(n-1)$}
{\True $n(n-1)$}
{$2n$}
\loigiai{Theo lý thuyết ta có $\mathrm{A}_n^2=\dfrac{n!}{(n-2)!}=n(n-1)$.}
\end{ex}%!Cau!%
\begin{ex}%[2-TT-5- Đề thi tháng 2-2019, Toán 12 trường THPT chuyên Bắc Giang- 2019]%[Nguyễn Thế Anh-EX6]%[1D2B2-4]
Dãy số nào dưới đây là dãy số bị chặn?
\choice
{\True $u_n=\dfrac{n}{n+1}$}
{$u_n=\sqrt{n^2+1}$}
{$u_n=2^n+1$}
{$u_n=n+\dfrac{1}{n}$}
\loigiai{
Với $n\in\mathbb{N}^*$ ta có 
\begin{itemize}
\item $u_n=\dfrac{n}{n+1}$ bị chặn vì $\dfrac{1}{2}\leq \dfrac{n}{n+1}<1$.
\item $u_n=\sqrt{n^2+1}$ không bị chặn vì $\lim\limits_{n\to +\infty}\sqrt{n^2+1}=+\infty$.
\item $u_n=2^n+1$ không bị chặn vì $\lim\limits_{n\to +\infty}(2^n+1)=+\infty$.
\item $u_n=n+\dfrac{1}{n}$ không bị chặn vì $\lim\limits_{n\to +\infty}\left(n+\dfrac{1}{n}\right)=+\infty$.
\end{itemize}
}
\end{ex}%!Cau!%
\begin{ex}%[TT, THPT Kim Liên, Hà Nội-L2]%[Nguyễn Quang Dũng, dự án 12 EX-8-2019]%[1D2B2-6]
Tìm tất cả các số tự nhiên $n$ thỏa mãn $\mathrm{P}_n\cdot\mathrm{A}_n^2+72=6\left(\mathrm{A}_n^2+2\mathrm{P}_n\right)$.
\choice
{$n\in\{-3;3;4\}$}
{\True $n\in\{3;4\}$}
{$n=3$}
{$n=4$}
\loigiai{
Điều kiện $n\in\mathbb{N},n\ge 2$.\\
Ta có 
\begin{eqnarray*}
\mathrm{P}_n\cdot\mathrm{A}_n^2+72=6\left(\mathrm{A}_n^2+2\mathrm{P}_n\right)
&\Leftrightarrow & n!\cdot\dfrac{n!}{(n-2)!}+72=6\left(\dfrac{n!}{(n-2)!}+2n!\right)\\
& \Leftrightarrow &\left(n!-6\right)\left(\dfrac{n!}{(n-2)!}-12\right)=0\\
& \Leftrightarrow & \hoac{& n!=6\\& \dfrac{n!}{(n-2)!}=12}
\Leftrightarrow \hoac{&n=3\\&n^2-2n-12=0}\Leftrightarrow \hoac{&n=3\\&n=4.}
\end{eqnarray*}
Vậy tập tất cả các giá trị $m$ thỏa mãn là $\{3;4\}$.
}
\end{ex}%!Cau!%
\begin{ex}%[Thi thử, THPT Phan Đình Phùng - Đắc Lắc, 2019]%[Nguyễn Minh Hiếu, 12EX8]%[1D2B2-3]
	Số giao điểm tối đa của $5$ đường tròn phân biệt là
	\choice
	{\True $ 20 $}
	{$ 22 $}
	{$ 18 $}
	{$ 10 $}
	\loigiai{
		Hai đường tròn phân biệt có tối đa hai giao điểm.\\
		Do đó số giao điểm tối đa của $5$ đường tròn phân biệt là $2\cdot \mathrm{C}_5^2=20$.
	}
\end{ex}%!Cau!%
\begin{ex}%[Thi thử, Trần Đại Nghĩa - Đắc Lắk, 2019]%[Trần Nhân Kiệt, 12EX8-2019]%[1D2B2-4]
	Sắp xếp $20$ người vào $2$ bàn tròn $A$, $B$ phân biệt, mỗi bàn gồm $10$ chỗ ngồi. Số cách sắp xếp là
	\choice
	{\True $\mathrm{C}_{20}^{10}\cdot 9!\cdot 9!$}
	{$\mathrm{C}_{20}^{10}\cdot 10!\cdot10!$}
	{$\dfrac{\mathrm{C}_{20}^{10}\cdot 9!\cdot 9!}{2}$}
	{$2\mathrm{C}_{20}^{10}\cdot 9!\cdot 9!$}
\loigiai{
\begin{itemize}
\item Chọn $10$ người trong $20$ người để xếp vào bàn tròn $A$ $\Rightarrow$ có $\mathrm{C}_{20}^{10}$ cách chọn.
\item Xếp $10$ người đã chọn vào bàn tròn $A$ có $9!$ cách xếp.
\item Xếp $10$ người còn lại vào bàn tròn $B$ có $9!$ cách xếp.
\end{itemize}
Vậy có $\mathrm{C}_{20}^{10}\cdot 9!\cdot 9!$ cách xếp.
}
\end{ex}%!Cau!%
\begin{ex}%[Thi thử L2, THPT Hà Huy Tập - Hà Tĩnh, 2019]%[Phan Ngọc Toàn, dự án EX8]%[1D2B2-1]
Cho $\mathrm{C}_n^3=10$	 thì $n$ có giá trị bằng
\choice
{$6$}
{\True $5$}
{$3$}
{$4$}
\loigiai{
Điều kiện: $\heva{& n\in \mathbb{N}\\ & n\geq 3.}$\\
Ta có
\begin{align*}
\mathrm{C}_n^3=10  &\Leftrightarrow  \dfrac{n!}{3!(n-3)!}=10\\
 & \Leftrightarrow   n(n-1)(n-2)=60\\
&\Leftrightarrow  n^3-3n^2+2n-60=0\\
&\Leftrightarrow (n-5)\left(n^2+2n+12\right)=0\\
&\Leftrightarrow  \hoac{& n=5 \ (\text{nhận})\\& n^2+2n+12=0 \ (\text{Vô nghiệm}).} \end{align*}
Vậy giá trị cần tìm là $n=5$.
}

\end{ex}%!Cau!%
\begin{ex}%[Thi thử, Toán học tuổi trẻ - Đề 6, 2019]%[Phan Văn Thành, 12EX9]%[1D2B2-1]
	Một giải thi đấu bóng đá quốc gia có $16$ đội thi đấu vòng tròn $2$ lượt tính điểm. Hai đội bất kỳ đều đấu với nhau đúng hai trận. Sau mỗi trận đấu, đội thắng được $3$ điểm, đội thua $0$ điểm, nếu hòa mỗi đội được $1$ điểm. Sau giải đấu, Ban tổ chức thống kê được $80$ trận hòa. Hỏi tổng số điểm của tất cả các đội có được sau giải đấu bằng bao nhiêu? 
	\choice
	{$ 720 $}
	{$ 560 $}
	{$ 280 $}
	{\True $ 640 $}
	\loigiai{Chọn hai đội trong $16$ đội để thi đấu với nhau có $\mathrm{C}^2_{16} = 120$ cách.\\
	Do hai đội thi đấu với nhau đúng hai trận nên có tổng số trận đấu của giải bóng đá là $120 \cdot 2 = 240$ trận.\\
	Do giải đấu có $80$ trận hòa nên số trận đấu không hòa là $240 - 80 = 160$ trận.\\
	Vậy tổng số điểm của tất cả các đội là $160 \cdot 3 + 80 \cdot 2 = 640$ điểm. 
	}
\end{ex}%!Cau!%
\begin{ex}%[Thi thử, Toán Học và Tuổi Trẻ (Đề số 3), 2019]%[Đặng Tân Hoài, 12-EX-6-2019]%[1D2B2-1]
	Một tập hợp $M$ có tất cả $2^{2018}$ tập con. Hỏi $M$ có bao nhiêu tập con có ít nhất $2017$ phần tử?	
	\choice
	{\True $ 2019 $}
	{$ 2018 $}
	{$ \dfrac{2017 \times 2018}{2} $}
	{$ 2^{2017} $}
	\loigiai{
		Một tập hợp có $n$ phần tử thì số tập hợp con của nó là $2^n$. Theo giả thiết, ta có tập hợp $M$ có $2018$ phần tử. Do đó
		\begin{itemize}
			\item Số tập con có đúng $2017$ phần tử là $\mathrm{C}_{2018}^{2017}=2018$.
			\item Số tập con có đúng $2018$ phần tử là $1$.\\
			Vậy có $2019$ tập con cần tìm.
		\end{itemize}
	}
\end{ex}%!Cau!%
\begin{ex}%[Đề dự đoán số 6 từ câu 1 đến 35]%[Đăng Tạ, dự án EX-8]%[1D2B2-1]
	Với $0\leq k \leq n$ và $k,n\in \mathbb{Z}$ thì
	\choice
	{$\mathrm{A}_n^k=\dfrac{\mathrm{C}_n^k }{ \mathrm{P}_n}$}
	{$\mathrm{A}_n^k=\mathrm{C}_n^k \cdot \mathrm{P}_n$}
	{$\mathrm{C}_n^k=\mathrm{A}_n^k \cdot \mathrm{P}_n$}
	{\True $\mathrm{C}_n^k=\dfrac{\mathrm{A}_n^k}{  \mathrm{P}_k}$}
	\loigiai{
		$\mathrm{A}_n^k=\dfrac{n!}{(n-k)!}=\dfrac{n!}{k!(n-k)!}\cdot k!=\mathrm{C}_n^k\cdot \mathrm{P}_k$.
	}
\end{ex}