%!Cau!%
\begin{ex}%[KSCL, Sở GD và ĐT - Thanh Hóa, 2018]%[Bùi Ngọc Diệp, 12EX-5]%[1D1Y1-1]
	Cho các mệnh đề sau
	\begin{enumerate}
		\item Hàm số $f(x) = \dfrac{\sin x}{x^2 + 1}$ là hàm số chẵn.
		\item Hàm số $f(x) = 3 \sin x + 4 \cos x$ có giá trị lớn nhất bằng $5$.
		\item Hàm số $f(x) = \tan x$ tuần hoàn với chu kì $2 \pi$.
		\item Hàm số $f(x) = \cos x$ đồng biến trên khoảng $(0; \pi)$.
	\end{enumerate}
	Trong các mệnh đề trên có bao nhiêu mệnh đề đúng?
	\choice
	{$0$}
	{\True $1$}
	{ $2$}
	{ $3$}
	\loigiai{
		\begin{itemize}
			\item Hàm số $f(x) = \dfrac{\sin x}{x^2 + 1}$ là hàm số lẻ. Suy ra mệnh đề a) sai.
			\item Hàm số $f(x) = 3 \sin x + 4 \cos x$ có giá trị lớn nhất bằng $\sqrt{3^2 + 4^2} = 5$ suy ra mệnh đề b) đúng.
			\item Hàm số $f(x) = \tan x$ tuần hoàn với chu kì $\pi$ suy ra mệnh đề c) sai.
			\item Hàm số $f(x) = \cos x$ nghịch biến trên khoảng $(0; \pi)$ suy ra mệnh đề d) sai.
		\end{itemize}
		Vậy có $1$ mệnh đề đúng trong các mệnh đề đã cho.}
\end{ex}%!Cau!%
\begin{ex}%[Đề thi thử THPTQG lần 2 THPT Thoại Ngọc Hầu, An Giang, năm 2019]%[Nguyễn Thành Khang, dự án 2019-Ex-7]%[1D1Y1-1]
	Tìm điều kiện để hàm số $y=\dfrac{2\cos x}{\sin x-1}$ có nghĩa.
	\choice
	{$x\ne\dfrac{\pi}{2}+k\pi\ (k\in\mathbb{Z})$}
	{$x\ne k2\pi\ (k\in\mathbb{Z})$}
	{\True $x\ne\dfrac{\pi}{2}+k2\pi\ (k\in\mathbb{Z})$}
	{$x\ne k\pi\ (k\in\mathbb{Z})$}
	\loigiai{
		Hàm số đã cho có nghĩa khi $\sin x\ne 1\Leftrightarrow x\ne\dfrac{\pi}{2}+k2\pi\ (k\in\mathbb{Z})$.
	}
\end{ex}