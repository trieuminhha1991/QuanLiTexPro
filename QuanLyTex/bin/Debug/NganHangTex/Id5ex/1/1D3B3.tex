%!Cau!%
\begin{ex}%[De tap huan, So GD&DT Dien Bien, 2019]%[Ngoc Diep, dự án EX5]%[1D3B3-3]
	Cho cấp số cộng có $u_5 =21, u_6 =27$. Tìm công sai $d$.
	\choice
	{$d=5$}
	{$d=7$}
	{$d=8$}
	{\True $d=6$}
	\loigiai{
		Theo định nghĩa cấp số cộng ta có $d= u_6 -u_5 = 6$.
	}
\end{ex}%!Cau!%
\begin{ex}%[Đề tập huấn, Sở GD và ĐT - Quảng Trị, 2018]%[Nguyễn Văn Nay, 12EX10]%[1D3B3-5]
	Một cấp số cộng có số hạng thứ năm và thứ chín lần lượt là $3$ và $35$. Tính tổng $30$ số hạng đầu tiên của cấp số cộng đó.
	\choice
	{$203$}
	{$2618$}
	{\True $2610$}
	{$5220$}
	\loigiai{
		Gọi $u_1$ và $d$ lần lượt là số hạng đầu và công sai của cấp số cộng.\\
		Theo đề ta có $\heva{&u_5=3\\&u_9=35} \Leftrightarrow \heva{&u_1+4d=3\\&u_1+8d=35} \Leftrightarrow \heva{&u_1=-29\\&d=8.}$\\
		Suy ra tổng $S_{30}=\dfrac{30(2u_1+29d)}{2}=2610$.	
	}
\end{ex}%!Cau!%
\begin{ex}%[Đề tập huấn tỉnh Lai Châu,2019]%[Nguyễn Trung Kiên, dự án 12-EX-5-2019]%[1D3B3-3]
	Cho dãy số $(u_n)$ có $u_1=1$, $u_{n+1}=u_n+2$, $\forall n\in\mathbb{N}, n\geq 1$. Khẳng định nào sau đây là khẳng định \textbf{đúng}?
	\choice
	{\True $u_5=9$}
	{$u_3=4$}
	{$u_2=2$}
	{$u_6=13$}
	\loigiai
	{Từ giả thiết $u_1=1$, $u_{n+1}=u_n+2$, $\forall n\in\mathbb{N}, n\geq 1$ ta có $(u_n)$ là cấp số cộng có số hạng đầu $u_1=1$, công sai $d=2$ nên $u_n=1+2(n-1)=2n-1, \forall n\in\mathbb{N^*}$.\\
	Vậy $u_2=3$, $u_3=5$, $u_5=9$, $u_6=11$.}
\end{ex}%!Cau!%
\begin{ex}%[Thi thử, THPT chuyên KHTN Hà Nội, 2019]%[KV Thanh, 12EX5]%[1D3B3-2]
	Cho $\left(u_{n}\right)$ là một cấp số cộng thỏa mãn $u_{1}+u_{3}=8$ và $u_{4}=10$. Công sai của cấp số cộng đã cho bằng
	\choice
	{\True $3$}
	{$6$}
	{$2$}
	{$4$}
	\loigiai{	
	Gọi $d$ là công sai của cấp số cộng đã cho. Khi đó, ta có
	$$\heva{&u_1+u_3=8\\&u_4=10}\Leftrightarrow\heva{&u_1+u_1+2d=8\\&u_1+3d=10}\Leftrightarrow\heva{&2u_1+2d=8\\&u_1+3d=10}\Leftrightarrow\heva{&u_1+d=4\\&u_1+3d=10}\Leftrightarrow\heva{&u_1=1\\&d=3.}$$
	Vậy công sai của cấp số cộng đã cho bằng $3$.
	} 
\end{ex}%!Cau!%
\begin{ex}%[Thi thử L1, Star Education HCM, 2019]%[Nguyễn Ngọc Dũng, dự án 12EX6]%[1D3B3-5]
Cho dãy số $\left( u_n \right)$ xác định bởi $\heva{& u_1=321 \\  & u_{n+1}=u_n-3}$ với mọi $n\ge 1$ . Tính tổng của $125$ số hạng đầu tiên của dãy số $\left(u_n \right)$.
\choice
{$63375$}
{$16687{,}5$}
{\True $16875$}
{$63562{,}5$}
\loigiai{
Với dãy số $(u_n)$ xác định như trên ta dễ thấy $(u_n)$ là cấp số cộng có số hạng đầu là $u_1=321$, công sai $d=-3$. Do đó, tổng của $125$ số hạng đầu của $(u_n)$ là
$$S_{125} = nu_1+n(n-1)\cdot \dfrac{d}{2} =16875.$$
}
\end{ex}%!Cau!%
\begin{ex}%[THPT Đức Thọ - Hà Tĩnh - Lần 1 - 2019]%[Phan Anh - EX7]%[1D3B3-3]
Cho cấp số cộng $\left(u_n\right)$ có $u_1=-2$ và công sai $d=3$. Tìm số hạng $u_{10}$.
	\choice
	{$u_{10}=-29$}
	{$u_{10}=28$}
	{\True $u_{10}=25$}
	{$u_{10}=-2\cdot3^9$}
	\loigiai{Ta có $u_{10}=u_1+9d=-2+9\times3=25$.}
\end{ex}%!Cau!%
\begin{ex}%[Thi Thử Lần 1, THPT Chuyên Lê Khiết - Quảng Ngãi, 2019]%[Dương BùiĐức, dự án 12EX7]%[1D3B3-1] 
	Cho cấp số nhân $\left( u_n\right) $ có $u_{1}=-1$, $q=-\dfrac{1}{10}$. Số $\dfrac{1}{10^{103}}$ là số hạng thứ mấy?
	\choice
	{Số hạng thứ 101}
	{Số hạng thứ 102}
	{Số hạng thứ 103}
	{\True Số hạng thứ 104}
	\loigiai
	{
		Vì $\left(u_n \right) $ là cấp số nhân nên ta có 
		\allowdisplaybreaks
		\begin{eqnarray*}
			u_n=u_1\cdot q^{n-1}\Leftrightarrow \dfrac{1}{10^{103}}=(-1)\cdot \left(-\dfrac{1}{10} \right)^{n-1} \Leftrightarrow \left(-\dfrac{1}{10} \right)^{103}=\left(-\dfrac{1}{10} \right)^{n-1}\Leftrightarrow n-1=103\Leftrightarrow n=104.
		\end{eqnarray*}
	}
\end{ex}%!Cau!%
\begin{ex}%[Thi thử L1, THPT Ngô Quyền - Hà Nội, 2019]%[Phan Ngọc Toàn, dự án EX7]%[1D3B3-1] 
	Trong các dãy số sau, dãy số nào \textbf{không phải} cấp số cộng?
	\choice
	{$1;1;1;1;1$}
	{$-8;-6;-4;-2;0$}
	{\True $3;1;-1;-2;-4$}
	{$\dfrac{1}{2}; \dfrac{3}{2}; \dfrac{5}{2}; \dfrac{7}{2}; \dfrac{9}{2}$}
	\loigiai{
\begin{itemize}
	\item $1;1;1;1;1$ là $1$ cấp số cộng với $\heva{&u_1=1\\&d=0.}$
	\item $-8;-6;-4;-2;0$ là $1$ cấp số cộng với $\heva{&u_1=-8\\&d=-2.}$
	\item $\dfrac{1}{2}; \dfrac{3}{2}; \dfrac{5}{2}; \dfrac{7}{2}; \dfrac{9}{2}$ là $1$ cấp số cộng với $\heva{&u_1=\dfrac{1}{2}\\&d=1.}$
\end{itemize}	
}
\end{ex}%!Cau!%
\begin{ex}%[Thi thử, Chuyên Đại học Vinh, 2019]%[Huỳnh Xuân Tín, 12EX7]%[1D3B3-3]
	Cho cấp số cộng $\left(u_n \right) $, có $u_1=-2$, $u_4=4$. Số hạng $u_6$ là
	\choice
	{$8$}
	{$6$}
	{\True $10$}
	{$12$}
	\loigiai{Áp dụng công thức của cấp số cộng $u_n=u_1+(n-1)d$, ta có
		$$u_4=u_1+3d\Leftrightarrow 4=-2+3d\Leftrightarrow d=2.$$
		Vậy $u_6=u_1+5d=-2+5\cdot2=8$.}
\end{ex}%!Cau!%
\begin{ex}%[Thi thử L1, Chuyên Lê Quý Đôn,Lai Châu, 2019]%[Nguyễn Tài Tuệ, dự án EX7]%[1D3B3-6]
	Một người muốn chia $1.000.000$ đồng cho bốn người con, đứa lớn hơn đứa nhỏ kế
	tiếp là $100.000$ đồng. Hỏi đứa con lớn nhất được bao nhiêu tiền?
	\choice
	{$200.000$ đồng}
	{$300.000$ đồng}
	{\True $400.000$ đồng}
	{$100.000$ đồng}
	\loigiai
	{
		Số tiền của bốn người con theo thứ tự từ đứa lớn nhất đến đứa nhỏ nhất lập thành cấp số cộng với công sai $ d=-100.000 $ đồng. Mà tổng số tiền của $ 4 $ người con là $ 1.000.000 $ đồng, nên gọi số tiền của đứa lớn nhất là $ x $ thì ta có
		$$\dfrac{4}{2}(2 x+3 d)=1.000.000 \Leftrightarrow 2(2 x-300.000)=1.000.000 \Leftrightarrow x=400.000.$$
		Vậy đứa lớn nhất được $400.000$ đồng.
	}
\end{ex}%!Cau!%
\begin{ex}%[Đề thi thử L2, Liên trường Nghệ An, 2019]%[Nguyễn Đắc Giáp, dự án 12EX8]%[1D3B3-5]
	Cho dãy số $(u_n)$ có $u_1=-5$, $u_{n+1}=u_n+2$, $n \in \mathbb{N}^*$. Tổng $S_5=u_1+u_2+ \cdots +u_5$ bằng
	\choice
	{$5$}
	{\True $-5$}
	{$-15$}
	{$-24$}
	\loigiai{
		Với mọi $n \in \mathbb{N}^*$, ta có $u_{n+1}=u_n+2 \Leftrightarrow u_{n+1}-u_n=2$.
		Suy ra $(u_n)$ là cấp số cộng có số hạng đầu $u_1=-5$ và công sai $d=2$.\\
		Do đó 
		$$S_5=u_1+u_2+ \cdots +u_5=\dfrac{5}{2}(2u_1+4d)=\dfrac{5}{2}\left[2 \cdot (-5)+4 \cdot 2\right]=-5.$$
	}
\end{ex}%!Cau!%
\begin{ex}%[Chuyên ĐHSPHN - 19]%[Phan Anh - EX8]%[1D3B3-2]
Cho cấp số cộng $\left(u_n\right)$ có $u_1=-5$, công sai $d=4$. Khẳng định nào sau đây là đúng?
\choice
{$u_n=-5.4^{n-1}$}
{$u_n=-5+4^{n-1}$}
{\True $u_n=-5+4(n-1)$}
{$u_n=-5.4^{n}$}
\loigiai{
Công thức tổng quát của cấp số cộng $\left(u_n\right)$ có số hạng đầu là $u_1$ và công sai $d$ là
$$u_n=u_1+(n-1)d=-5+4(n-1).$$}
\end{ex}%!Cau!%
\begin{ex}%[KTCL 12 L4, Ninh Bình - Bạc Liêu, 2018-2019]%[Vũ Nguyễn Hoàng Anh, 12EX8-19]%[1D3B3-2]
	Cho $\left(u_{n}\right)$ là một cấp số cộng thỏa mãn $u_{1}+u_{3}=8$ và $u_{4}=10$. Công sai của cấp số cộng đã cho bằng
	\choice
	{\True $3$}
	{$6$}
	{$2$}
	{$4$}
	\loigiai{	
		Gọi $d$ là công sai của cấp số cộng đã cho. Khi đó, ta có
		$$\heva{&u_1+u_3=8\\&u_4=10}\Leftrightarrow\heva{&u_1+u_1+2d=8\\&u_1+3d=10}\Leftrightarrow\heva{&2u_1+2d=8\\&u_1+3d=10}\Leftrightarrow\heva{&u_1+d=4\\&u_1+3d=10}\Leftrightarrow\heva{&u_1=1\\&d=3.}$$
		Vậy công sai của cấp số cộng đã cho bằng $3$.
	} 
\end{ex}%!Cau!%
\begin{ex}%[thi thử, THPT Triệu Thái, Vĩnh Phúc]%[Phan Quốc Trí, dự án 12EX-8-2019]%[1D3B3-1]
	Trong các dãy số sau đây, dãy số nào là cấp số cộng?	
	\choice
	{$u_n=3^n$}
	{$u_n=(-3)^{n+1}$}
	{\True $u_n=3n + 1$}
	{Tất cả đều là cấp số cộng}
	\loigiai{
		Dễ thấy $u_n = 3^n$  và $(-3)^{n+1}$ là một cấp số nhân.\\
		Xét $u_n=3n + 1$. Ta có $$u_{n+1}-u_{n}= 3(n+1)+1 - (3n+1)=3.$$
		Do đó $u_n=3n + 1$ là một cấp số cộng.				
	}
\end{ex}%!Cau!%
\begin{ex}%[Thi thử L2, THPT Hà Huy Tập - Hà Tĩnh, 2019]%[Phan Ngọc Toàn, dự án EX8]%[1D3B3-1]
	Trong các dãy số sau, dãy số nào là cấp số cộng?
	\choice
	{$u_n=n^2$}
	{$u_n=(-1)^n \cdot n$}
	{$u_n= \dfrac{n}{3^n}$}
	{\True $u_n=2n$}
	\loigiai{
		Xét dãy số $(u_n)$ với $u_n=2n$, $\forall n \in \mathbb{N}^*$ ta có  $u_{n+1}-u_n=2(n+1)-2n=2$ (không đổi). Do đó dãy số này là một cấp số cộng.	
	}
\end{ex}%!Cau!%
\begin{ex}%[Nguyễn Trung Kiên, dự án 12-EX-6-2019]%[1D3B3-1]
	Cho một cấp số cộng $(u_n)$ biết $u_1=\dfrac{1}{3}$, $u_8=26$. Tìm công sai $d$.
	\choice
	{$d=\dfrac{10}{3}$}
	{\True $d=\dfrac{11}{3}$}
	{$d=\dfrac{3}{11}$}
	{$d=\dfrac{3}{10}$}
	\loigiai
	{Áp dụng công thức tổng quát của cấp số cộng ta có
	$$u_8=u_1+7d\Leftrightarrow d=\dfrac{u_8-u_1}{7}=\dfrac{11}{3}.$$}
\end{ex}%!Cau!%
\begin{ex}%[Thi thử Sở GD và ĐT-Bình Thuận, 2019]%[Sang Nguyen, 12EX9]%[1D3B3-3] 
	Cho cấp số cộng $(u_n)$ biết $u_5=18$ và $4S_n=S_{2n}$. Tìm số hạng đầu $u_1$ và công sai $d$ của cấp số cộng.
	\choice 
	{$u_1=3$, $d=2$}
	{$u_1=2$, $d=3$}
	{$u_1=2$, $d=2$}
	{\True $u_1=2$, $d=4$}   
	\loigiai{
		Từ giả thiết ta có hệ 
		\begin{eqnarray*}
			&&\heva {&u_5=18\\&4S_n=S_{2n}}\\
			&\Leftrightarrow & \heva {&u_1+4d=18\\&4n\cdot \dfrac{u_1+u_n}{2}}=2n\cdot \dfrac{u_1+u_{2n}}{2}\\&\Leftrightarrow& \heva {&u_1+4d=18\\&2(u_1+u_n)=u_1+u_{2n}}\\
			&\Leftrightarrow & \heva {&u_1+4d=18\\&2\left[2u_1+(n-1)d\right]=2u_1+(2n-1)d}\\&\Leftrightarrow& \heva {&u_1+4d=18\\&2u_1-d=0}\\&\Leftrightarrow& \heva {&u_1=2\\&d=4.}
		\end{eqnarray*}
	} 
\end{ex}%!Cau!%
\begin{ex}%[Thi thử L3, Lương Thế Vinh, Hà Nội, 2019]%[Nguyễn Tiến, dự án EX9]%[1D3B3-3]
	Cho cấp số cộng $(u_n)$ có $u_1=-5$ và $d=3$. Mệnh đề nào sau đây \textbf{đúng}?
	\choice
	{$u_{15}=45$}
	{\True $u_{13}=31$}
	{$u_{10}=35$}
	{$u_{15}=34$}
	\loigiai{
		Cấp số cộng $(u_n)$ có $u_1=-5$ và $d=3$ nên ta có
		$$u_{10}=u_1+9d=-22; \, \, u_{13}=u_1+12d=31; \, \, u_{15}=u_1+14d=37.$$
		Từ $4$ đáp án đã cho, ta thấy mệnh đề \textbf{đúng} là \lq\lq $u_{13}=31$\rq\rq.
	}
\end{ex}%!Cau!%
\begin{ex}%[Thi thử, Toán Học và Tuổi Trẻ (Đề số 3), 2019]%[Đặng Tân Hoài, 12-EX-6-2019]%[1D3B3-4]
	Tìm tất cả các giá trị thực của $x$ để $\cos 2x$, $\dfrac{1}{2}\cos 4x$, $\cos 6x$ là ba số hạng liên tiếp trong một cấp số cộng.
	\choice
	{$ x=\dfrac{\pi}{8}+k\dfrac{\pi}{2},~x=\pm \dfrac{\pi}{6}+k\pi,~k \in \mathbb{Z} $}
	{\True $ x=\dfrac{\pi}{8}+k\dfrac{\pi}{4},~x=\pm \dfrac{\pi}{6}+k\pi,~k \in \mathbb{Z} $}
	{$ x=\dfrac{\pi}{2}+k\pi,~x=\pm \dfrac{\pi}{3}+k2\pi,~k \in \mathbb{Z} $}
	{$ x=\dfrac{\pi}{8}+k\pi,~x=\pm \dfrac{\pi}{6}+k2\pi,~k \in \mathbb{Z} $}
	\loigiai{
		Yêu cầu bài toán tương đương $$\cos 6x + \cos 2x =2 \cdot \dfrac{1}{2}\cos 4x \Leftrightarrow 2 \cos 4x  \cos 2x=	\cos 4x \Leftrightarrow \hoac{& \cos 4x=0\\ & \cos 2x=\dfrac{1}{2}} \Leftrightarrow \hoac{& x=\dfrac{\pi}{8}+k\dfrac{\pi}{4}\\& x=\pm \dfrac{\pi}{6}+k\pi} (k \in \mathbb{Z}).$$
	}
\end{ex}%!Cau!%
\begin{ex}%[Thi Thử Lần 2, THPT Lương Thế Vinh - Hà Nội, 2019]%[Dương BùiĐức, dự án 12EX6]%[1D3B3-3]
Cho dãy số $(u_n)$ là một cấp số cộng, biết $u_2+u_{21}=50$. Tính tổng của $ 22 $ số hạng đầu tiên của dãy.
\choice
{$1100$}
{$50$}
{\True $550$}
{$2018$}
\loigiai{
Ta có $ u_{1}+u_{22}=u_{2}+u_{21}=50\Rightarrow S_{22}=\dfrac{22}{2}\cdot (u_{1}+u_{22})=550 $.
}
\end{ex}%!Cau!%
\begin{ex}%[Đề số 4, 2019]%[Phạm An Bình, 12EX8]%[1D3B3-3]
	Cho cấp số cộng $(u_n)$ có số hạng thứ hai $u_2=2$ và công sai $d=3$. Giá trị của $u_4$ bằng
	\choice
	{\True $8$}
	{$11$}
	{$14$}
	{$5$}
	\loigiai{
		Ta có $u_4=u_1+3d=u_2+2d=8$.
	}
\end{ex}%!Cau!%
\begin{ex}%[Đề dự đoán số 6 từ câu 1 đến 35]%[Đăng Tạ, dự án EX-8]%[1D3B3-3]
	Cho cấp số cộng $\left(u_n\right)$ có $u_1=2$ và $u_2=3u_1$. Giá trị $u_4$ bằng
	\choice
	{$u_4=54$}
	{$u_4=18$}
	{\True $u_4=14$}
	{$u_4=12$}
	\loigiai{
		Cấp số cộng trên có $u_1=2$ và công sai $d=u_2-u_1=3u_1-u_1=2u_1=4$\\
		Do đó $u_4=u_1+3d=2+3\cdot 4 = 12$.	
	}
\end{ex}