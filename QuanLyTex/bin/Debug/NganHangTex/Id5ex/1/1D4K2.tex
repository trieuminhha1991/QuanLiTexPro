%!Cau!%
\begin{ex}%[Thi thử, Toán học tuổi trẻ, 2019-2]%[Nguyễn Trường Sơn, 12-EX-5-2019]%[1D4K2-3]
	Giá trị của giới hạn $\lim \limits_{x \to 0} \dfrac{(2^x-1)(3^x-1)\cdots (n^x-1)}{x^{n-1}}$ bằng
	\choice
	{$\ln (n!)$}
	{\True $\ln2 \cdot \ln3 \cdots \ln n$}
	{$n!$}
	{$2+3+\cdots+n$}
	\loigiai{Ta có $\lim \limits_{x \to 0} \dfrac{n^x-1}{x}=\lim \limits_{x \to 0} \dfrac{\mathrm{e}^{\ln n^x}-1}{x}=\ln n\lim \limits_{x \to 0} \dfrac{\mathrm{e}^{x\ln n}-1}{x \ln n}=\ln n$.\\
		Vậy $\lim \limits_{x \to 0} \dfrac{(2^x-1)(3^x-1)\cdots (n^x-1)}{x^{n-1}}=\lim \limits_{x \to 0} \left(  \dfrac{2^x-1}{x} \cdot \dfrac{3^x-1}{x}\cdots \dfrac{n^x-1}{x}\right) =\ln 2 \cdot \ln 3 \cdots \ln n$.
		
	}
\end{ex}%!Cau!%
\begin{ex}%[Thi thử L2, THPT Hà Huy Tập - Hà Tĩnh, 2019]%[Phan Ngọc Toàn, dự án EX8]%[1D4K2-4] 
Trong các bộ bộ số $(a, b)$ là các số nguyên dương thỏa mãn $$\lim\limits_{x\to-\infty}\left(\sqrt{9x^2+ax}+\sqrt[3]{27x^3+bx^2+5}\right)=\dfrac{7}{27},$$ tồn tại bộ số $(a, b)$ thỏa mãn hệ thức nào sau đây?
	\choice
	{$a+2b=33$}	
	{\True $a+2b=34$}	
	{$a+2b=35$}	
	{$a+2b=36$}
	\loigiai{Ta có\\ $
	\lim\limits_{x\to-\infty}\left(\sqrt{9x^2+ax}+\sqrt[3]{27x^3+bx^2+5}\right)=\lim\limits_{x\to-\infty}\left(\sqrt{9x^2+ax}+3x\right)+\lim\limits_{x\to-\infty}\left(\sqrt[3]{27x^3+bx^2+5}-3x\right)$.
\begin{itemize}
\item $I_1=\lim\limits_{x\to-\infty}\left(\sqrt{9x^2+ax}+3x\right)
=\lim\limits_{x\to-\infty}\dfrac{ax}{\left(\sqrt{9x^2+ax}-3x\right)} =-\lim\limits_{x\to-\infty}\dfrac{a}{\left(\sqrt{9+\frac{a}{x}}+3\right)}=-\dfrac{a}{6}
$.
\item Ta có\\ $\begin{aligned}
I_2&=\lim\limits_{x\to-\infty}\left(\sqrt[3]{27x^3+bx^2+5}-3x\right)\\
&=
\lim\limits_{x\to-\infty}\dfrac{bx^2+5}{\sqrt[3]{(27x^3+bx^2+5)^2}+3x\sqrt[3]{27x^3+bx^2+5}+9x^2}\\
&=\lim\limits_{x\to-\infty}\dfrac{b+\dfrac{5}{x^2}}{\left(\sqrt[3]{27+\dfrac{b}{x}+\dfrac{5}{x^3}}\right)^2+3\cdot\sqrt[3]{27+\dfrac{b}{x}+\dfrac{5}{x^3}}+9}=\dfrac{b}{27}.
\end{aligned}$
\end{itemize}
Suy ra $-\dfrac{a}{6}+\dfrac{b}{27}=\dfrac{7}{27}$. Vì $(a,\ b) \in\mathbb{Z}^+$ nên $\heva{&a=2\\&b=16.}$ \\
Do đó $a+2b=34$.
	}
\end{ex}