%!Cau!%
\begin{ex}%[Hải Phòng, 2018]%[Phan Anh Tiến, 12-EX-05]%[1D2Y3-1]
	Trong khai triển nhị thức $(a+2)^{n+6}$ với $n\in\mathbb{N}$ có tất cả $17$ số hạng. Vậy $n$ bằng
	\choice
	{$ 11 $}
	{$ 12 $}
	{\True $ 10 $}
	{$ 17 $}
	\loigiai{Trong khai triển của $(a+2)^{n+6}$ có tất cả $n+7$ số hạng, nên $n=10$.
		}
\end{ex}%!Cau!%
\begin{ex}%[Thi thử L2, THPT Nguyễn Trung Thiên - Hà Tĩnh, 2019]%[Nguyễn Thành Nhân,12EX7]%[1D2Y3-1]
	Khai triển nhị thức $\left(2x^2+3\right)^{16}$ có bao nhiêu số hạng? 
	\choice
	{ $16$}
	{\True  $17$}
	{ $15$}
	{$5^{16}$}
	\loigiai{Khai triển$\left(a+b\right)^n$ ta được $n+1$ số hạng. Với $n=16$ thì khai triển $\left(2x^2+3\right)^{16}$ ta được $17$ số hạng. }
\end{ex}%!Cau!%
\begin{ex}%[KTCL 12 L4, Ninh Bình - Bạc Liêu, 2018-2019]%[Vũ Nguyễn Hoàng Anh, 12EX8-19]%[1D2Y3-1]
	Trong khai triển nhị thức $(x+2)^{n+6}$ với $n\in\mathbb{N}$ có tất cả $19$ số hạng. Vậy $n$ bằng
	\choice
	{$ 11 $}
	{\True $ 12 $}
	{$ 10 $}
	{$ 19 $}
	\loigiai{Trong khai triển của $(x+2)^{n+6}$ có tất cả $n+7$ số hạng, nên $n=12$.
		}
\end{ex}