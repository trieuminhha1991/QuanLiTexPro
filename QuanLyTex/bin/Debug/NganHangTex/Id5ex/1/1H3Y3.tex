%!Cau!%
\begin{ex}%[Hải Phòng, 2018]%[Phan Anh Tiến, 12-EX-05]%[1H3Y3-2]
	Cho hình chóp $S.ABC$ có đáy $ABC$ là tam giác vuông ở $B$, $SA\perp (ABC)$. Gọi $AH$ là đường cao của tam giác $SAB$. Khẳng định nào sau đây là \textbf{sai}?
	\choice
	{$ AH\perp SC $}
	{$ AS\perp BC $}
	{\True $ AH\perp AC $}
	{$ AH\perp BC $}
	\loigiai{\immini{Ta có $SA\perp (ABC)\Rightarrow SA\perp BC$.\\
		$AH\perp (SBC)\Rightarrow \heva{&AH\perp BC\\&AH\perp SC}$.\\ Do đó, $AH\perp SC$ là sai.}
		{\begin{tikzpicture}[scale=.7, line join = round, line cap = round]
			\tikzset{label style/.style={font=\footnotesize}}
			\tkzDefPoints{0/0/A,5/0/C,2/-3/B}
			\coordinate (S) at ($(A)+(0,5)$);
			
			\coordinate (H) at ($(S)!1/3!(B)$);
			\tkzDrawPolygon(A,B,C,S)
			\tkzDrawSegments(S,B A,H)
			\tkzDrawSegments[dashed](A,C)
			\tkzDrawPoints(A,B,C,S,H)
			\tkzMarkRightAngle(A,B,C)
			\tkzMarkRightAngle(A,H,B)
			\tkzLabelPoints[above](S)
			\tkzLabelPoints[below](B)
			\tkzLabelPoints[left](A)
			\tkzLabelPoints[right](C,H)
		\end{tikzpicture}}	
	}
\end{ex}%!Cau!%
\begin{ex}%[HK2, THPT Nguyễn Huệ, Vĩnh Phúc, 2019]%[Thịnh Trần, dự án(12EX-5-2019)]%[1H3Y3-2]
	Cho hình chóp $S.ABCD$ có đáy là hình bình hành, hai đường chéo $AC$, $BD$ cắt nhau tại $O$ và $SA = SB = SC = SD$. Khi đó, khẳng định nào sau đây là \textbf{sai}?
	\choice
	{\True $AC\perp BD$}
	{$SO\perp BD$}
	{$SO\perp AC$}
	{$SO\perp (ABCD)$}
	\loigiai{
		\immini{
			\vspace*{-0.8cm}
			\begin{itemize}
				\item Vì $ABCD$ là hình bình hành nên khẳng định $AC\perp BD$ là sai.
				\item Ta có $\triangle SAC$, $\triangle SBD$ cân tại $O$ có $SO$ là trung tuyến nên $SO$ đồng thời là đường cao $\Rightarrow SO\perp AC$, $SO\perp BD$.
				\item Vì $SO\perp AC$, $SO\perp BD$ nên $SO\perp (ABCD)$.
			\end{itemize}	
		}{
			\begin{tikzpicture}[scale=1, font=\footnotesize, line join=round, line cap=round, >=stealth]
			\tikzset{label style/.style={font=\footnotesize}}
			\tkzDefPoints{0/0/D,3.5/0/C,1.5/1.5/A}
			\coordinate (B) at ($(A)+(C)-(D)$);
			\tkzInterLL(A,C)(B,D)    \tkzGetPoint{O}
			\coordinate (S) at ($(O)+(0,3.5)$);
			\tkzDrawPolygon(S,B,C,D)
			\tkzDrawSegments(S,C)
			\tkzDrawSegments[dashed](A,S A,B A,D A,C B,D S,O)
			
			\tkzDrawPoints[fill=black](D,C,A,B,O,S)
			\pgfresetboundingbox
			\tkzLabelPoints[above](S)
			\tkzLabelPoints[left](A,D)
			\tkzLabelPoints[right](B,C)
			\tkzLabelPoints[above right](O)
			\end{tikzpicture}
		}
	}
\end{ex}%!Cau!%
\begin{ex}%[Đề tập huấn, Bắc Kạn, 2018-2019]%[Cao Thành Thái, 12EX5-2019]%[1H3Y3-1]
 Trong không gian, số mặt phẳng đi qua điểm $M$ và vuông góc với đường thẳng $a$ là
 \choice
  {\True $1$}
  {$2$}
  {$0$}
  {vô số}
 \loigiai
  {
  Trong không gian, có một và chỉ một mặt phẳng đi qua điểm $M$ và vuông góc với đường thẳng $a$.
  }
\end{ex}%!Cau!%
\begin{ex}%[Đề tập huấn tỉnh Lai Châu,2019]%[Nguyễn Trung Kiên, dự án 12-EX-5-2019]%[1H3Y3-2]
	Cho hình chóp $S.ABC$ có đáy là tam giác đều, biết $SA\perp (ABC)$. Khẳng định nào sau đây là khẳng định \textbf{đúng}?
	\choice
	{$AB\perp BC$}
	{\True $SA\perp BC$}
	{$SB\perp AB$}
	{$SC\perp BC$}
	\loigiai
	{Vì $SA\perp (ABC)$ nên $SA\perp AB$, $SA\perp BC$ và $SA\perp AC$.}
\end{ex}%!Cau!%
\begin{ex}%[Đề thi thử THPTQG lần 2 THPT Thoại Ngọc Hầu, An Giang, năm 2019]%[Nguyễn Thành Khang, dự án 2019-Ex-7]%[1H3Y3-2]
	Cho hình chóp $S.ABCD$ có đáy là hình vuông $ABCD$, $SA$ vuông góc với đáy. Kẻ $AH$ vuông góc với $SB$ ($H\in SB$). Chọn mệnh đề đúng.
	\choice
	{\True $AH\perp SC$}
	{$AH\perp (SBD)$}
	{$AH\perp (SCD)$}
	{$AH\perp SD$}
	\loigiai{
		\immini{
			Ta có $\heva{&SA\perp BC \\ &AB\perp BC}\Rightarrow BC\perp (SAB)\Rightarrow BC\perp AH$.\\
			Mà $AH\perp SB$ nên $AH\perp (SBC)\Rightarrow AH\perp SC$.
		}{
			\begin{tikzpicture}[scale=0.8, font=\footnotesize, line join=round, line cap=round,>=stealth]
			\tkzInit[xmin=-0.5, xmax=4.5, ymin=0, ymax=4.5]
			\tkzClip
			\tkzDefPoints{0/0.6/D,1.2/1.3/A,3.8/1.3/B,1.2/3.8/S}
			\tkzDefPointBy[translation=from A to B](D)\tkzGetPoint{C}
			\tkzDefPointBy[homothety=center S ratio 2/5](B)\tkzGetPoint{H}
			\tkzDrawPoints[fill=black](A,B,C,D,S,H)
			\tkzDrawSegments(B,C C,D S,B S,C S,D)
			\tkzDrawSegments[dashed](A,B D,A A,H S,A)
			\tkzLabelPoints[above](S)
			\tkzLabelPoints[below](D,C,A)
			\tkzLabelPoints[right](B)
			\tkzLabelPoints[above right](H)
			\tkzMarkRightAngles(A,H,B)
			\end{tikzpicture}
		}
	}
\end{ex}%!Cau!%
\begin{ex}%[Thi thử L2, THPT Nguyễn Trung Thiên - Hà Tĩnh, 2019]%[Nguyễn Thành Nhân,12EX7]%[1H3Y3-3]
	Cho hình chóp $S.ABCD$ có đáy $ABCD$ là hình vuông cạnh $a$. Cạnh bên $SA$ vuông góc với $\left(ABCD\right)$. Góc giữa cạnh $SC$ và mặt $\left(SAD\right)$ là góc nào sau đây?
	\choice
	{$\widehat{SCA}$ }
	{$\widehat{CSA}$  }
	{$\widehat{SCD}$}
	{\True $\widehat{CSD}$}
		\loigiai{
	\immini
	{Ta có $\heva{&CD \perp SA\\&CD \perp AD \\} \Rightarrow CD\perp \left(SAD\right)$.\\
	 Vậy góc giữa $SC$ và mặt $\left(SAD\right)$ là $\widehat{CSD}$.
	}
	{
\begin{tikzpicture}[scale=1,font=\footnotesize,line join=round,line cap=round, >=stealth]
\tkzDefPoints{0/0/A,4/0/B,1.5/-1/C,0/3/S}
\tkzDefPointBy[translation = from B to A](C)\tkzGetPoint{D}
\tkzDrawSegments(S,D S,C S,B B,C C,D)
\tkzDrawSegments[dashed](A,B A,D S,A)
\tkzMarkRightAngles(B,A,S)
\tkzMarkRightAngles(A,D,C)
\tkzDrawPoints(A,B,C,D,S)
\tkzLabelPoints[left](D,S)
\tkzLabelPoints[above left=-3pt](A)
\tkzLabelPoints[below](C,B)
\end{tikzpicture}
	} 
	}
\end{ex}