%!Cau!%
\begin{ex}%[2-DTH-14-NINHBINH-19]%[Nguyễn Thế Anh, dự án EX5]%[1D2Y1-2]
	Một lớp học có $19$ bạn nữ và $16$ bạn nam. Hỏi có bao nhiêu cách chọn ra $2$ bạn, trong đó có $1$ bạn nam và $1$ bạn nữ?
	\choice
	{$35$ cách}
	{$595$ cách}
	{\True $304$ cách}
	{$1190$ cách}
	\loigiai{ Có $19$ cách chọn $1$ bạn nữ từ $19$ bạn nữ và $16$ cách chọn $1$ bạn nam từ $16$ bạn nam.\\
		Vậy số cách chọn là $19\cdot 16=304$ cách.
	}
	\end{ex}%!Cau!%
\begin{ex}%[Đề tập huấn Sở Ninh Bình, 2019]%[Nguyễn Văn Hải, dự án(12EX-5-2019)]%[1D2Y1-2]
Một lớp học có $19$ bạn nữ và $16$ bạn nam. Hỏi có bao nhiêu cách chọn ra $2$ bạn, trong đó có $1$ nam và $1$ bạn nữ?
\choice
{$35$ cách}
{$595$ cách}
{\True $304$ cách}
{$1190$ cách}
\loigiai{
Có $19$ cách chọn $1$ bạn nữ và $16$ cách chọn $1$ bạn nam.\\
Theo quy tắc nhân, số cách chọn được $1$ bạn nữ và $1$ bạn nam là $19\cdot 16=304$ cách.
}
\end{ex}%!Cau!%
\begin{ex}%[Thi thử, THPT chuyên KHTN Hà Nội, 2019]%[KV Thanh, 12EX5]%[1D2Y1-2]
	Một lớp học có $15$ nam và $10$ nữ. Số cách chọn hai học sinh trực nhật sao cho có cả nam và nữ là
	\choice
	{$300$}
	{$25$}
	{\True $150$}
	{$50$}
	\loigiai{
	Việc chọn hai học sinh trực nhật sao cho có cả nam và nữ là một việc được hoàn thành bởi $2$ bước.\\
	Bước 1. Chọn $1$ học sinh nam: Có $15$ cách.\\
	Bước 2. Chọn $1$ học sinh nữ: Có $10$ cách.\\
	Vậy số cách chọn hai học sinh trực nhật sao cho có cả nam và nữ là $15\cdot 10=150$.
	} 
\end{ex}%!Cau!%
\begin{ex}%[Thi thử, Sở GD và ĐT - BRVT, 2019]%[Phan Văn Thành, 12EX8]%[1D2Y1-2]
	Một người vào cửa hàng ăn, người đó chọn thực đơn gồm $1$ món ăn trong $7$ món, $1$ loại quả tráng miệng trong $4$ loại quả tráng miệng và một nước uống trong $5$ loại nước uống. Có bao nhiêu cách chọn thực đơn.
	\choice
	{$  16 $}
	{$  28 $}
	{\True $  140 $}
	{$  120 $}
	\loigiai{
		Việc chọn một thực đơn bao gồm $3$ công việc liên tiếp nhau
		\begin{itemize}
			\item[$\bullet$] Công việc $1$: chọn $1$ món ăn, có $7$ cách chọn.
			\item[$\bullet$] Công việc $2$: chọn $1$ loại quả tráng miệng, có $4$ cách chọn.
			\item[$\bullet$] Công việc $3$: chọn $1$ loại nước uống, có $5$ cách chọn.
		\end{itemize}
		Theo quy tắc nhân, số cách chọn một thực đơn là $7 \cdot 4 \cdot 5 = 140$ (cách).
	}
\end{ex}%!Cau!%
\begin{ex}%[Thi thử, Sở GD và ĐT - Hưng Yên-Lần 1, 2019]%[Duong Xuan Loi, 12-EX-8]%[1D2Y1-2]
	Trong tủ quần áo của bạn An có $4$ chiếc áo khác nhau và $3$ chiếc quần khác nhau. Hỏi bạn Hùng có bao nhiêu cách chọn $1$ bộ quần áo để mặc?
	\choice
	{$27$}
	{$64$}
	{$7$}
	{\True $12$}
	\loigiai{
		Số cách chọn $1$ bộ quần áo là $3\cdot 4 = 12$.
	}
\end{ex}%!Cau!%
\begin{ex}%[Thi thử THPTQG 2019 môn Toán lần 2 trường Nho Quan A – Ninh Bình,2019][Nguyễn Thành Nhân,12EX8]%[1D2Y1-2]
Trong tủ quần áo của thầy Đông có $6$ cái áo sơ mi khác màu và $5$ cái quần khác màu. Hỏi thầy Đông có tất cả bao nhiêu cách chọn ra một bộ quần áo?
	\choice
	{$5$}
	{ \True $30$}
	{ $11$}
	{$6$}
	\loigiai{ Để chọn một bộ quần áo, thầy Đông cần chọn $1$ cái quần vào $1$ cái áo.
	\begin{itemize}
	\item Số cách chọn một cái quần là $6$.
	\item Số cách chọn một cái áo là $5$.
	\item Theo quy tắc nhân ta có: $6 \times 5=30$ cách chọn một bộ quần áo.
	\end{itemize}
		}
\end{ex}