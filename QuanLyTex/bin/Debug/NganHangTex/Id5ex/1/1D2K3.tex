%!Cau!%
\begin{ex}%[DTH, Sở GD và ĐT - Hà Nam, 2019]%[Đào-V- Thủy, 12EX5]%[1D2K3-2]
	Tìm số hạng không chứa $x$ trong khai triển của $\left( x\sqrt{x}+ \dfrac{1}{x^3}\right)^n$, với $x>0$, nếu biết rằng $\mathrm{C}_n^2- \mathrm{C}_n^1= 44$.
	\choice
	{\True $165$}
	{$238$}
	{$485$}
	{$525$}
	\loigiai{
		Điều kiện $n\ge 2$, $n\in \mathbb{N}$.\\
		Ta có $\mathrm{C}_n^2- \mathrm{C}_n^1= 44\Leftrightarrow \dfrac{n(n-1)}{2}- n= 44\Leftrightarrow \hoac{&n=11&\text{(nhận)}\\ &n=-8 &\text{(loại)}}$.\\
		Với $n=11$, số hạng tổng quát trong khai triển nhị thức $\left( x\sqrt{x}+ \dfrac{1}{x^3}\right)^{11}$ là
		\[
		\mathrm{C}_{11}^2 \left( x\sqrt{x}\right)^{11-k} \left( \dfrac{1}{x^3}\right)^k= \mathrm{C}_{11}^k x^{\tfrac{33}{2}- \tfrac{11}{2}k}.
		\]
		Theo giả thiết ta có $\dfrac{33}{2}- \dfrac{11}{2}k=0$ hay $k=3$.\\
		Vậy số hạng không chứa $x$ trong khai triển đã cho là $\mathrm{C}_{11}^3= 165$.
	}
\end{ex}%!Cau!%
\begin{ex}%[Thi thử lần I, Sở GD&ĐT Sơn La 2019]%[Nguyễn Anh Quốc,  dự án EX5]%[1D2K3-2]
	Với $n$ là số nguyên dương thỏa mãn $A_n^3+2A_n^2=100$ ($A_n^k$ là số các chỉnh hợp chập $k$ của tập hợp có $n$ phần tử). Số hạng chứa $x^5$ trong khai triển của biểu thức $(1+3x)^{2n}$ là
	\choice
	{$61236$}
	{$252$}
	{$256x^5$}
	{\True $61236x^5$}
	\loigiai
	{Ta có
		\begin{eqnarray*}
		&&\mathrm{A}_n^3+2A_n^2=100\\
		&\Leftrightarrow& \dfrac{n!}{(n-3)!}+2\cdot \dfrac{n!}{(n-2)!}=100 \\
		&\Leftrightarrow& n(n-1)(n-2)+2n(n-1)=100 \\
		&\Leftrightarrow& n^3-n^2-100=0\\
		&\Leftrightarrow& n=5.
		\end{eqnarray*}
	Với $n=5$ ta có $(1+3x)^{2n}=(1+3x)^{10}=\sum\limits_{k=0}^{10} {\mathrm{C}_{10}^k{3^k}\cdot{x^k}}.$\\
		Suy ra số hạng chứa $x^5$ là $3^5\cdot \mathrm{C}_{10}^{5}=61236x^5$.}
\end{ex}%!Cau!%
\begin{ex}%[Đề tập huấn, Sở GD và ĐT - Vĩnh Phúc, 2019]%[Mai Sương, EX-5-2019]%[1D2K3-2]
Với $n$ là số nguyên dương thỏa mãn $\mathrm{C}_n^1 + \mathrm{C}_n^2= 55$, số hạng không chứa $x$ trong khai triển của biểu thức $\left(x^3 + \dfrac{2}{x^2}\right)^n$ bằng
	\choice
	{$322560$}
	{$3360$}
	{$80640$}
	{\True $13440$}
	\loigiai{
	Điều kiện $n \in \mathbb{N}^{*}, n\ge 2$.\\
	Ta có $\mathrm{C}_n^1 + \mathrm{C}_n^2= 55 
	\Leftrightarrow n + \dfrac{n(n - 1)}{2}= 55
	\Leftrightarrow n^2 + n - 110 = 0
	\Leftrightarrow \hoac{&n = 10\\&n = - 11}
	\Rightarrow n = 10.$\\
	Với $n=10$, ta có biểu thức $\left(x^3 + \dfrac{2}{x^2}\right)^n=\left(x^3 + \dfrac{2}{x^2}\right)^{10}$.\\
	Số hạng tổng quát $T_{k+1}=\mathrm{C}_{10}^k\cdot \left(x^3\right)^{10-k}\cdot \left(\dfrac{2}{x^2}\right)^k
	=\mathrm{C}_{10}^k\cdot 2^k \cdot x^{30-5k} $.\\
	Để số hạng không chứa $x$ thì $30-5k=0 \Leftrightarrow k=6.$\\
	Do đó, hệ số của số hạng không chứa $x$ trong khai triển là  $\mathrm{C}_{10}^6\cdot 2^6=13440.$
	}
	
	\end{ex}%!Cau!%
\begin{ex}%[Đề tập huấn số 2, Sở GD và ĐT Quảng Ninh, 2019]%[Đỗ Đường Hiếu, 12EX5-19]%[1D2K3-2]
	Tổng các hệ số trong khai triển $(1+x)^{3n}$ bằng $64$. Số hạng không chứa $x$ trong khai triển $\left(2nx+\dfrac{1}{2nx^2}\right)^{3n}$ là
	\choice
	{$360$}
	{$210$}
	{$250$}
	{\True $240$}
	\loigiai{
		Tổng các hệ số trong khai triển $(1+x)^{3n}$ bằng $64$ nên suy ra $(1+1)^{3n}=64\Leftrightarrow 8^n=64\Leftrightarrow n=2$.\\
		Khi đó $\left(2nx+\dfrac{1}{2nx^2}\right)^{3n}$ trở thành $\left(4x+\dfrac{1}{4x^2}\right)^6$, có số hạng tổng quát là
		$$T_{k+1}=\mathrm{C}^k_6\left(4x\right)^{6-k}\cdot\left(\dfrac{1}{4x^2}\right)^k=\mathrm{C}^k_6 4^{6-2k}x^{6-3k}.$$
		Số hạng không chứa $x$ ứng với $6-3k=0\Leftrightarrow k=2$.\\
		Số hạng đó là $\mathrm{C}^2_6 4^2=240$.
	}
\end{ex}%!Cau!%
\begin{ex}%[Đề tập huấn, Sở GD và ĐT - Quảng Ninh, 2019]%[Lê Hồng Phi, 12EX5]%[1D2K3-2]
	Tổng các hệ số nhị thức Niu-tơn trong khai triển $(1+x)^{3n}$ bằng $64$. Số hạng không chứa $x$ trong khai triển $\left(2nx+\dfrac{1}{2nx^2}\right)^{3n}$ là
	\choice
	{$360$}
	{$210$}
	{$250$}
	{\True $240$}
	\loigiai
	{Ta có khai triển $f(x)=(1+x)^{3n}=a_{3n}x^{3n}+a_{3n-1}x^{3n-1}+\cdots +a_1x+a_0$.\\
	Vì tổng các hệ số bằng $64$ nên $$a_{3n}+a_{3n-1}+\cdots +a_1+a_0=64\Leftrightarrow (1+1)^{3n}=64\Leftrightarrow 2^{3n}=2^6\Leftrightarrow 3n=6\Leftrightarrow n=2.$$
	Số hạng tổng quát trong khai triển $\left(4x+\dfrac{1}{4x^2}\right)^6$ là $\mathrm{C}_6^k\cdot (4x)^{6-k}\cdot \left(\dfrac{1}{4x^2}\right)^k=\mathrm{C}_6^k\cdot 4^{6-2k}\cdot x^{6-3k}$.\\
	Để có số hạng không chứa $x$ thì $6-3k=0\Leftrightarrow k=2$.\\
	Vậy số hạng không chứa $x$ là $\mathrm{C}_6^2\cdot 4^2=240$.	
	}
\end{ex}%!Cau!%
\begin{ex}%[Thi thử, Sở GD và ĐT - Hà Tĩnh, 2019]%[Đặng Tân Hoài, 12-EX-5-2019]%[1D2K3-2]
	Với số nguyên dương $n$ thỏa mãn $\mathrm{C}_n^2-n=27$, trong khai triển $\left(x+\dfrac{3}{x^2}\right)^n$ số hạng không chứa $x$ là 
	\choice
	{$84$}
	{\True $2268$}
	{$61236$}
	{$27$}
	\loigiai{
	Với $n \ge 2,~n \in \mathbb{N}$ ta có $$\mathrm{C}_n^2-n=27 \Leftrightarrow \dfrac{n!}{2! \times (n-2)!}-n=27 \Leftrightarrow n(n-1)-2n=54 \Leftrightarrow n^2-3n-54=0 \Leftrightarrow \hoac{& n=9~\mbox{(nhận)}\\& n=-6~\mbox{(loại)}.}$$
	Khi đó, số hạng tổng quát thứ $k+1$ trong khai triển của $\left(x+\dfrac{3}{x^2}\right)^9$ là
	\[T_{k+1}=\mathrm{C}_9^k x^{9-k} \left(\dfrac{3}{x^2}\right)^k=3^k \mathrm{C}_9^k x^{9-3k},~0 \le k \le 9, k \in \mathbb{N} \]
	Số hạng không chứa $x$ ứng với $9-3k=0 \Leftrightarrow k=3$ (thỏa mãn $0 \le k \le 9, k \in \mathbb{N}$).\\
	Vậy số hạng cần tìm là $T_4=3^3 \mathrm{C}_9^3 =2268$.
}
\end{ex}%!Cau!%
\begin{ex}%[Thi thử, Toán học tuổi trẻ, 2019-2]%[Nguyễn Trường Sơn, 12-EX-5-2019]%[1D2K3-2]
Số hạng không chứa $x$ trong khai triển $\left( 1+x+x^2+\dfrac{1}{x}\right) ^9$ bằng
	\choice
	{\True$ 13051$}
	{$13050$}
	{$13049$}
	{ $ 13048 $}
	\loigiai{
	Ta có: \begin{eqnarray*}
			\left( 1+x+x^2+\dfrac{1}{x}\right) ^9&=&\left( 1+\dfrac{1}{x}+x+x^2 \right) ^9=\left( \dfrac{(1+x)(x^2+1)}{x} \right) ^9\\&=& \dfrac{(1+x)^9(x^2+1)^9}{x^9}=\dfrac{\sum\limits_{k=0}^{9}\mathrm{C}_9^kx^k\cdot \sum\limits_{i=0}^{9} \mathrm{C}_9^i x^{2i} }{x^9}
	\end{eqnarray*}
Xét phương trình: $k+2i=9$ trong đó $i$, $k$ là hai số nguyên không âm thỏa mãn $0 \le i,k \le 9$. Dễ dàng suy ra $k$ là số tự nhiên lẻ. Ta có bảng:
\begin{center}
	\begin{tabular}{|c|c|c|c|c|c|}
		\hline
		$k$& $1$& $3$ & $5$& $7$&$9$\\
		\hline 
		$i$&$4$&$3$&$2$&$1$&$0$\\
		\hline
	
	\end{tabular}
\end{center}
	Hệ số của số hạng chứa $x^9$ trong khai triển của $(x+1)^9 \cdot (x^2+1)^9$ bằng $$\mathrm{C}_9^1 \cdot \mathrm{C}_9^4+\mathrm{C}_9^3 \cdot \mathrm{C}_9^3+\mathrm{C}_9^5 \cdot \mathrm{C}_9^2+\mathrm{C}_9^7 \cdot \mathrm{C}_9^1+\mathrm{C}_9^9 \cdot \mathrm{C}_9^0=13051.$$
	Vậy số hạng không chứa $x$ bằng $13051$.
	}
\end{ex}%!Cau!%
\begin{ex}%[Thi thử, Toán học tuổi trẻ, 2019-2]%[Nguyễn Trường Sơn, 12-EX-5-2019]%[1D2K3-3]
	Giá trị của tổng $1+2^2\mathrm{C}_{99}^2+2^4\mathrm{C}_{99}^4+\cdots+2^{98}\mathrm{C}_{99}^{98}$ bằng
	\choice
	{$\dfrac{3^{99}}{2}$}
	{$\dfrac{3^{99}+1}{2}$}
	{$3^{99}$}
	{\True $\dfrac{3^{99}-1}{2}$}
	\loigiai{Xét khai triển $(1+x)^{99} = \mathrm{C}_{99}^0+\mathrm{C}_{99}^1x+\mathrm{C}_{99}^2x^2+ \cdots +\mathrm{C}_{99}^{99} x^{99}$.\\
		Thay $x=2$ vào khai triên ta có $3^{99} = \mathrm{C}_{99}^0+\mathrm{C}_{99}^12+\mathrm{C}_{99}^22^2+ \cdots +\mathrm{C}_{99}^{99} 2^{99}$.\\
		Thay $x=-2$ vào khai triển ta có $-1 = \mathrm{C}_{99}^0-\mathrm{C}_{99}^12+\mathrm{C}_{99}^22^2- \cdots - \mathrm{C}_{99}^{99} 2^{99}$.\\
		Vậy $1+2^2\mathrm{C}_{99}^2+2^4\mathrm{C}_{99}^4+\cdots+2^{98}\mathrm{C}_{99}^{98}=\dfrac{3^{99}-1}{2}$
		
	}
\end{ex}%!Cau!%
\begin{ex}%[Thi thử L1, THPT Quỳnh Lưu 2, Nghệ An, 2019]%[Nguyễn Quang Tân, dự án 12-EX-8-2019]%[1D2K3-3]
	Giải phương trình $ \mathrm{C}_{1}^{n}+3\cdot\mathrm{C}_{2}^{n}+7\cdot\mathrm{C}_{3}^{n}+ \cdots + \left(2^n-1\right)\cdot
	\mathrm{C}_{n}^{n}=3^{2n}-2^n-6480 $ trên tập $ \mathbb{N^{*}} $.
	\choice
	{$ n=3 $}
	{\True $ n=4 $}
	{$ n=5 $}
	{$ n=6 $}
	\loigiai{
		Xét khai triển $ \left(1+x\right)^n=	\mathrm{C}_{0}^{n}+x\mathrm{C}_{1}^{n}+x^2\mathrm{C}_{2}^{n}+ \cdots + x^n\mathrm{C}_{n}^{n} $.\\
		Thay $ x=2 $ ta có $ 3^n=\mathrm{C}_{0}^{n}+2\mathrm{C}_{1}^{n}+2^2\mathrm{C}_{2}^{n}+ \cdots + 2^n\mathrm{C}_{n}^{n}\quad (1) $.\\
		Thay $ x=1 $ ta có $ 2^n=\mathrm{C}_{0}^{n}+\mathrm{C}_{1}^{n}+\mathrm{C}_{2}^{n}+ \cdots + \mathrm{C}_{n}^{n}\quad (2) $.\\
		Trừ vế theo vế của $ (1) $ cho $ (2) $ thì $ \mathrm{C}_{1}^{n}+3\mathrm{C}_{2}^{n}+7\mathrm{C}_{3}^{n}+ \cdots + \left(2^n-1\right)\mathrm{C}_{n}^{n} = 3^n - 2^n $.\\
		Khi đó $ 3^n - 2^n = 3^{2n}-2^n-6480 \Leftrightarrow 3^n=81 \Leftrightarrow n=4 $. 
	}
\end{ex}%!Cau!%
\begin{ex}%[Nguyễn Tài Tuệ, Đề Thi THPT QG lần 4 trường THPT Yên Khánh A, Ninh Bình, Dự án 12EX8-2019]%[1D2K3-3]
	Cho tập A có $20$ phần tử. Có bao nhiêu tập con của A khác rỗng và số phần tử là số chẵn?
	\choice
	{$2^{20}-1$}
	{\True $2^{19}-1$}
	{$2^{19}$}
	{$2^{20}$}
	\loigiai{
		Ta có số tập con của A khác rỗng và số phần tử là số chẵn là $\mathrm{C}_{20}^2+ \cdots +\mathrm{C}_{20}^{20}$.\\
		Xét khai triển $(1+x)^{20}=\mathrm{C}_{20}^0+\mathrm{C}_{20}^1x+ \cdots +\mathrm{C}_{20}^{20}x^{20}$.\\
		Cho $x=1$, ta được: $2^{20}=\mathrm{C}_{20}^0+\mathrm{C}_{20}^1+ \cdots +\mathrm{C}_{20}^{20}$. \quad $(1)$\\
		Cho $x=-1$, ta được $0=\mathrm{C}_{20}^0-\mathrm{C}_{20}^1+\mathrm{C}_{20}^2- \cdots +\mathrm{C}_{20}^{20}$.\quad $(2)$\\
		Cộng vế theo vế $(1)$  và $(2)$, ta được: $\mathrm{C}_{20}^0+\mathrm{C}_{20}^2+ \cdots +\mathrm{C}_{20}^{20}=2^{19} \Rightarrow \mathrm{C}_{20}^2+ \cdots +\mathrm{C}_{20}^{20}=2^{19}-1$.}
\end{ex}%!Cau!%
\begin{ex}%[Thi thử, THPT Lê Hồng Phong - Nam Định, 2019, lần 1]%[Nguyễn Minh Hiếu, 12EX9]%[1D2K3-2]
	Tìm số hạng không chứa $x$ trong khai triển $\left(x^2-\dfrac{2}{x}\right)^{15}$.
	\choice
	{$ 2^7\cdot \mathrm{C}_{15}^7 $}
	{\True $ 2^{10}\cdot \mathrm{C}_{15}^{10}  $}
	{$ -2^{10}\cdot \mathrm{C}_{15}^{10}  $}
	{$ -2^7\cdot \mathrm{C}_{15}^7  $}
	\loigiai{
		Ta có $\left(x^2-\dfrac{2}{x}\right)^{15}=\displaystyle \sum\limits_{k=0}^{15}\mathrm{C}_{15}^k\left(x^2\right)^{15-k}\left(-\dfrac{2}{x}\right)^k= \displaystyle \sum\limits_{k=0}^{15}\mathrm{C}_{15}^k(-2)^kx^{30-3k}$.\\
		Số hạng không chứa $x$ tương ứng với số hạng chứa $k$ thỏa mãn $30-3k=0\Leftrightarrow k=10$.\\
		Vậy số hạng không chứa $x$ trong khai triển đã cho là $(-2)^{10}\cdot \mathrm{C}_{15}^{10}=2^{10}\cdot \mathrm{C}_{15}^{10}$.
	}
\end{ex}