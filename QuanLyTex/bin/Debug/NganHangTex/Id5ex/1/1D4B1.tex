%!Cau!%
\begin{ex}%[Thi thử, Sở GD và ĐT - Quảng Nam, 2019]%[Nguyện Ngô, 12EX8]%[1D4B1-3]
Cho cấp số cộng $(u_n)$ có số hạng đầu $u_1=2$ và công sai $d=3$. Tìm $L=\lim\dfrac{n}{u_n}$.
\choice
{\True $\dfrac{1}{3}$}
{$\dfrac{1}{2}$}
{$3$}
{$2$}
\loigiai{
\begin{itemize}
\item Số hạng tổng quát $u_n=u_1+(n-1)d=2+(n-1)3=3n-1$.
\item Ta có $L=\lim\dfrac{n}{u_n}=\lim\dfrac{n}{3n-1}=\lim\dfrac{1}{3-\dfrac{1}{n}}=\dfrac{1}{3}$.
\end{itemize}
}
\end{ex}%!Cau!%
\begin{ex}%[Thi thử, THPT Phan Đình Phùng - Đắc Lắc, 2019]%[Nguyễn Minh Hiếu, 12EX8]%[1D4B1-3]
	$\lim\left(\dfrac{1}{n^2}+\dfrac{2}{n^2}+\cdots+\dfrac{n}{n^2}\right)$ bằng
	\choice
	{$ 1 $}
	{\True $ \dfrac{1}{2} $}
	{$ \dfrac{1}{3} $}
	{$ 0 $}
	\loigiai{
		Ta có $\dfrac{1}{n^2}+\dfrac{2}{n^2}+\cdots+\dfrac{n}{n^2}=\dfrac{1+2+\cdots+n}{n^2}=\dfrac{\dfrac{n(n+1)}{2}}{n^2}=\dfrac{n+1}{2n}$.\\
		Do đó $\lim\left(\dfrac{1}{n^2}+\dfrac{2}{n^2}+\cdots+\dfrac{n}{n^2}\right)=\lim\dfrac{n+1}{2n}=\lim\dfrac{n\left(1+\dfrac{1}{n}\right)}{2n}=\dfrac{1}{2}$.
	}
\end{ex}