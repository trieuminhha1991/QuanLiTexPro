%!Cau!%
\begin{ex}%[Thi thử, Toán học tuổi trẻ, 2019-2]%[Nguyễn Trường Sơn, 12-EX-5-2019]%[1H3K1-4]
	Cho tứ diện $SABC$ có trọng tâm $G$. Một mặt phẳng qua $G$ cắt các tia $SA$, $SB$, $SC$ theo thứ tự tại $A'$, $B'$, $C'$. Đặt $\dfrac{SA'}{SA}=m$, $\dfrac{SB'}{SB}=n$, $\dfrac{SC'}{SC}=p$. Đẳng thức nào dưới đây là đúng
		\choice
	{$\dfrac{1}{m^2}+\dfrac{1}{n^2}+\dfrac{1}{p^2}=4$}
	{$\dfrac{1}{mn}+\dfrac{1}{np}+\dfrac{1}{pm}=4$}
	{\True $\dfrac{1}{m}+\dfrac{1}{n}+\dfrac{1}{p}=4$}
	{$m+n+p=4$}
	\loigiai{
			\immini{
			Gọi $n$ là trọng tâm tam giác $ABC$.\\ Ta có: $3\overrightarrow{SN}=\overrightarrow{SA}+\overrightarrow{SB}+\overrightarrow{SC}$.\\
			Do $G$ là trọng tâm tứ diện nên $\overrightarrow{SN}=\dfrac{4}{3}\overrightarrow{SG}$.\\
			Vậy ta có $$4\overrightarrow{SG}=\overrightarrow{SA}+\overrightarrow{SB}+\overrightarrow{SC}=\dfrac{SA}{SA'}\overrightarrow{SA'}+\dfrac{SB}{SB'}\overrightarrow{SB'}+\dfrac{SC}{SC'}\overrightarrow{SC'}.$$
			Do $A'$, $B'$, $C'$, $G$ đồng phẳng nên ta có: $\dfrac{SA}{SA'}+\dfrac{SB}{SB'}+\dfrac{SC}{SC'}=4$ hay $\dfrac{1}{m}+\dfrac{1}{n}+\dfrac{1}{p}=4$.
		}{
				\begin{tikzpicture}[scale=1, font=\footnotesize, line join=round, line cap=round, >=stealth]]
			\tkzDefPoints{0/0/A,1.2/-1.6/B,4.5/0/C}
			\tkzCentroid(A,B,C)\tkzGetPoint{N}
			\coordinate (S) at ($(N)+(0,3.5)$);
			\coordinate (M) at ($(B)!1/2!(C)$);
			\coordinate (G) at ($(S)!0.75!(N)$);
			\coordinate (A') at ($(S)!0.7!(A)$);
			\coordinate (B') at ($(S)!0.8!(B)$);
			\tkzInterLL(A',G)(S,M)\tkzGetPoint{I}
			\tkzInterLL(B',I)(C,S)\tkzGetPoint{C'}
			\tkzDrawPolygon(A,B,C,S)
			\tkzDrawSegments(S,B A',B' B',C' S,M)
			\tkzDrawSegments[dashed](M,A A,C S,N A',C' A',I)
			\tkzDrawPoints[fill=black,size=4](A,B,C,S,N,M,G,A',B',C')
			\tkzLabelPoints[above](S)
			\tkzLabelPoints[below](B,N,M)
			\tkzLabelPoints[left](A,A',B')
			\tkzLabelPoints[right](C,G,C')
			\end{tikzpicture}
		}\noindent	
	}
\end{ex}%!Cau!%
\begin{ex}%[Thi thử, Toán học tuổi trẻ, 2019-2]%[Nguyễn Trường Sơn, 12-EX-5-2019]%[1H3K1-4]
	Cho hình chóp tam giác $S.ABC$ có $SA=a$, $SB=b$, $SC=c$ và $\widehat{BSC}=120^\circ$, $\widehat{CSA}=90^\circ$, $\widehat{ASB}=60^\circ$. Gọi $G$ là trọng tâm tam giác $ABC$. Độ dài của đoạn $SG$ bằng
	\choice
	{$\dfrac{1}{3}\sqrt{a^2+b^2+c^2+ab+bc+ca}$}
	{ $\sqrt{a^2+b^2+c^2+ab-bc}$}
	{$\dfrac{1}{3}\sqrt{a^2+b^2+c^2+ab-ca}$}
	{\True $\dfrac{1}{3}\sqrt{a^2+b^2+c^2+ab-cb}$}
	\loigiai{
		Ta có \begin{eqnarray*}
			&&3\overrightarrow{SG}=\overrightarrow{SA}+\overrightarrow{SB}+\overrightarrow{SC}\\&\Rightarrow& 9SG^2=SA^2+SB^2+SC^2+2\overrightarrow{SA}\cdot \overrightarrow{SB}+2\overrightarrow{SA} \cdot
			\overrightarrow{SC}+2\overrightarrow{SB} \cdot \overrightarrow{SC}\\&\Rightarrow& 9SG^2	=a^2+b^2+c^2+2ab \cos 60^\circ	+2bc \cos 120^\circ=a^2+b^2+c^2+ab-bc \\&\Rightarrow& SG=\dfrac{1}{3}\sqrt{a^2+b^2+c^2+ab-bc} 
		\end{eqnarray*} 
	}
\end{ex}