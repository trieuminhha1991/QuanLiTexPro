%!Cau!%
\begin{ex}%[Thi thử, Lào Cai - Phú Thọ, 2019]%[Bùi Anh Tuấn, dự án (12EX-5)]%[1H3Y4-1]
	Mệnh đề nào dưới đây đúng?
	\choice
	{\True Hình chóp đều là hình chóp có đáy là đa giác đều và các cạnh bên bằng nhau}
	{Hình chóp có đáy là tam giác đều là hình chóp đều}
	{Hình lăng trụ có đáy là một đa giác đều là hình lăng trụ đều}
	{Hình lăng trụ tứ giác đều là hình lập phương}
	\loigiai{Hình chóp đều là hình chóp có đáy là đa giác đều và các cạnh bên bằng nhau.}
\end{ex}%!Cau!%
\begin{ex}%[Thi thử, Sở GD và ĐT - Hưng Yên-Lần 1, 2019]%[Duong Xuan Loi, 12-EX-8]%[1H3Y4-3] 
	Cho hình lập phương $ABCD.A'B'C'D'$. Góc giữa hai mặt phẳng $(ABCD)$ và $(A'B'C'D')$ bằng bao nhiêu?
	\choice
	{$45^{\circ}$}
	{$90^{\circ}$}
	{\True $0^{\circ}$}
	{$60^{\circ}$}
	\loigiai{
		Hai mặt phẳng $(ABCD)$ và $(A'B'C'D')$ song song nên góc giữa chúng bằng $0^{\circ}$.
	}
\end{ex}%!Cau!%
\begin{ex}%[Thi thử, Toán Học và Tuổi Trẻ (Đề số 3), 2019]%[Đặng Tân Hoài, 12-EX-6-2019]%[1H3Y4-1]
	Khẳng định nào sau đây đúng?	
	\choice
	{\True Hai đường thẳng phân biệt cùng vuông góc với một mặt phẳng thì song song với nhau}
	{Hai mặt phẳng song song khi và chỉ khi góc giữa chúng bằng $0^{\circ}$}
	{Hai đường thẳng trong không gian cắt nhau khi và chỉ khi góc giữa chúng lớn hơn $0^{\circ}$ và nhỏ hơn $90^{\circ}$ }
	{Hai mặt phẳng phân biệt cùng vuông góc với một mặt phẳng thì song song với nhau}
	\loigiai{
		Hai đường thẳng phân biệt cùng vuông góc với một mặt phẳng thì song song với nhau.	
	}
\end{ex}