%!Cau!%
\begin{ex}%[De tap huan, So GD&DT Dien Bien, 2019]%[Ngoc Diep, dự án EX5]%[1D3B4-3]
	Cho cấp số nhân $(u_n)$ có $u_5 =2$ và $u_9=6$. Tính $u_{21}$.
	\choice
	{$18$}
	{$54$}
	{\True $162$}
	{$486$}
	\loigiai{
		Ta có $\heva{&u_1 \cdot q^4 =2\\ & u_1 \cdot q^8 =6 } \Rightarrow q^4 =3$.\\
		Lại có $u_{21} = u_1 \cdot q^{20} = u_1 \cdot q^8 \cdot q^{12} = u_9 \cdot (q^4)^3 = 6 \cdot 27 =162$.
	}
\end{ex}%!Cau!%
\begin{ex}%[Đề ĐK 12 Nguyễn Khuyến, HCM, ngày 24 tháng 03 năm 2019]%[Vinh Vo, 12EX7-2019]%[1D3B4-1]
	Phương trình $ x^2 - 3x + a = 0 $ có hai nghiệm $ x_1, x_2 $ và phương trình $ x^2 - 12x + b = 0 $ có hai nghiệm $ x_3, x_4 $. Giả sử rằng $ x_1, x_2, x_3, x_4 $  theo thứ tự lập thành cấp số nhân với công bội lớn hơn $ 1 $. Giá trị của $ a + b $ là
	\choice
	{$ 13 $}
	{$ 29 $}
	{\True $ 34 $}
	{$ 37 $}
	\loigiai{
	Gọi $ q > 1 $ là công bội.\\
	Ta có $ \heva{& x_1 + x_2 = 3\\ & x_3 + x_4 = 12} \Rightarrow \heva{& x_1(1 + q) = 3  \\ & x_1q^2(1+q) = 12 } \Rightarrow q^2 = 4 \Rightarrow q = 2$.\\
	Ta được $\heva{& x_1 = 1\\& x_2 = 2\\ & x_3 = 4 \\ & x_4 = 8} \Rightarrow \heva{&a = 2\\ & b = 32} \Rightarrow a + b = 34$.	
}
\end{ex}%!Cau!%
\begin{ex}%[Thi thử L1, THPT Hậu Lộc 2, Thanh Hoá, 2019]%[Dương Phước Sang, 12EX-5-2019]%[1D3B4-3]
	Cho cấp số nhân $(u_n)$ có số hạng đầu $u_1=3$ và công bội $q=2$. Giá trị của $u_4$ bằng
	\choice
	{\True $24$}
	{$54$}
	{$48$}
	{$9$}
	\loigiai{
		Áp dụng công thức số hạng tổng quát của cấp số nhân, ta có: $u_4=u_1q^3=3\cdot 2^3=24$.}
\end{ex}%!Cau!%
\begin{ex}%[Thi thử, THPT Chuyên Ngoại Ngữ - Hà Nội, 2019]%[Trần Nhân Kiệt, 12EX7-2019]%[1D3B4-3]
Số $1458$ là số hạng thứ bao nhiêu của cấp số nhân $(u_n)$ có số hạng đầu $u_1=2$ và công bội $q=3$?
	\choice
	{$8$}
	{$5$}
	{$6$}
	{\True $7$}
	\loigiai{
Ta có $u_n=u_1\cdot q^{n-1}=2\cdot 3^{n-1}$.\\
Xét $u_n=1458\Leftrightarrow 2\cdot 3^{n-1}=1458\Leftrightarrow 3^{n-1}=729\Leftrightarrow 3^{n-1}=3^6\Leftrightarrow n=7$.\\
Vậy số $1458$ là số hạng thứ $7$ của cấp số nhân $(u_n)$.
	}
\end{ex}%!Cau!%
\begin{ex}%[Thi thử L1, THPT Quỳnh Lưu 2, Nghệ An, 2019]%[Nguyễn Quang Tân, dự án 12-EX-8-2019]%[1D3B4-1]
	Trong các dãy số $(u_n)$ sau đây, dãy số nào là cấp số nhân?
	\choice
	{$u_n = 2n$}
	{\True  $u_n = 2 \cdot (-3)^{2n+1}$}
	{$u_n = 2^n - 1$}
	{$u_n = \dfrac{1}{n}$}
	\loigiai{Ta thấy với $\forall n \ge 2$, $n \in \mathbb{N}$ dãy số $u_n = 2 \cdot (-3)^{2n+1}$ có $\dfrac{u_n}{u_{n-1}} = \dfrac{2 \cdot (-3)^{2n+1}}{2 \cdot (-3)^{2(n-1)+1}}=  \dfrac{2 \cdot (-3)^{2n+1}}{2 \cdot (-3)^{2n - 1}} = 9$ nên $(u_n)$ là cấp số nhân với công bội $q = 9$, $u_1 = -54$. }
\end{ex}%!Cau!%
\begin{ex}%[Thi thử  L2, trường Phúc Trạch-Hà Tĩnh, 2019]%[Lê Hồng Phi, 12EX8]%[1D3B4-2]
Cho cấp số nhân có số hạng đầu $u_1=2$  và số hạng thứ $11$ là  $u_{11}=\dfrac{1}{512}$. Tìm công bội $q$ của cấp số nhân, biết $q>0$.
\choice
{$q=\dfrac{1}{4}$}
{$q=2$}
{$q=\dfrac{1}{3}$}
{\True $q=\dfrac{1}{2}$}
\loigiai{
Ta có $u_{11}=u_1\cdot q^{10}\Leftrightarrow \dfrac{1}{512}=2\cdot q^{10}\Leftrightarrow q^{10}=\dfrac{1}{1024}\Leftrightarrow q=\dfrac{1}{2}$ (vì $q>0$).\\
Vậy công bội của cấp số nhân là $q=\dfrac{1}{2}$.}
\end{ex}%!Cau!%
\begin{ex}%[Thi thử L1, Chuyên Nguyễn Trãi, Hải Dương, 2019]%[Đinh Thanh Hoàng, dự án EX6]%[1D3B4-2]
	Cho một cấp số nhân $(u_n)$ có $u_1=\dfrac{1}{4}$, $u_4=\dfrac{1}{4^4}$. Số hạng tổng quát bằng
	\choice
	{\True $\dfrac{1}{4^n}$, $n\in \mathbb{N}^*$}
	{$\dfrac{1}{n^4}$, $n\in \mathbb{N}^*$}
	{$\dfrac{1}{{4}^{n+1}}$, $n\in \mathbb{N}^*$}
	{$\dfrac{1}{4n}$, $n\in \mathbb{N}^*$}
	\loigiai{
		Dãy $(u_n)$ là một cấp số nhân với công bội $q$, từ giả thiết ta có
			$$u_4=u_1q^3\Leftrightarrow q^3=\dfrac{u_4}{u_1}\Leftrightarrow q^3=\dfrac{1}{4^3}\Leftrightarrow q=\dfrac{1}{4}.$$
		Vậy số hạng tổng quát của cấp số nhân là $u_n=u_1q^{n-1}=\dfrac{1}{4}\cdot\left(\dfrac{1}{4}\right)^{n-1}=\left(\dfrac{1}{4}\right)^n=\dfrac{1}{4^n}$, $n\in \mathbb{N}^*$.
	}
\end{ex}%!Cau!%
\begin{ex}%[Thi thử, Trần Đại Nghĩa - Đắc Lắk, 2019]%[Trần Nhân Kiệt, 12EX8-2019]%[1D3B4-3]
	Cho cấp số nhân $\left(u_n\right)$ có số hạng đầu $u_1=2$ và $u_4=54.$ Giá trị $u_{2019}$ bằng
	\choice
	{$2\cdot 2^{2018}$}
	{$2\cdot 3^{2020}$}
	{\True $2\cdot 3^{2018}$}
	{$2\cdot 2^{2020}$}
\loigiai{
Ta có $u_4=u_1\cdot q^3\Leftrightarrow 54=2\cdot q^3\Leftrightarrow q=3$.\\
Số hạng $u_{2019}=u_1\cdot q^{2018}=2\cdot 3^{2018}$.
}
\end{ex}%!Cau!%
\begin{ex}%[Thi thử L2, Thanh Chương 1 Nghệ An, 2019]%[Nguyễn Văn Nay, dự án EX8]%[1D3B4-5]
	Cho cấp số nhân $(u_n)$ có số hạng đầu $u_1=3$, $q=2$. Tổng năm số hạng đầu của cấp số nhân là
	\choice
	{\True $S_5=93$}
	{$S_5=11$}
	{$S_5=96$}
	{$S_5=48$}
	\loigiai{
		Ta có $S_5=u'\cdot\dfrac{q^5-1}{q-1}=3\cdot\dfrac{2^5-1}{1}=93$.	
	}
\end{ex}%!Cau!%
\begin{ex}%[Đề thi thử Sở GD-ĐT Quảng Bình - 2019]%[Dự án 12-EX9-2019, Nguyễn Anh Quốc]%[1D3B4-3]
Cho cấp số nhân $\left( u_n\right)$ có $u_1=1$, $u_2=-2$. Giá trị $u_{2019}$ bằng	
	\choice
	{$u_{2019}=-2^{2018}$}
	{\True $u_{2019}=2^{2018}$}
	{$u_{2019}=-2^{2019}$}
	{$u_{2019}=2^{2019}$}
	\loigiai{ Gọi $q$ là công bội của cấp số nhân khi đó ta có $$u_2=u_1\cdot q\Rightarrow -2=1\cdot q	\Rightarrow q=-2.$$
	Suy ra $u_{2019}=u_1\cdot q^{2018}=1\cdot (-2)^{2018}=2^{2018}.$
	}
\end{ex}%!Cau!%
\begin{ex}%[Thi thử, Toán Học và Tuổi Trẻ (Đề số 3), 2019]%[Đặng Tân Hoài, 12-EX-6-2019]%[1D3B4-2]
	Cho đoạn thẳng $AB=2^{100}$ (cm). Gọi $M_1$ là trung điểm của $AB$. Gọi $M_{k+1}$ là trung điểm của $M_kB$ ($k=1,2,\ldots ,99$). Tính độ dài đoạn thẳng $M_1M_{100}$.
	\choice
	{\True $2^{99}-1$ (cm)}
	{$2^{97}+1$ (cm)}
	{$2^{99}-2$ (cm)}
	{$2^{98}$ (cm)}
	\loigiai{
		Ta có $M_1M_k=M_1B-M_kB=\dfrac{AB}{2}-\dfrac{AB}{2^k}=\dfrac{AB}{2^k}(2^{k-1}-1),~k=1,2,\ldots ,100$.\\
		Với $k=100$ thì $M_1M_{100}=2^{99}-1$.
	}
\end{ex}%!Cau!%
\begin{ex}%[Thi Thử Lần 2, THPT Lương Thế Vinh - Hà Nội, 2019]%[Dương BùiĐức, dự án 12EX6]%[1D3B4-3]
Cho cấp số nhân $(u_n)$ thỏa mãn $\heva{&u_1+u_3=10\\ &u_4+u_6=80}$. Tìm $u_3$.
\choice
{$u_3=6$}
{$u_3=2$}
{\True $u_3=8$}
{$u_3=4$}
\loigiai{
Ta có $\heva{&u_1+u_3=10\\ &u_4+u_6=80}\Leftrightarrow \heva{&u_1(1+q^{2})=10\\ &u_4(1+q^{2})=80}\Rightarrow \dfrac{u_{4}}{u_{1}}=8\Rightarrow u_{3}=8$.
}
\end{ex}