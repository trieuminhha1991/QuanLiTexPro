%!Cau!%
\begin{ex}%[Thi thử, Sở GD và ĐT -Lạng Sơn, 2019]%[Trần Duy Khương, 12EX5-2019]%[1D3Y4-1]
		Cho cấp số nhân $(u_n)$ có $u_1=-2,u_2=10$. Công bội $q$ của cấp số nhân này là 
		\choice
		{\True $q=-5$}
		{$q=8$}
		{$q=-12$}
		{$q=12$}
		\loigiai{Công bội $q$ của cấp số nhân này là $q=\dfrac{u_2}{u_1}=-5$.}
	\end{ex}%!Cau!%
\begin{ex}%[Thi thử, Lào Cai - Phú Thọ, 2019]%[Bùi Anh Tuấn, dự án (12EX-5)]%[1D3Y4-3]
	Cho cấp số nhân $\left(u_n\right)$ có số hạng đầu $u_1=5$ và công bội $q=-2$. Số hạng thứ sáu bằng
	\choice
	{$160$}
	{$-320$}
	{\True $-160$}
	{$320$}
	\loigiai{Ta có $u_6=u_1\cdot q^5=5\cdot(-2)^5=-160$.}
\end{ex}%!Cau!%
\begin{ex}%[Đề tập huấn, Bắc Kạn, 2018-2019]%[Cao Thành Thái, 12EX5-2019]%[1D3Y4-5]
 Cho cấp số nhân $u_1,u_2,u_3,\ldots,u_n$ với công bội $q$ ($q\neq 0$, $q\neq 1$). Đặt \[S_n=u_1+u_2+u_3+\cdots +u_n.\] Khẳng định nào sau đây là đúng?
 \choice
  {$S_n = \dfrac{u_1\left(q^n+1\right)}{q+1}$}
  {\True $S_n = \dfrac{u_1\left(q^n-1\right)}{q-1}$}
  {$S_n = \dfrac{u_1\left(q^{n-1}-1\right)}{q+1}$}
  {$S_n = \dfrac{u_1\left(q^{n-1}-1\right)}{q-1}$}
 \loigiai
  {
  Ta có $S_n=u_1+u_2+u_3+\cdots +u_n = u_1\cdot \dfrac{1-q^n}{1-q} = \dfrac{u_1\left(q^n-1\right)}{q-1}$.
  }
\end{ex}%!Cau!%
\begin{ex}%[Thi thử, Sở GD và ĐT - Bình Phước lần 1, 2019]%[Lê Thanh Nin, 12EX7]%[1D3Y4-3]
	Cho cấp số nhân $(u_n)$ có số hạng đầu $u_1=\dfrac{1}{2}$ và công bội $q=2$. Giá trị của $u_{25}$ bằng
	\choice
	{\True $2^{23}$}	
	{$2^{24}$}
	{$2^{25}$}
	{$2^{26}$}
	\loigiai{
		Ta có $u_{25}=u_1q^{24}=\dfrac{1}{2}\cdot 2^{24}=2^{23}.$}
\end{ex}%!Cau!%
\begin{ex}%[Đề thi thử Chuyên Sơn La lần 1, Sơn La, 2018-2019]%[Cao Thành Thái, dự án 12-EX-7-2019]%[1D3Y4-3]
 Cho cấp số nhân $(u_n)$ có số hạng đầu $u_1=3$ và công bội $q=2$. Giá trị của $u_5$ bằng
 \choice
  {$162$}
  {$11$}
  {$96$}
  {\True $48$}
 \loigiai
  {
  Ta có $u_5=u_1\cdot q^4=3\cdot 2^4=48$.
  }
\end{ex}%!Cau!%
\begin{ex}%[Dự án EX-7-2019]%[Phạm Tuấn]%[1D3Y4-3]
Cho cấp số nhân $(u_n)$  với $u_1=3$, $u_2 =-6$. Giá trị $u_5$ bằng 
\choice
{$u_5=-24$}
{\True $u_5 =48$}
{$u_5=24$}
{$u_5 =-48$}
\loigiai{
Ta có công bội $q= \dfrac{u_2}{u_1} = -2$. Do đó $u_5 = u_1q^4 =48$. 
}
\end{ex}%!Cau!%
\begin{ex}%[Thi thử L2, THPT Nguyễn Trung Thiên - Hà Tĩnh, 2019]%[Nguyễn Thành Nhân,12EX7]%[1D3Y4-3]
	Cho dãy số $\left(u_n\right)$ thỏa mãn $\heva{&u_1=1\\&u_n=3u_{n-1},\forall n>1 \\}$. Giá trị của $u_5$ bằng
	\choice
	{\True $81$}
	{ $243$  }
	{$729$}
	{$15$}
	\loigiai{ Dãy $\left(u_n\right)$ là một cấp số nhân với công bội $q=3$. Do đó số hạng tổng quát của dãy là $$u_n=u_1\cdot q^{n-1}=3^{n-1}.$$
Vậy $u_5=3^4=81$.}
\end{ex}%!Cau!%
\begin{ex}%[Thi thử, trường THPT Nguyễn Công Trứ, tỉnh Hà Tĩnh, 2019]%[Phạm Doãn Lê Bình, 12EX7-19]%[1D3Y4-3]
	Cho cấp số nhân $(u_n)$ có số hạng đầu $u_1=3$ và công bội $q=2$. Giá trị của $u_5$ bằng
	\choice
	{$11$}
	{$96$}
	{$24$}
	{\True $48$}
	\loigiai{
	Ta có $u_5 = u_1 \cdot q^4 = 3 \cdot 2^4 = 48$.
	}
\end{ex}%!Cau!%
\begin{ex}%[Đề KSCL lớp 12 môn Toán Sở giáo dục và đào tạo Thanh Hóa, 2018-2019]%[Cao Thành Thái, dự án 12-EX-8-2019]%[1D3Y4-3]
 Cho cấp số nhân $(u_n)$ có số hạng đầu $u_1=3$ và công bội $q=2$. Giá trị của $u_4$ bằng
 \choice
  {\True $24$}
  {$48$}
  {$18$}
  {$54$}
 \loigiai
  {
  Ta có $u_4 = u_1q^3 = 3 \cdot 2^3 = 24$.
  }
\end{ex}%!Cau!%
\begin{ex}%[Thi thử, Sở GD và ĐT - Điện Biên, 2019]%[Tô Ngọc Thy, dự án EX8]%[1D3Y4-3]
	Cho cấp số nhân $\left(u_n\right)$ có $u_1=3$, $q=\dfrac{-1}{2}$. Khi đó $\dfrac{3}{256}$ là số hạng thứ mấy?
	\choice
	{Thứ $8$}
	{\True Thứ $9$}
	{Thứ $7$}
	{Thứ $6$}
	\loigiai{
		Ta có $u_n=u_1\cdot q^{n-1}\Leftrightarrow\dfrac{3}{256}=3\cdot\left(-\dfrac{1}{2}\right)^{n-1}\Leftrightarrow n=9$.}
\end{ex}%!Cau!%
\begin{ex}%[Thi thử, Sở GD và ĐT - Hà Tĩnh, 2019]%[Nguyễn Anh Tuấn, 12-EX8-19]%[1D3Y4-1]
	Ba số nào sau đây tạo thành một cấp số nhân?
	\choice
	{\True $ -1;2;-4 $}
	{$ 1;2;-4 $}
	{$ -1;2;4 $}
	{$ 1;-2;-4 $}
	\loigiai{
		Xét mỗi bộ ba số, ta có
		\begin{itemize}
			\item Với $ -1;2;-4 $, ta có $ \dfrac{2}{-1}=\dfrac{-4}{2} $ nên bộ ba số này là một cấp số nhân với công bội $ q=-2 $.
			\item Với $ 1;2;-4 $, ta có $ \dfrac{2}{1} \ne \dfrac{-4}{2} $ nên bộ ba số này không là một cấp số nhân.
			\item Với $ -1;2;4 $, ta có $ \dfrac{2}{-1} \ne \dfrac{4}{2} $ nên bộ ba số này không là một cấp số nhân.
			\item Với $ 1;-2;-4 $, ta có $ \dfrac{2}{-1} \ne \dfrac{-4}{-2} $ nên bộ ba số này không là một cấp số nhân.
		\end{itemize}
	}
\end{ex}%!Cau!%
\begin{ex}%[TT, THPT Chuyên Hà Tĩnh, 19]%[Trần Bá Huy, 12-EX-8-2019]%[1D3Y4-5]
Cho cấp số nhân $(u_n)$ có số hạng đầu $u_1=3$, công bội $q=-2$. Tính tổng $10$ số hạng đầu tiên của cấp số nhân $(u_n)$.
\choice
{$-513$}
{\True $-1023$}
{$513$}
{$1023$}
\loigiai{
Tổng của $10$ số hạng đầu bằng
$$S_{10}=u_1\cdot\dfrac{q^{10}-1}{q-1}=3\cdot\dfrac{(-2)^{10}-1}{-2-1}=-1023.$$
}
\end{ex}%!Cau!%
\begin{ex}%[Thi thử, Kinh Môn - Hải Dương, 2019]%[Lê Vũ Hải, 12EX8]%[1D3Y4-1]
	Cho cấp số nhân $\left(u_{n}\right)$, với $u_{1}=-3$, $u_{4}=\dfrac{1}{9}$. Công bội của cấp số nhân đã cho bằng
	\choice
	{$ -3 $}
	{$ 3 $}
	{\True $ -\dfrac{1}{3} $}
	{$ \dfrac{1}{3} $}
	\loigiai{
		Gọi $q$ là công bội của cấp số nhân trên, ta có $u_{4} = u_{1} \cdot q^{3} \Leftrightarrow \dfrac{1}{9} = (-3)\cdot q^{3} \Leftrightarrow q= -\dfrac{1}{3}$.
	}
\end{ex}%!Cau!%
\begin{ex}%[Dự án EX-8 2019]%[Phạm Tuấn]%[1D3Y4-3]
Cho cấp số nhân $\left (u_n\right )$ có số hạng đầu $u_1=1$ và công bội $q = 3$. Giá trị của $u_5$ là
\choice
{$13$}
{$162$}
{$16$}
{\True $81$}
\loigiai{
Ta có $u_5 = u_1q^4 = 3^4 = 81$.
}
\end{ex}%!Cau!%
\begin{ex}%[Thi thử L2, Chuyên Lê Quý Đôn - Đà Nẵng, 2019]%[Đinh Thanh Hoàng, 12-EX-8-2019]%[1D3Y4-3]
	Cho cấp số nhân $(u_n)$ có số hạng đầu $u_1=3$ và công bội $q=2$. Giá trị của $u_5$ bằng
	\choice
	{$24$}
	{$96$}
	{\True $48$}
	{$162$}
	\loigiai{
		Vì $\left(u_n\right)$ là cấp số nhân với công bội $q$ nên $u_n=u_1{q}^{n-1}\Rightarrow u_5=u_1q^4=3\cdot 2^4=48$.
	}
\end{ex}%!Cau!%
\begin{ex}%[TT, THPT Kim Liên, Hà Nội-L2]%[Nguyễn Quang Dũng, dự án 12 EX-8-2019]%[1D3Y4-3]
Cho cấp số nhân $\left(u_n\right)$ có số hạng đầu $u_1=3$ và công bội $q=-2$. Giá trị của $u_4$ bằng
\choice
{$24$}
{\True $-24$}
{$48$}
{$-3$}
\loigiai{
Ta có $u_4=u_1\times q^{4-1}=3\times (-2)^{3}=-24$.}
\end{ex}%!Cau!%
\begin{ex}%[Thi thử THPTQG 2019 môn Toán lần 2 trường Nho Quan A – Ninh Bình,2019]%[Nguyễn Thành Nhân,12EX8]%[1D3Y4-1]
Cho cấp số nhân $(u_n)$ có $u_1=81$ và $u_2=9$. Gọi $q$ là công bội của cấp số nhân đó. Đáp án nào sau đây là \textbf{đúng}?
	\choice
	{$-9$}
	{  $-\dfrac{1}{9}$}
	{ $9$}
	{ \True $\dfrac{1}{9}$}
	\loigiai{ Công bội $q$ được xác định:  $q=\dfrac{u_2}{u_1}=\dfrac{9}{81}=\dfrac{1}{9}$.}
\end{ex}%!Cau!%
\begin{ex}%[Thi thử L2, Sở GD&DT Phú Thọ, 2019]%[Trần Nhân Kiệt, dự án EX9]%[1D3Y4-3]
	Cho cấp số nhân $(u_n)$ có $u_1=3$ và có công bội $q=\dfrac{1}{4}$. Giá trị của $u_3$ bằng
	\choice
	{$\dfrac{3}{8}$}
	{\True $\dfrac{3}{16}$}
	{$\dfrac{16}{3}$}
	{$\dfrac{3}{4}$}
	\loigiai{
		Ta có $u_3=u_1\cdot q^2=3\cdot \left(\dfrac{1}{4}\right)^2=\dfrac{3}{16}$.
	}
\end{ex}%!Cau!%
\begin{ex}%[Thi thử, THPT Lê Hồng Phong - Nam Định, 2019, lần 1]%[Nguyễn Minh Hiếu, 12EX9]%[1D3Y4-3]
	Cho cấp số nhân $\left(u_n\right)$ có hai số hạng đầu tiên là $u_1=-3$ và $u_2=9$. Công bội của cấp số nhân đã cho bằng
	\choice
	{$ -81 $}
	{$ 81 $}
	{$ 3 $}
	{\True $ -3 $}
	\loigiai{
		Từ công thức $u_n=u_1q^{n-1}$, ta có $u_2=u_1q\Leftrightarrow q=\dfrac{u_2}{u_1}=-3$.
	}
\end{ex}%!Cau!%
\begin{ex}%[Đề-thi-thử-THPT-Quốc-gia-2019-môn-Toán-hội-các-trường-chuyên-lần-3]%[Nguyễn Thành Nhân,12EX8]%[1D3Y4-3]
Cho cấp số nhân $(u_n)$ có số hạng đầu $u_1=3$ và số hạng $u_2=-6$. Giá trị $u_4$ bằng
	\choice
	{$12$}
	{\True $-24$}
	{ $-12$}
	{$24$}
	\loigiai{ Công bội của cấp số nhân là $q=\dfrac{-6}{3}=-2$. \\
	Số hạng $u_4=u_1\cdot q^3=3\cdot (-2)^3=-24$. }
\end{ex}%!Cau!%
\begin{ex}%[12-EX-ĐHVinh-L3]%[Nguyễn Quang Tân]%[1D3Y4-3] 
	Cho cấp số nhân $(u_n)$ có $u_1=2$, $u_2=-1$. Mệnh đề nào sau đây đúng?
	
	\choice
	{$u_{2019}=-2^{2019}$}
	{\True $u_{2019}=\dfrac{1} {2^{2017}}$}
	{$u_{2019}=2^{2019}$}
	{$u_{2019}=-\dfrac{1} {2^{2017}}$}
	\loigiai{
		Cấp số nhân có $u_1=2$, $u_2=-1$ nên công bội $ q=-\dfrac{1}{2}$. Vậy $u_{2019}=u_1\cdot q^{2018}=2\cdot \left(-\dfrac{1}{2}\right)^{2018}=\dfrac{1} {2^{2017}}$.}
\end{ex}%!Cau!%
\begin{ex}%[12-EX-ĐHVinh-L3]%[Nguyễn Quang Tân]%[1D3Y4-3] 
	Cho cấp số nhân $(u_n)$ có $u_1=3$, $u_2=-9$. Mệnh đề nào sau đây đúng?
	\choice
	{$u_{2019}=3^{2018}$}
	{ $u_{2019}=(-3)^{2019}$}
	{\True $u_{2019}=3^{2019}$}
	{$ u_{2019}=-3^{2018}$}
	\loigiai{
		Ta có công bội $q=\dfrac{u_2}{u_1}=-3$.
		Vậy $u_{2019}=u_1\cdot q^{2018}=3(-3)^{2018}=3^{2019}$.}
\end{ex}%!Cau!%
\begin{ex}%[Đề dự đoán số 2]%[Nguyễn Thành Khang, dự án 2019-12-Ex-8]%[1D3Y4-5]
	Cho cấp số nhân $(u_n)$ có số hạng đầu $u_1=12$ và công sai $q=\dfrac{3}{2}$. Tổng $5$ số hạng đầu của cấp số nhân bằng
	\choice
	{$\dfrac{93}{4}$}
	{\True $\dfrac{633}{4}$}
	{$\dfrac{633}{2}$}
	{$\dfrac{93}{2}$}
	\loigiai{
		Gọi $S_5$ là tổng $5$ số hạng đầu của cấp số nhân đã cho. Khi đó ta có $$S_5=u_1\cdot \dfrac{1-q^5}{1-q}=12\cdot \dfrac{1-\left(\dfrac{3}{2}\right)^5}{1-\dfrac{3}{2}}=\dfrac{633}{4}.$$
	}
\end{ex}%!Cau!%
\begin{ex}%[Phát triển đề số 5]%[Đoàn Minh Tân, EX8]%[1D3Y4-3]
	Cho cấp số nhân $(u_n)$ có số hạng đầu $u_1=2$ và công bội $q=2$. Giá trị của $u_6$ bằng
	\choice
	{$32$}
	{$96$}
	{$128$}
	{\True $64$}
	\loigiai{
		Ta có $u_6 = u_1 \cdot q^5 = 2 \cdot 2^5 = 64$.
	}
\end{ex}